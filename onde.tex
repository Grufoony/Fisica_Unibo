\documentclass[a4paper]{article}
\usepackage[T1]{fontenc}
\usepackage[utf8]{inputenc}
\usepackage[main=italian, english]{babel}
\usepackage{bookmark}
\usepackage[a4paper, total={6in, 9in}]{geometry}
\usepackage{hyperref}
\usepackage{amssymb}
\usepackage{amsfonts}
\usepackage{amsmath}
\newcommand*{\field}[1]{\mathbb{#1}}

\usepackage{geometry}
\geometry{a4paper, top=1.5cm, bottom=2cm, left=2cm, right=2cm}

\begin{document}
	\title{Formulario Fenomeni Ondulatori}
	\author{Marco Caporale}
	\date{}
	\maketitle
\section{Matematica}
	\begin{itemize}
		\item $z = r e^{i \varphi} = x + iy = r(\cos \varphi + i \sin \varphi)$ Rappresentazioni del numero complesso $z$
		\item $z^*=\overline{z} = r(\cos \varphi - i \sin \varphi)$ Complesso coniugato
		\item $z_1 z_2 = r_1 r_2 e^{i (\varphi_1 +\varphi_2)}$ Prodotto fra numeri complessi
		\item $z_1/z_2 = \frac{r_1}{r_2} e^{i (\varphi_1 -\varphi_2)}$ Rapporto fra numeri complessi
		\item $z^n = r^n e^{in \varphi} = r^n [\cos n\varphi + i \sin n \varphi]$ Potenza di numero complesso
		\item $(\cos \alpha + i \sin \alpha)^n = (\cos n \alpha + i \sin n \alpha)$ Identità di de Moivre
		\item $x = \Re (z)$ Parte reale del complesso
		\item $\cos \varphi = \frac{e^{i \varphi}+e^{-i \varphi}}{2}$ Rappresentazioni complesse di seno e coseno\\
		$\sin \varphi = \frac{e^{i \varphi}-e^{-i \varphi}}{2i}$
		\item Operatore lineare $\hat{L}(x)$\\
		$\rightarrow \forall x,y; \hspace{3mm} \hat{L}(x+y)=\hat{L}(x) +\hat{L}(y)$\\
		$\rightarrow \forall x, \forall a; \hspace{2mm} \hat{L}(ax)= a\hat{L}(x)$
		\item Integrali notevoli di seni e coseni; $T$ periodo; $n,m \in \field{N}$ \\
		$\int_{0}^{T} \cos(n \omega t) \cos (m \omega t)dt = \frac{T}{2} \delta_{nm}$\\
		$\int_{0}^{T} \sin(n \omega t) \sin (m \omega t)dt = \frac{T}{2} \delta_{nm}$\\
		$\int_{0}^{T} \sin(n \omega t) \cos (m \omega t)dt = 0$
		\item Serie di Fourier, $f(t)$ limitata, periodica di periodo $T$\\
		$f_N(t)=a_0+\sum_{n=1}^{+ \infty} [a_n \cos n \omega t+b_n \sin n \omega t]$\\
		$a_0 = \frac{1}{T} \int_{0}^{T} f(t) dt$\\
		$a_n = \frac{2}{T} \int_{0}^{T} f(t) \cos n \omega t dt$\\
		$b_n = \frac{2}{T} \int_{0}^{T} f(t) \sin n \omega t dt$
		\item Serie complessa di Fourier\\
		$f(t)=\sum_{n=-\infty}^{+ \infty} c_n e^{in \omega t}$\\
		$c_n=\frac{1}{T} \int_{0}^{T} f(t) e^{-in \omega t}dt$
		\item Trasformata e antitrasformata di Fourier\\
		$\tilde{f}(\omega)=\int_{-\infty}^{\infty} f(t) e^{-i \omega t} dt$\\
		$f(t)=\frac{1}{2 \pi} \int_{-\infty}^{\infty} \tilde{f}(\omega) e^{i \omega t} d\omega$
		
		
	\end{itemize}
	
\section{Oscillazioni Lineari}
	%\subsection{Introduzione}
	\begin{itemize}
		\item $\overrightarrow{F} = -k \overrightarrow{x}$ Legge di Hooke
		\item $\ddot{x} + \omega_0^2 x = 0 $ Moto armonico semplice\\
		$x(t) = l \cos (\omega_0 t + \varphi_0)$ Equazione oraria\\
		$l = \sqrt{x_0^2 + \frac{v_0^2}{\omega_0^2}} $\\
		$\varphi_0 = -\arctan(\frac{v_0}{\omega_0 x_0})$ \\
		$T = \frac{2 \pi}{\omega_0} = 2 \pi \sqrt{\frac{m}{k}}$\\
		(Nel caso del circuito LC avremo $\omega_0 = \frac{1}{\sqrt{LC}}$)\\
		$z(t)=c_1 e^{i \omega_0 t }+ c_2 e^{-i \omega_0 t}$ Soluzione dell'equazione differenziale
		\item $\ddot{x} + \frac{\beta}{m} \dot{x} + \frac{k}{m}x$ Oscillatore armonico smorzato\\ %Vedere i moti smorzati
		$\Delta = \frac{\beta}{m}^2-4\frac{k}{m}$\\
		$\lambda_{1,2}=\frac{1}{2}(-\frac{\beta}{m}\pm \sqrt{\Delta})$\\ \\
		$\rightarrow \Delta > 0$ Moto sovrasmorzato\\
		$x(t)=c_1e^{-|\lambda_1|t}+c_2e^{-|\lambda_2|t}$\\ \\
		$\rightarrow \Delta=0$ Moto smorzato critico $\lambda_1=\lambda_2=-\frac{\beta}{2m}= -\eta$\\
		$x(t)=(c_1+c_2t)e^{-\eta t}$\\ \\
		$\rightarrow \Delta <0$ Moto oscillatorio smorzato $\lambda_{1,2}=\frac{1}{2}(-\frac{\beta}{m}+i\sqrt{|\Delta|})=-\eta \pm i \omega$\\
		$x(t)=c_1e^{-\eta t+i \omega t}+c_2e^{-\eta t-i \omega t}=e^{-\eta t}(D_1 \cos \omega t + D_2 \sin \omega t)=Ae^{-\eta t} \cos(\omega t + \phi_0)$\\
		$x(t)=A_0e^{-\frac{t}{2 \tau}} \cos(\omega t + \varphi_0)$ avendo usato $\tau = \frac{1}{\gamma}=\frac{m}{\beta}$
		\item Oscillazioni forzate  $x(t)=x_{omo}(t)+x_{part}(t)$\\
		Forzante periodica\\
		$A=\frac{F_0/m}{(-\Omega^2+\omega_0^2)+i\gamma \Omega}=\frac{F_0}{\chi}=\frac{F_0}{i \Omega Z_m}$ ampiezza oscillazione da forzante periodica di pulsazione $\Omega$\\
		$\delta = \arctan\frac{\gamma \Omega}{\omega^2_0-\Omega^2}$\\
		$z(t)=\frac{F_0/m}{\sqrt{(\omega_0^2-\Omega^2)^2+(\gamma \Omega)^2}}e^{i(\Omega t-\delta)}$\\
		$x(t)=\Re(z(t))=\frac{F_0/m}{\sqrt{(\omega_0^2-\Omega^2)^2+(\gamma \Omega)^2}} \cos(\Omega t-\delta)$\\
		$\dot{x(t)}=\Re(\dot{z(t)})=-\frac{F_0 \Omega}{m\sqrt{(\omega_0^2-\Omega^2)^2+(\gamma \Omega)^2}} \sin(\Omega t-\delta)$\\
		Andamenti caratteristici della soluzione stazionaria (particolare)\\
		$\rightarrow \Omega << \omega_0$\\
		$A\simeq \frac{F_0}{m \omega_0^2}= \frac{F_0}{k}$; $\delta \simeq0$\\
		$x(t)=\frac{F_0}{k}\cos\Omega t$ in fase con la forzante\\
		$\rightarrow \Omega >> \omega_0$\\
		$A\simeq \frac{F_0}{m \Omega^2}$; $\delta \simeq \pi$\\
		$x(t)=-\frac{F_0}{m \Omega^2}\cos\Omega t$ in opposizione di fase con la forzante\\
		$\rightarrow \Omega \approx \omega_0$\\
		$A\simeq \frac{F_0}{\beta \Omega}$; $\delta \simeq \frac{\pi}{2}$\\
		$x(t)=\frac{F_0}{\beta \omega_0}\sin\Omega t$ in quadratura di fase con la forzante\\ %non sono sicuro dell'omega a denominatore
		\item $Z_m=\frac{f(t)}{\dot{z}(t)}$ Impedenza meccanica
		\item $R(\Omega)=\frac{(\gamma \Omega)^2}{(\omega^2_0-\Omega^2)^2+(\gamma \Omega)^2}$ Funzione di risposta
		\item $Q=\frac{\omega_0}{\gamma}$ Fattore di qualità\\
		RLC serie $Q = \frac{1}{R} \sqrt{\frac{L}{C}}$,\hspace{5mm} RLC parallelo $Q = R \sqrt\frac{C}{L}$
		\item $x(t)= \frac{F_0/m}{\sqrt{(\omega_0^2 -\Omega^2)^2 + (\gamma \Omega)^2}} \cos(\Omega t-\delta) = |A| \cos(\Omega t -\delta)=A_{el} \cos \Omega t + A_{ass} \sin\Omega t$\\
		$A_{el}(\Omega)=\frac{F_0 (\omega_0^2-\Omega^2)}{m[(\omega_0^2-\Omega^2)^2+(\gamma \Omega)^2]}$ Ampiezza elastica\\
		$A_{ass}(\Omega)=\frac{F_0 \gamma \Omega}{m[(\omega_0^2-\Omega^2)^2+(\gamma \Omega)^2]}$ Ampiezza assorbitiva\\
		
	\end{itemize}

\section{Onde Meccaniche}
 \begin{itemize}
 	\item $\frac{\partial^2 \xi}{\partial x^2}=\frac{1}{v^2} \frac{\partial^2 \xi}{\partial t^2}$ Equazione delle onde di D'Alembert
 	\item $\frac{\partial^2 \xi}{\partial x^2}=\frac{\mu}{T} \frac{\partial^2 \xi}{\partial t^2}$ Onda su corda, dati $\mu$ densità della corda e $T$ tensione della corda
 	\item $\xi(x,t)=A \sin(kx \mp \omega t)$ Onde Armoniche\\
 	$\xi(x,t)=A \cos(kx \mp \omega t)= \Re(Ae^{i(kx \mp \omega t)})$
 	\item $v=\sqrt{\frac{T}{\mu}}$ Velocità delle onde meccaniche su corda
 	\item $v_f = \frac{\omega}{k} (=v)$ Velocità di fase, $v$ è quella di eq. D'Alembert
 	\item $\lambda = \frac{2 \pi}{\omega}v=v T_p$ Relazione lunghezza d'onda $\lambda$, periodo $T_p$, pulsazione $\omega$ e velocità $v$
 	\item $\omega(k)=f(k)$ Relazione di dispersione\\
 	$\omega(k) = v_f k$ Mezzi non dispersivi $\rightarrow$ eq. di D'Alembert è esatta\\
 	$\omega(k) \neq v_f k$ Mezzi dispersivi $\rightarrow$ eq. di D'Alembert prima approssimazione
 	\item $\xi(x,t)=\sum_{n=-\infty}^{+\infty}[c_ne^{i(k_nx-\omega_nt)}+d_ne^{i(k_nx+\omega_nt)}]$\\
 	Generalizzazione serie di Fourier complessa con onde progressive e regressive
 	\item \textbf{Onde progressive su corda}
 	\item $\frac{\partial \xi}{\partial t}=-v \frac{\partial \xi}{\partial x}$ Relazione derivate parziali per onda progressiva
 	\item $F_y = -T \frac{\partial \xi}{\partial x}|_{x=0}= +\frac{T}{v} \frac{\partial \xi (0,t)}{\partial t}$ Deformazione all'inizio della corda
 	\item $Z=\frac{F_y(t)}{v_c(t)}$ Impedenza meccanica della corda\\
 	$Z=\frac{T}{v}=T \sqrt{\frac{\mu}{T}}=\sqrt{T \mu}$
 	\item $P(x,t)=-T \frac{\partial \xi}{\partial x} \frac{\partial \xi}{\partial t}=-T \frac{\partial \xi(x,t)}{\partial x} \frac{\partial \xi(x,t)}{\partial t}$ Potenza nella corda\\
 	$P(x,t)=-T \frac{\partial \xi}{\partial x} \frac{\partial \xi}{\partial t}=\frac{T}{v}(\frac{\partial \xi(x,t)}{\partial t})^2 =Tv(\frac{\partial \xi(x,t)}{\partial x})^2$\\
 	$P(x,t)=Z(\frac{\partial \xi(x,t)}{\partial t})^2$ Formula valida per le onde meccaniche
 	\item $u_K(x,t)=\frac{1}{2}\mu(\frac{\partial \xi}{\partial t})^2$ Densità lineare di energia cinetica\\
 	$u_P(x,t)=\frac{1}{2}T(\frac{\partial \xi}{\partial x})^2$ Densità lineare di energia potenziale\\
 	$u_K(x,t)=u_P(x,t)$\\
 	$u(x,t) = u_K(x,t)+u_P(x,t) =2u_K(x,t)=2u_P(x,t)$ Densità lineare energia meccanica onda\\
 	$P(x,t)=v u(x,t)$ Legame potenza ed energia trasmessa
 	\item Onde di trasporto Potenza ed Energia per onde meccaniche\\
 	$\frac{\partial^2 P(x,t)}{\partial x^2}=\frac{1}{v^2}\frac{\partial^2P(x,t)}{\partial t^2}$\\
 	$\frac{\partial^2 u(x,t)}{\partial x^2}=\frac{1}{v^2}\frac{\partial^2u(x,t)}{\partial t^2}$\\
 	Mediate sul periodo\\
 	$\langle P(x,t) \rangle =\frac{1}{T_P} \int_{0}^{T_P}P(x,t)dt=\frac{1}{2}Z \omega^2 A^2 \neq 0$\\
 	$\langle u(x,t) \rangle =\frac{1}{T_P} \int_{0}^{T_P}u(x,t)dt=\frac{1}{2}\mu \omega^2 A^2 \neq 0$\\
 	Energia trasportata in un periodo ($v T_P=\lambda, \lambda \mu =m$)\\
 	$E=\int_{0}^{T_P}P(x,t)dt=\frac{1}{2}m \omega^2 A^2$ \hspace{5mm} (Uguale a quella dell'oscillatore armonico)
 	\item Intensità per onde periodiche\\
 	$I=\frac{E(\Delta t)}{\Delta t}$\\
 	Per corda ed enti unidimensionali\\
 	$I=\frac{1}{T_P}\int_{0}^{T_P}P(x,t)dt=\langle P \rangle= \langle Z(\frac{\partial \xi (x,t)}{\partial t})^2 \rangle $ ($=\frac{1}{2}Z\omega^2 A^2$ Per onda armonica)\\
 	$I=\frac{1}{T_P} \int_{0}^{T_P} P(x,t) d t= \frac{1}{2} Z \sum_{n=-1}^{+\infty} \omega_n^2 (a_n^2+b_n^2)$\\
 	$I=\sum_{n=1}^{+\infty} I_n$ con $I_n=\frac{1}{2}Z\omega_n^2(a_n^2+b_n^2)$; $I_n$ spettro di potenza dell'onda
 	\item Onde su corda con discontinuità di Z ($Z_1\neq Z_2$), $i$ onda incidente, $t$ trasmessa, $r$ riflessa\\
 	$A_t=\frac{2Z_1}{Z_1+Z_2}A_i$\\
 	$A_r=\frac{Z_1-Z_2}{Z_1+Z_2}A_i$\\
 	$I_i=I_t+I_r$ Conservazione energia\\
 	$T=\frac{I_t}{I_i}=\frac{4Z_1 Z_2}{(Z_1+Z_2)^2}$ Coefficiente di trasmissione\\
 	$R=\frac{I_r}{I_i}=(\frac{Z_1-Z_2}{Z_1+Z_2})^2$ Coefficiente di riflessione
 	
 	\item \textbf{Onde acustiche} 
 	\item $\rho(x,t)-\rho_0 =-\rho_0(\frac{\partial \xi}{\partial x})_x $ Variazione di densità al passaggio dell'onda di compressione
 	\item $\frac{\partial^2 \xi}{\partial x^2}=\frac{\rho_0}{\gamma P_0} \frac{\partial^2 \xi}{\partial t^2}$ Eq. D'Alembert per onda acustica
 	\item $\sqrt{\frac{\gamma P_0}{\rho_0}}$ Velocità onda acustica
 	\item $\frac{\partial^2 \rho}{\partial x^2}=\frac{\rho_0}{\gamma P_0} \frac{\partial^2 \rho}{\partial t^2}$ Onda di densità associata\\
 	$\frac{\partial^2 p}{\partial x^2}=\frac{\rho_0}{\gamma P_0} \frac{\partial^2 p}{\partial t^2}$ Onda di pressione associata
 	\item $Z=\frac{\gamma P_0}{v}=\sqrt{\gamma P_0 \rho_0}$ Impedenza acustica specifica\\
 	$\delta p=Z \frac{\partial \xi}{\partial t}$ Relazione causa-effetto fra passaggio dell'onda e variazione di pressione
 	\item $I=\langle\frac{\mathcal{P}}{S}\rangle=\langle Z(\frac{\partial \xi}{\partial t})^2\rangle$ Intensità dell'onda acustica (attenzione ha udm diversa da onda su corda)
 	\item $\beta_s =10\log_{10}\frac{I}{I_0}$ Livello sonoro in deciBel
 	\item Interferenza per onde con stesso $\omega$ (\textit{Onde coerenti}); ampiezze $A_1$, $A_2$; seconda onda sfasata di $\varphi_0$\\ \\
 	$\xi(x,t)=\xi_1(x,t)+\xi_2(x,t)=\Re(Ae^{i\varphi_0}e^{i(kx-\omega t)})$\\
 	con $A=|A_1+A_2 e^{i\varphi_0}|=\sqrt{A_1^2+A_2^2+2A_1A_2\cos\varphi_0}$\\
 	$2A_1A_2\cos\varphi_0$ \hspace{0.4cm} Termine d'interferenza\\
 	$I=I_1+I_2+2\sqrt{I_1I_2}\cos\varphi_0$ \hspace{0.4cm} Intensità onda risultante
 	\item Battimenti\\
 	Onde con stessa ampiezza ($A$); pulsazioni simili ($\omega_1 \approx \omega_2$); numeri d'onda simili ($k_1 \approx k_2$)\\
 	$\Delta k = \frac{k_2-k_1}{2} \hspace{0.6cm} \Delta \omega = \frac{\omega_2 - \omega_1}{2} \hspace{0.6cm} k_0 = \frac{k_2+k_1}{2} \hspace{0.6cm} \omega_0 = \frac{\omega_2 + \omega_1}{2}$\\
 	$\xi(x,t)=2A\cos(\Delta k x-\Delta \omega t)\cos(k_0 x-\omega_0t)$\\
 	$2A\cos(\Delta k x-\Delta \omega t)$ \hspace{0.4cm} Componente \textit{modulante}\\
 	$\cos(k_0 x-\omega_0t)$ \hspace{0.4cm} Componente \textit{portante}\\
 	$v_f=\frac{\omega_0}{k_0}$ \hspace{0.4cm} Velocità di fase\\
 	$v_g=\frac{\Delta \omega}{\Delta k} =\frac{d\omega}{dk}$ \hspace{0.4cm} Velocità di gruppo\\
 	Per mezzi non dispersivi $v_f=v_g$; per mezzi dispersivi $v_f\neq v_g$ (generalmente osserviamo $v_g<v_f$)
 	\item Effetto doppler (S sorgente, R ricevente; in avvicinamento)\\ 
 	$f_R=f_m(1+\frac{v_R}{v_m})=f_S\frac{1+\frac{v_R}{v_m}}{1-\frac{v_S}{v_m}}=f_S \frac{v_m + v_R}{v_m-v_S}$ \\
 	(dove $v_m$ velocità onda nel mezzo)
 	\item Generalizzazione in  3D\\
 	$\nabla^2 \xi(\overrightarrow{r},t)=\frac{1}{v^2}\frac{\partial^2 \xi(\overrightarrow{r},t)}{\partial t^2}$ D'Alembert\\
 	$\xi(\overrightarrow{r},t)=f(\hat{u}  \cdot \overrightarrow{r}-vt)$ Onde piane\\
 	$\xi(\overrightarrow{r},t)=A\cos(k\hat{u}\cdot \overrightarrow{r}-\omega t)$ Onde armoniche piane (vale principio sovrapposizione)\\
 	$\overrightarrow{k}=k \hat{u}$ Vettore d'onda
 	$k=\frac{\omega}{v}$ Numero d'onda
 	\item Onde sferiche (passaggio alla rappresentazione polare) $(x,y,z,t)\rightarrow(r,\theta,\varphi,t)$\\
 	$\xi(\overrightarrow{r},t)=\xi(r,\theta,\varphi,t)=\frac{f(r\mp vt)}{r}Y^m_l(\theta, \varphi)$\\
 	$f(r-vt)=A\cos (kr-\omega t+\alpha)$ Onde armoniche sferiche\\
 	Periodicità temporale $T_p =\frac{2\pi}{\omega}$\\
 	Periodicità spaziale scala con $\frac{1}{r}$, abbiamo pseudoperiodo $\lambda = \frac{2\pi}{k}$\\
 	\item Onde piane vettoriali
 	\begin{equation*}
 		\begin{cases}
 			 & \text{$\xi_x(z,t)=f_x(z-vt)$}\\
 			 & \text{$\xi_y(z,t)=f_y(z-vt)$}\\
 			 & \text{$\xi_z(z,t)=f_z(z-vt)$}
 		\end{cases}
 	\end{equation*}
 	Per onda generica $\overrightarrow{\xi}(\overrightarrow{r},t)=\overrightarrow{f}(\hat{u}\cdot \overrightarrow{r} - vt)$\\
 	Onda trasversale $\hat{u} \cdot \overrightarrow{f}(\hat{u}\cdot \overrightarrow{r} - vt) = 0$\\
 	Onda longitudinale $\hat{u} \times \overrightarrow{f}(\hat{u}\cdot \overrightarrow{r} - vt) = 0$\\
 	\item Polarizzazione lineare
 	\begin{equation*}
 		\begin{cases}
 			& \text{$\xi_x(z,t)=f_x(z-vt)=A_x f(z-vt)$}\\
 			& \text{$\xi_y(z,t)=f_y(z-vt)=A_y f(z-vt)$}\\
 			& \text{$\xi_z(z,t)=f_z(z-vt)=0$}
 		\end{cases} \iff \frac{\xi_x(z,t)}{\xi_y(z,t)}=\frac{A_x}{A_y}
 	\end{equation*} 
 	\item Polarizzazione circolare ed ellittica
 	\begin{equation*}
 		\begin{cases}
 			& \text{$\xi_x(z,t)=f_x(z-vt)=A_x \cos(kz-\omega t)$}\\
 			& \text{$\xi_y(z,t)=f_y(z-vt)=A_y \cos(kz-\omega t +\beta)$}\\
 			& \text{$\xi_z(z,t)=f_z(z-vt)=0$}
 		\end{cases} \iff ((\frac{\xi_x}{A_x})^2+(\frac{\xi_y}{A_y})^2-2\cos(\beta \frac{\xi_x}{A_x} \frac{\xi_y}{A_y}) = \sin^2{\beta})
 	\end{equation*} 
 Casi \begin{itemize}
 	\item $\beta = 0 \Rightarrow \frac{\xi_y}{A_y}=\frac{\xi_x}{A_x}$ Polarizzazione lineare
 	\item $\beta=\pm \pi/2 \Rightarrow ((\frac{\xi_x}{A_x})^2+(\frac{\xi_y}{A_y})^2=1)$ Assi paralleli agli assi coordinati
 	\item $\beta=\pm \pi/2; A_x =A_y \Rightarrow ((\frac{\xi_x}{A_x})^2+(\frac{\xi_y}{A_y})^2=1)$ Polarizzazione circolare
 \end{itemize}
Polarizzazione destrorsa $-\pi \leq \beta \leq 0$ (Vedendolo arrivare, antiorario)\\
Polarizzazione sinistrorsa $0 \leq \beta \leq \pi$ (Vedendolo arrivare, orario)
\item Huygens-Fresnel\\
$f(\theta) =\frac{1}{2}(1+\cos\theta)$ Fattore correttivo per la propagazione
\item Interferenza onde sonore\\
$I=I_1+I_2+2\sqrt{I_1I_2}\cos \varphi_0=4 I_0 \cos^2 \frac{\varphi}{2}$
 		
 \end{itemize}	


\section{Onde Elettromagnetiche}
\begin{itemize}
	\item Equazioni di Maxwell (forma locale)
	\begin{equation*}
		\begin{cases}
			& \nabla \cdot \overrightarrow{E}=\frac{\rho}{\varepsilon_0}\\
			& \nabla \times \overrightarrow{E}=-\frac{\partial \overrightarrow{B}}{\partial t}\\
			& \nabla \cdot \overrightarrow{B} = 0\\
			& \nabla \times \overrightarrow{B} = \mu_0 \overrightarrow{J}+\mu_0\varepsilon_0\frac{\partial \overrightarrow{E}}{\partial t}
		\end{cases}
	\end{equation*} 
Nel vuoto 
\begin{equation*}
	\begin{cases}
		& \nabla \cdot \overrightarrow{E}=0\\
		& \nabla \times \overrightarrow{E}=-\frac{\partial \overrightarrow{B}}{\partial t}\\
		& \nabla \cdot \overrightarrow{B} = 0\\
		& \nabla \times \overrightarrow{B} =\mu_0\varepsilon_0\frac{\partial \overrightarrow{E}}{\partial t}
	\end{cases} \iff(\overrightarrow{B}=\mu_0\overrightarrow{H})
	\begin{cases}
		& \nabla \cdot \overrightarrow{E}=0\\
		& \nabla \times \overrightarrow{E}=-\mu_0\frac{\partial \overrightarrow{H}}{\partial t}\\
		& \nabla \cdot \overrightarrow{H} = 0\\
		& \nabla \times \overrightarrow{H} =\varepsilon_0\frac{\partial \overrightarrow{E}}{\partial t}
	\end{cases}
\end{equation*} 
Che contengono le Eq di D'Alembert
\begin{itemize}
	\item[$\ast$] $\nabla^2\overrightarrow{E}=\mu_0 \varepsilon_0 \frac{\partial^2 \overrightarrow{E}}{\partial t^2}$
	\item[$\ast$] $\nabla^2\overrightarrow{B}=\mu_0 \varepsilon_0 \frac{\partial^2 \overrightarrow{B}}{\partial t^2}$
	\item[$\ast$] $\nabla^2\overrightarrow{H}=\mu_0 \varepsilon_0 \frac{\partial^2 \overrightarrow{H}}{\partial t^2}$
\end{itemize}
\item $\frac{\partial}{\partial t} = - i \omega$\\
$\nabla \cdot = i \overrightarrow{k} \cdot$\\
$\nabla \times = i \overrightarrow{k} \times$\\
$\nabla  = i \overrightarrow{k}$\\


\item Onde in mezzi senza cariche e senza correnti ($\mu=\mu_r \mu_0,\hspace{2mm} \varepsilon = \varepsilon_r \varepsilon_0$)
	\begin{equation*}
	\begin{cases}
		& \nabla \cdot \overrightarrow{E}=0\\
		& \nabla \times \overrightarrow{E}=-\mu\frac{\partial \overrightarrow{H}}{\partial t}\\
		& \nabla \cdot \overrightarrow{H} = 0\\
		& \nabla \times \overrightarrow{H} =\varepsilon\frac{\partial \overrightarrow{E}}{\partial t}
	\end{cases}
	\end{equation*} 
$|\overrightarrow{E}|=v|\overrightarrow{B}|=v \mu|\overrightarrow{H}|$\\
$v=\frac{1}{\sqrt{\mu \varepsilon}}=\frac{1}{\sqrt{\mu_0 \mu_r \varepsilon_0 \varepsilon_r}}$\\

Impedenza elettromagnetica del mezzo $Z=v \mu = \sqrt{\frac{\mu}{\varepsilon}}\hspace{5mm} [Z]=[R]$ misurata in Ohm\\
$|\overrightarrow{E}|=Z|\overrightarrow{H}|$ causa effetto $\rightarrow$ $E$ causa $H$\\
Impedenza del vuoto $Z_0=c\mu_0=\sqrt{\frac{\mu_0}{\varepsilon_0}}=377\Omega$

\item Mezzi ottici ($\mu_r \approx 1$)\\
$v \approxeq \frac{c}{\sqrt{\varepsilon_r}}$\\
Indice di rifrazione $n$ \hspace{5mm} $n=\frac{c}{v}\approxeq \sqrt{\varepsilon_r} $\\
$Z=v \mu = \sqrt{\frac{\mu}{\varepsilon}}\approxeq \frac{Z_0}{n}$
\item Vettore di Poynting\\
$\overrightarrow{S}=\frac{\overrightarrow{E} \times \overrightarrow{B}}{\mu_0}=\overrightarrow{E} \times \overrightarrow{H}\hspace{7mm} [\overrightarrow{S}]=W/m^2$\\
Conservazione energia elettromagnetica in forma locale\hspace{5mm} $\frac{\partial u_{EM}}{\partial t}+\nabla \cdot \overrightarrow{S}=0$\\
$u_{EM}=\frac{1}{vZ}E^2=\frac{Z}{v}H^2=\varepsilon E^2 = \frac{1}{\mu}B^2=\mu H^2$\\
$|\overrightarrow{S}|=ZH^2=\frac{1}{Z}E^2=v u_{EM}$
\item Intensità per onde elettromagnetiche\\
$I=\langle |\overrightarrow{S}|\rangle =\langle \frac{1}{Z} E^2 \rangle = \langle Z H^2\rangle =\frac{1}{2}\frac{E_0^2}{Z}=\frac{1}{2}ZH_0^2=v\langle u_{EM}\rangle$
\item Pressione di radiazione
$p=\frac{I}{c}=\frac{|\overrightarrow{S}|}{c}$
\item Campo di radiazione
$E_\theta(r,t)=\frac{q}{4 \pi \varepsilon_0}\frac{\sin \theta}{c^2 r} a (t-\frac{r}{c})$\\
Carica oscillante di moto armonico\\
$E_\theta(r,t)=\frac{q}{4 \pi \varepsilon_0}(\frac{\omega}{c})^2 \frac{\sin \theta}{r} A \cos[ \omega \cdot(t-\frac{r}{c})]$
\item Legge di Lambert\\
$I(z)=I_0 e^{-\mu z}$\\
$\mu$ coefficiente di assorbimento, $\mu=2\beta$, $\beta=k_I=\frac{\omega n_I(\omega)}{c}$
\item Velocità di gruppo\\
$v_g=\frac{d \omega}{dk}=\frac{c}{n_r+\omega \frac{dn_r}{d\omega}}$\\
$\frac{dn_r}{d\omega}\geq 0\Rightarrow v_g \leq v_f$ Dispersione normale\\
$\frac{dn_r}{d\omega}\leq 0\Rightarrow v_g \geq v_f$ Dispersione anomala
\item Eq. di Maxwell nei metalli ($\overrightarrow{J}=\sigma \overrightarrow{E}$)
\begin{equation*}
	\begin{cases}
		& \nabla \cdot \overrightarrow{E}=0\\
		& \nabla \times \overrightarrow{E}=-\frac{\partial \overrightarrow{B}}{\partial t}\\
		& \nabla \cdot \overrightarrow{B} = 0\\
		& \nabla \times \overrightarrow{B} =\mu \sigma \overrightarrow{E}+\mu\varepsilon\frac{\partial \overrightarrow{E}}{\partial t}
	\end{cases}
\end{equation*}\\
Da cui eq. del campo elettrico (non di D'Alembert, solo nel limite $\sigma \rightarrow 0$)\\
$\nabla^2 \overrightarrow{E}=\mu \varepsilon\frac{\partial^2 \overrightarrow{E}}{\partial t^2}+\mu \sigma\frac{\partial\overrightarrow{E}}{\partial t}$\\
$k=\frac{\omega}{c}\sqrt{1+i\frac{\mu \sigma v^2}{\omega}}=k_r+ik_I$\\
Il mezzo risulta dispersivo e vale Legge di Lambert ($\mu = 2k_I$)\\
Generalmente $\overrightarrow{E} \perp \overrightarrow{B}$, non in fase\\
$\overrightarrow{k}\times\overrightarrow{E}=\omega \overrightarrow{B} \rightarrow B_0=E_0|\overrightarrow{k}|\frac{e^{i\delta}}{\omega}$\\
\item Legge di Malus per i polarizzatori (onda ruotata di $\varphi_0$ rispetto asse polarizzatore, $E$ oscilla su $x$)\\
$I(\varphi_0)=\langle \frac{E_t^2}{Z} \rangle=\langle \frac{E_x^2 \cos^2\varphi_0}{Z} \rangle=I_0\cos^2 \varphi_0$
\item Riflessione e Rifrazione per onde EM\\
In incidenza normale
\begin{equation*}
	\begin{cases}
		& E_t=\frac{2n_1}{n_1+n_2}E_i\\
		& E_r=\frac{n_1-n_2}{n_1+n_2}E_i\\	
	\end{cases} \iff
	\begin{cases}
		& E_t=\frac{2Z_2}{Z_1+Z_2}E_i\\
		& E_r=\frac{Z_2-Z_1}{Z_1+Z_2}E_i\\	
	\end{cases} \iff
	\begin{cases}
		& H_t=\frac{2Z_1}{Z_1+Z_2}H_i\\
		& H_r=\frac{Z_2-Z_1}{Z_1+Z_2}H_i\\	
	\end{cases}
\end{equation*}\\
$I_r+I_t=I_i$ Conservazione energia nell'interfaccia tra i due mezzi non conduttori\\
$T=\frac{4Z_1Z_2}{(Z_1+Z_2)^2}$ Coefficiente di trasmissione, $I_t=TI_i$\\
$R=(\frac{Z_2-Z_1}{Z_1+Z_2})^2$ Coefficiente di trasmissione, $I_r=RI_i$\\
$R+T=1$
\item Incidenza non normale\\
$n_1 \sin\theta_i=n_2 sin \theta_t$ Legge di Snell
\item Definito \textit{Piano Incidente} il piano definito dalla normale alla superficie e la direzione del moto dell'onda.\\
Posti $\alpha = \frac{\cos \theta_t}{\cos \theta_i}, \hspace{4mm}\beta=\frac{n_2}{n_1}$,\hspace{4mm} Pedici: i=incidente, r=riflesso, t=trasmesso\\
Polarizzazione sul piano incidente
\begin{equation*}
	\begin{cases}
		& E_{i \parallel}+E_{r \parallel  }=\alpha E_{t \parallel}\\
		& E_{i \parallel}-E_{r \parallel  }=\beta E_{t \parallel}
	\end{cases}=
	\begin{cases}
		& E_{t\parallel}=\frac{2}{\alpha+\beta}E_{i\parallel}\\
		& E_{r\parallel}=\frac{\alpha - \beta}{2}E_{t\parallel}=\frac{\alpha - \beta}{\alpha + \beta}E_{i\parallel}
	\end{cases}\Rightarrow
\end{equation*}
$\rightarrow$Formule di Fresnel per $E_\parallel$
\begin{equation*}
	\begin{cases}
		&E_{t\parallel}=\frac{2n_1\cos\theta_i}{n_1 \cos \theta_t + n_2 \cos \theta_i}E_{i\parallel}\\
		&E_{r\parallel}=\frac{n_1 \cos \theta_t-n_2 \cos \theta_i}{n_1 \cos \theta_t+n_2 \cos \theta_i}E_{i\parallel}
	\end{cases}
\end{equation*}
\begin{equation*}
	\begin{cases}
		&t_\parallel=\frac{2n_1\cos\theta_i}{n_1 \cos \theta_t + n_2 \cos \theta_i}\\
		&r_\parallel=\frac{n_1 \cos \theta_t-n_2 \cos \theta_i}{n_1 \cos \theta_t+n_2 \cos \theta_i}
	\end{cases}
	\begin{cases}
		&E_{t\parallel}=t_\parallel E_{i\parallel}\\
		&E_{r\parallel}=r_\parallel E_{i\parallel}
	\end{cases}
\end{equation*}
\begin{equation*}
	\begin{cases}
		&R_\parallel=(\frac{\alpha-\beta}{\alpha+\beta})^2=(\frac{n_1 \cos \theta_t-n_2 \cos \theta_i}{n_1 \cos \theta_t+n_2 \cos \theta_i})^2=(r_\parallel)^2\\
		&T_\parallel=\frac{4\alpha \beta}{(\alpha+\beta)^2}
	\end{cases} 
	\begin{cases}
		&I_{r\parallel}=R_\parallel I_{i\parallel}\\
		&I_{t\parallel}=T_\parallel I_{i\parallel}\frac{\cos \theta_i}{\cos\theta_t}
	\end{cases}
\end{equation*}\\
$\frac{I_{i\parallel}}{\cos\theta_i}=\frac{I_{r\parallel}}{\cos \theta_i}+\frac{I_{t_\parallel}}{\cos \theta_t}$ (non si conserva l'intensità)\\ \\
Polarizzazione sul piano perpendicolare al piano incidente
\begin{equation*}
	\begin{cases}
		& E_{i \perp}+E_{r \perp  }= E_{t perp}\\
		& E_{i \perp}-E_{r \perp  }=\alpha \beta E_{t \perp}
	\end{cases}=
	\begin{cases}
		& E_{t\perp}=\frac{2}{(1+\alpha\beta)}E_{i\perp}\\
		& E_{r\perp}=\frac{(1- \alpha \beta) }{2}E_{t\perp}=\frac{(1- \alpha \beta)}{(1+ \alpha \beta)}E_{i\perp}
	\end{cases}\Rightarrow
\end{equation*}
$\rightarrow$Formule di Fresnel per $E_\perp$\\
\begin{equation*}
	\begin{cases}
		&E_{t\perp}=\frac{2n_1\cos\theta_i}{n_1 \cos \theta_i + n_2 \cos \theta_t}E_{i\perp}\\
		&E_{r\perp}=\frac{n_1 \cos \theta_i-n_2 \cos \theta_t}{n_1 \cos \theta_i+n_2 \cos \theta_t}E_{i\perp}
	\end{cases}
\end{equation*}
\begin{equation*}
	\begin{cases}
		&t_\perp=\frac{2n_1\cos\theta_i}{n_1 \cos \theta_i + n_2 \cos \theta_t}\\
		&r_\perp=\frac{n_1 \cos \theta_i-n_2 \cos \theta_t}{n_1 \cos \theta_i+n_2 \cos \theta_t}
	\end{cases}
	\begin{cases}
		&E_{t\perp}=t_\perp E_{i\perp}\\
		&E_{r\perp}=r_\perp E_{i\perp}
	\end{cases}
\end{equation*}
\begin{equation*}
	\begin{cases}
		&R_\perp=(\frac{1-\alpha\beta}{1+\alpha\beta})^2=(\frac{n_1 \cos \theta_i-n_2 \cos \theta_t}{n_1 \cos \theta_i+n_2 \cos \theta_t})^2=(r_\perp)^2\\
		&T_\perp=\frac{4\alpha \beta}{(1+\alpha\beta)^2}
	\end{cases} 
	\begin{cases}
		&I_{r\perp}=R_\perp I_{i\perp}\\
		&I_{t\perp}=T_\perp I_{i\perp}\frac{\cos \theta_i}{\cos\theta_t}
	\end{cases}
\end{equation*}\\
$\frac{I_{i\perp}}{\cos\theta_i}=\frac{I_{r\perp}}{\cos \theta_i}+\frac{I_{t_\perp}}{\cos \theta_t}$ (non si conserva l'intensità)\\ \\
Angolo di Brewster $\theta_B$\hspace{4mm}($r_\parallel=0$)\\
$\tan\theta_B=\frac{n_2}{n_1}$
\item Interferenza di Young\\ \\
$\rightarrow\sin \theta_{max}=\frac{n\lambda}{d}$,\hspace{2mm} Interferenza costruttiva, campo massimo\hspace{2mm} $|\overrightarrow{E}(P)|=2E'_0,\hspace{2mm} I(P)=4I_0$\\
$\Delta \varphi=kd\sin\theta=\frac{2 \pi}{\lambda}d\sin \theta=2n\pi$\\ \\
$\rightarrow\sin\theta_{min}=(n+\frac{1}{2})\frac{\lambda}{d}$ ,\hspace{2mm} Interferenza costruttiva, campo massimo\hspace{2mm} $|\overrightarrow{E}(P)|=0,\hspace{2mm} I(P)=0$\\
$\Delta \varphi=\frac{2 \pi}{\lambda}d \sin\theta=(2n+1)\pi$\\ \\
Intensità: $I(\Delta \varphi)=I_0|1+e^{i\Delta \varphi}|^2=4I_0\cos^2\frac{\Delta\varphi}{2}=4I_0\cos^2(\frac{\pi}{\lambda}d\sin\theta)$\\ 
Controllo dell'interferenza dato dal fattore $\frac{d}{\lambda}$\\
$d\ll\lambda$ Un solo massimo di interferenza\\
$d\gg\lambda$ Non posso considerare le sorgenti puntiformi\\
$d>\lambda$ Interferenza alla Young\\
\\Interferenza con più sorgenti ($\varphi=k(r_2-r_1)=kd\sin\theta$)\\
$\overrightarrow{E}(P)=\frac{e^{\frac{N}{2}i\varphi}}{e^{i\frac{\varphi}{2}}}\frac{\sin(\frac{N}{2}\varphi)}{\sin\frac{\varphi}{2}}$\\
$I(\Delta \varphi)=I_0|\frac{e^{\frac{N}{2}i\varphi}}{e^{i\frac{\varphi}{2}}}\frac{\sin(\frac{N}{2}\varphi)}{\sin\frac{\varphi}{2}}|^2=I_0\frac{\sin^2(\frac{N}{2}\varphi)}{\sin^2\frac{\varphi}{2}}=I_0\frac{\sin^2(N\frac{\pi}{\lambda}d\sin\theta)}{\sin^2(\frac{\pi}{\lambda}d\sin\theta)}$\\
$I_{max}=N^2I_0$\\
$\sin\theta_{max}=\frac{m\lambda}{d}$\hspace{4mm} Posizione dei massimi principali\\
$\sin\theta_{min}=\frac{n\lambda}{Nd}$ Posizione dei minimi (eliminare dal computo quelli coincidenti coi massimi principali $\frac{\pi}{\lambda}d\sin\theta=m\pi$)\\
$\Delta\theta=\frac{\lambda}{Nd}$ Larghezze a mezza altezza
\item Reticolo (dispositivo di $N$ fenditure grandi, $N'$ fenditure per unità di lunghezza)\\
$\mathcal{D}=\frac{d\theta}{d\lambda}=\frac{m}{d \cos\theta_{max}}$\hspace{4mm} Dispersione, capacità di deviare la luce al variare della $\lambda$\\
$\mathcal{R}=\frac{\lambda}{d\lambda}=\frac{Nm}{ \cos\theta_{max}}\approx Nm $\hspace{4mm} Potere risolutivo, capacità di vedere separati massimi a lunghezze d'onda vicine
\item Diffrazione da Fenditura\\
Regime di Fresnel $\rightarrow$ vicino\\
Regime di Fraunhofer $\rightarrow$ lontano\\
$a$ ampiezza della fenditura, $\Delta \varphi=ka\sin\theta$\\
$\sin\theta_{min}=n\frac{\lambda}{a}$\hspace{4mm}Minimi di intensità a $\Delta \varphi=ka\sin\theta=2n\pi=\frac{2\pi}{\lambda}a\sin\theta$, $n\neq0$ che è massimo\\
$I(\theta)=I_0\frac{\sin^2{(\alpha / 2)}}{(\frac{\alpha}{2})^2}=I_0\frac{\sin^2{(\frac{ka\sin \theta}{2})}}{(\frac{ka\sin \theta}{2})^2}$\hspace{4mm} Intensità\\
Larghezza del massimo centrale:\\
-Come distanza fra 2 minimi \hspace{4mm} $\Delta \theta =\frac{\lambda}{a}$\\
-Come altezza a mezza altezza \hspace{4mm} $\Delta \theta =\frac{\lambda}{2a}$\\
\item Diffrazione da foro circolare\\
$I(\theta)=I_0\frac{J^2_1(\alpha)}{(\frac{\alpha}{2})^2}=4I_0\frac{J^2_1(\frac{kD\sin\theta}{2})}{(\frac{kD\sin\theta}{2})^2}$\hspace{4mm} $J_1(\alpha)$ Funzione di Bessel di ordine 1\\
Posizione del primo minimo $\sin\theta=1.22\frac{\lambda}{d}$\\
\item Potere separatore (o risolutivo), risoluzione\\
$\frac{1}{\gamma_R}=\frac{D}{1.22\lambda}$ capacità di spostare il minimo di una sorgente sul minimo dell'altra (limite teorico)

\end{itemize}
\section{Ottica Geometrica}
\begin{itemize}
	\item $\theta_i=\theta_r$ Riflessione
	\item $n_1\sin\theta_1=n_2\sin\theta_2$ Legge di Snell
	\item Specchio piano $\rightarrow m=1$ Ingrandimento
\end{itemize}

Se il formulario ti è piaciuto considera di supportarmi con una donazione al link \href{paypal.me/gorsedh}{paypal.me/gorsedh}\\
Fa un po' schifo il formulario ma almeno ci si passa l'esame ecco\\
I soldi sono comunque ben graditi grz

\end{document}