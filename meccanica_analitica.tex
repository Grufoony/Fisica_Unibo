\documentclass[a4paper]{article}
\usepackage[T1]{fontenc}
\usepackage[utf8]{inputenc}
\usepackage[main=italian, english]{babel}
\usepackage{bookmark}
\usepackage[a4paper, total={6in, 9in}]{geometry}
\usepackage{amsmath}

\begin{document}
	\title{Formulario meccanica analitica}
	\author{Grufoony}
    \maketitle
    \section{Flusso di fase}
        Sia dato un potenziale V(x):
        \begin{itemize}
            \item Punto di equilibrio: $V'(x)=0$
            \item Stabilità: $V''(x_A)>0 \Rightarrow$ Oscillatore armonico di periodo (piccole oscillazioni) $T=\frac{2\pi}{\omega}$
            \item Instabilità: $V''(x_B)<0 \Rightarrow$ Oscillatore iperbolico con separatrici $p=\pm\sqrt{km}(x-x)$
            \item Costante $k=\lvert V''(x) \rvert \Rightarrow \omega^2=\frac{k}{m}(x-x_B)$
        \end{itemize}
    \section{Lagrangiana}
        \begin{itemize}
            \item Lagrangiana $\mathcal{L}=\mathcal{T}-V$
            \item Velocità $v=\dot{x}\hat{x}+x\vec{\omega}\wedge\hat{x}$
            \item Energia cinetica: $\mathcal{T}=\frac{mv^2}{2}$
            \item Integrali del moto:
                \begin{equation}
                    \begin{cases}
                    p_\theta=\frac{\partial{\mathcal{L}}}{\partial{\dot{\theta}}}=mr^2\dot{\theta}\\
                    H=\mathcal{T}+V
                    \end{cases}
                \end{equation}
            \item Potenziale efficace: $V_{eff}(r)=\frac{p_\theta^2}{2mr^2}+V(r) \Rightarrow H=\frac{p^2}{2m}+V_{eff}(r)$ con $p=m\dot{r}=\frac{\partial{\mathcal{L}}}{\partial{\dot{r}}}$
        \end{itemize}
\end{document}