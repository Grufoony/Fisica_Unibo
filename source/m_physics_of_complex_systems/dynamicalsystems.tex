% chaotic dynamics
A dynamical system is given by a \emph{phase space} $M \subseteq \mathbb{R}^d$ endowed with a collection of maps $\phi^t:M\rightarrow M$, where $t \in \mathbb{G} = \lbra \mathbb{R}, \mathbb{C}, \mathbb{Z}, \mathbb{N} \rbra$, with the so-called ``group property'':
$$
	\phi^{t+s} = \phi^t o \phi^s
$$
The maps $\phi^t$ are usually called \emph{flows}. \\ 
This formalization works for systems whose equations don't depend explicitly on time. \\
A typical example of a dynamical system is that of the solutions of a differential equation. So if we have a Cauchy problem
$$
	\begin{cases}
		\dot{x} = f(x) \\
		x\left(0\right) = x_0
	\end{cases}
$$
where the flow is the function that takes the initial condition and returns the correct solution for the differential equation.
So if this ODE has a global solution $x(t)$ that satisfies the initial condition, then 
$$
	\phi^t(x_0) = x(t)
$$
We must also assume the uniqueness of the solution, that is, for one initial condition we only have one solution. \\
Non-autonomous ODEs, where the equation field depends on time
$$
	\dot{x} = f(x,t)
$$
can be converted into autonomous ODEs in higher dimensions, going back to the previous simpler case. \\
We consider $\overline{M} = M \times \mathbb{R}$ with $y = (x,t) \in \overline{M}$, and we define
$$
	F(x,t) = (f(x,t),1)
$$
with $f : M \times \mathbb{R} \rightarrow \mathbb{R}^d$ and $F : M \times R = \overline{M} \rightarrow \mathbb{R}^{d+1}$. At this point we can write the new differential equation:
$$
	\dot{y} = F(y)
$$
with initial condition $y(0) = (x_0,0)$. For this to be useful, we must show that knowing the solution to this new equation implies knowing the solution for the original one. \\
Say $y(t) = (x(s),t(s))$ is a solution for the new equation
$$
	\left(\begin{matrix}
		\dot{x}(s) \\ 
		\dot{t}(s)
	\end{matrix}\right) = 
	F
	\left(\begin{matrix}
		x(s) \\ 
		t(s)
	\end{matrix}\right) =
	\left(\begin{matrix}
		f(x(s),t(s)) \\ 
		1
	\end{matrix}\right) = 
$$
$$
	\dot{x} = f(x,s) \ \ \ \ \ \dot{t} = 1
$$
with initial conditions $x(0) = x_0$ and $t(0) = 0$. For what concerns t, it's the identity function, so $t(s) = s$, which means that I can invert $t$ and $s$, so going back to the first equation we have
$$
	\dot{x} = f(x,t)
$$
which was our original equation. This means that a solution for the non-autonomous equation is also a solution for the new autonomous one. \\
So, if we have a theory that treats non-autonomous systems, that theory must also be able to treat autonomous systems. \\ \\
If the $t$ variable is a natural number, we call it $n$
$$
	\phi^n = T^n
$$