The first question that someone should ask themselves is, of course, what is an image? \\
We know from everyday life that we can't have an image without light. In more general terms, in order to have light we need to have some sort of radiation. \\ 
So we can say that an image is a distribution of matter, which is called an \textit{object}, that is visible when illuminated. Alternatively it can be defined as a measure of the intensity of the reflected radiation. \\ \\
Any imaging technique is characterized by the way that the object and the radiation interact. An imaging system collects radiation emitted by objects. \\ \\
We define two quantities: The energy intensity, $E$, and the energy flux, $Q$. \\
Through this two quantities we can define \textit{irradiance}, \textit{radial intensity} and \textit{radiance}:
$$
	irradiance = \frac{dQ}{dA} \left[\frac{W}{m^2}\right] 
$$
$$
	radial intensity = \frac{dQ}{d\omega} \left[\frac{W}{sterad}\right]
$$
$$
	radiance = \frac{dQ}{d\omega dA} \left[\frac{W}{sterac\cdot m^2}\right]
$$
Usually sources of radiation are polychromatic, so there is a spectrum. It is thus essential to know how the radiation interacts with objects.  \\
Different wavelengths interact differently, thus producing different images. This can be used in what is called \textit{multispectral imaging}. 
