
% Copyright (c) 2022, Grufoony
% All rights reserved.

% THIS SOFTWARE IS PROVIDED BY THE COPYRIGHT HOLDERS AND CONTRIBUTORS "AS IS"
% AND ANY EXPRESS OR IMPLIED WARRANTIES, INCLUDING, BUT NOT LIMITED TO, THE
% IMPLIED WARRANTIES OF MERCHANTABILITY AND FITNESS FOR A PARTICULAR PURPOSE ARE
% DISCLAIMED. IN NO EVENT SHALL THE COPYRIGHT HOLDER OR CONTRIBUTORS BE LIABLE
% FOR ANY DIRECT, INDIRECT, INCIDENTAL, SPECIAL, EXEMPLARY, OR CONSEQUENTIAL
% DAMAGES (INCLUDING, BUT NOT LIMITED TO, PROCUREMENT OF SUBSTITUTE GOODS OR
% SERVICES; LOSS OF USE, DATA, OR PROFITS; OR BUSINESS INTERRUPTION) HOWEVER
% CAUSED AND ON ANY THEORY OF LIABILITY, WHETHER IN CONTRACT, STRICT LIABILITY,
% OR TORT (INCLUDING NEGLIGENCE OR OTHERWISE) ARISING IN ANY WAY OUT OF THE USE
% OF THIS SOFTWARE, EVEN IF ADVISED OF THE POSSIBILITY OF SUCH DAMAGE.

%PREAMBULO

\documentclass[a4paper, titlepage]{report}%definizione obbligatoria tipo di documento con opzioni a scelta
\usepackage[T1]{fontenc}                %codifica caratteri e font utilizzati
\usepackage[utf8]{inputenc}             %codifica del protocollo
\usepackage[english, italian]{babel}    %lingue utilizzate, italiano quella principale perchè l'ultima delle elencate

\usepackage{esvect}
\usepackage{mathtools}
\usepackage{bm}
\usepackage{url}
\usepackage{emptypage}       %elimina le facciate blank
\usepackage{hyperref}        %definisce i collegamenti ipertestuali e Web
\usepackage{indentfirst}
\usepackage{amsmath}         %pacchetto per linguaggio matematico
\usepackage{amssymb}         %pacchetto per linguaggio matematico
\usepackage{amsthm} 
\usepackage{physics}         %tre stili predefiniti per gli enunciati

\usepackage{booktabs, caption} %pacchetti per le tabelle
\usepackage{graphicx} %pacchetto per le figure
\graphicspath{ {./images/} }
\usepackage{wrapfig}
%Definisco gli stili per  teoremi, definizioni, ecc

\theoremstyle{definition} %inizializzo definizioni                
\newtheorem{definizione}{Definizione}[chapter]  %[chapter]-->numerazione della def con riferimento al capitolo
\theoremstyle{plain}
\newtheorem*{legge}{Legge Fisica}
\theoremstyle{plain}
\newtheorem{proposizione}{Proposizione}[chapter]
\theoremstyle{remark}
\newtheorem*{osservazione}{Osservazione}
\theoremstyle{remark}
\newtheorem*{nota}{Nota}
\theoremstyle{plain}
\newtheorem*{criterio}{Criterio}
\theoremstyle{plain}
\newtheorem*{serie}{Serie notevole}
\theoremstyle{plain}
\newtheorem*{integrale}{Integrale notevole}
\theoremstyle{plain}
\newtheorem*{integrali}{Integrali notevoli}
\theoremstyle{plain}
\newtheorem*{tecnica}{Tecnica di integrazione}


\usepackage[vmargin=2.5cm,hmargin=2cm]{geometry}



\renewcommand{\v}[1]{\mathbf{\hat{#1}}} 
%%VERSORE
\renewcommand{\vec}[1]{\vv{\bm{#1}}}   
%%VETTORE BOLD E CON ARROW
\newcommand{\Norm}[1]{ \left\lVert {\vv{\bm{#1}}} \right\rVert}
%%%NORMA DI UN VETTORE


\begin{document}

    \title{Foglio per esame scritto di Analisi 1}
    \author{Niccolò Zanotti}
    \maketitle

    %%%%%%%%%%%%%%%%%%%%%%%%%%
%%%%%% SUCCESSIONI %%%%%%%
%%%%%%%%%%%%%%%%%%%%%%%%%%

\section*{Successioni in $\mathbb{R}$}
%%%%%%%%%%%%  RAPPORTO  %%%%%%%%%
\begin{criterio}{Del rapporto}

    Sia ${\{ a_n \}}$ una successione in $\mathbb{R^+}$, ed esista
     $\lim_{n \to \infty} \frac{a_{n+1}}{a_n}$. Allora se 
     \begin{enumerate}
         \item $\lim_{n \to \infty} \frac{a_{n+1}}{a_n} < 1$,
               allora $a_n \rightarrow 0$;
        \item $\lim_{n \to \infty} \frac{a_{n+1}}{a_n} > 1$,
        allora $a_n \rightarrow +\infty$;
     \end{enumerate}
%%%%%%%%%%%%%%%%%%%%%%%%%%%%%%%%%
\begin{criterio}{}
    
    Siano $\alpha,\beta \in \mathbb{R^+}$ e sia ${\{ a_n \}}$ 
    una successione in $\mathbb{R}$. Allora se
\[
    \alpha \leq a_n \leq \beta \quad \forall n \in \mathbb{N}  
    \Longrightarrow \sqrt[n]{a_n} \rightarrow 1 \text{ per }
     n \rightarrow +\infty
\]
Inoltre se
\[
    a_n \rightarrow \lambda \in \mathbb{R^+} \Longrightarrow 
    \sqrt[n]{a_n} \rightarrow 1
\]
\end{criterio}
%%%%%%%%%%%%%%%%%%%%%%%%%%%%%%%%%
%%%%%Es 18 obrecht%%%%%%%%
\begin{criterio}{}
    
    Sia $\{a_n\}$ una successione in $\mathbb{R}$. Allora se
\[
     a_n \rightarrow \lambda \in \bar{R} \Longrightarrow 
     \frac{a_1+a_2+ ... + a_n}{n} \rightarrow \lambda  
\]
%%%%%%%%%%%%%%%
%%%%%%18 ii obrecht%%%%
%%%%%%%%%%%%%%
\begin{criterio}
    
    Sia $\{a_n\}$ una successione in $\mathbb{R}$. Allora se
\[
     a_{n+1}-a_n \rightarrow \lambda \in \bar{R} \Longrightarrow 
     \frac{a_n}{n} \rightarrow \lambda       
\]
%%%%%%%%%%%%%%%%
%%%%es 18 iii obrecht
%%%%%%%%%%%%%%%%%
\begin{criterio}
    
    Sia $\{a_n\}$ una successione in $\mathbb{R^+}$. Allora se
\[
     \frac{a_{n+1}}{a_n} \rightarrow \lambda \in \bar{R} \Longrightarrow
     \sqrt[n]{a_n} \rightarrow \lambda    
\]

\end{criterio}
    
\end{criterio}


\end{criterio}

   

\end{criterio}

\paragraph*{Approssimazione di Stirling}
\[
     n! \sim n^n e^{-n} \sqrt{2 \pi n}    
\]
\[
       log(n!) = nlog(n) -n + \frac{1}{2} log(n) + log(\sqrt{2 \pi}) + o(1)
\]



    %%%%%%%%%%%%%%%%%%%%%%%%%%
%%%%%%%% SERIE %%%%%%%%%%
%%%%%%%%%%%%%%%%%%%%%%%%%%

\section*{Serie numeriche in $\mathbb{R}$}

%%%%%%%%%%%%%%%%%%%%%%%
%%% GEOMETRICA %%%%%%%%
%%%%%%%%%%%%%%%%%%%%%
\begin{serie}{Geometrica}
\[
        \sum_{n=0}^{\infty}c^n
        \begin{cases}
            \text{converge} \quad se \ \abs*{c}<1 ,
            \quad s_n= \frac{1}{1-c}\\
            \text{diverge}  \quad  se \ c \geq  1 \\
            \text{indeterminata} \quad se \ c \leq -1 \\
        \end{cases}   
\]
\end{serie}
%%%%%%%%%%%%%%%%%%%%%%%%%%%%%%%%%%%%
%%%%%%%%%ARMONICA GEN %%%%%%%%%%%%
%%%%%%%%%%%%%%%%%%%%%%%%%%%%%%%%%
\begin{serie}{Armonica Generalizzata}
\[
            \sum_{n=1}^{\infty} \frac{1}{n^{\alpha}}
            \begin{cases}
                \text{converge} \quad se \ \alpha > 1 \\
                \text{diverge}  \quad  se \ \alpha \leq  1 \\
            \end{cases}  
            \alpha \in \mathbb{R}  
\]
\end{serie}
%%%%%%%%%%%%%%%%%%%%%%%%%%%%%%%%%%
%%%%%%%SERIE DI ABEL O ARMONICA GEN 
%%%%%%%%%%%DI TIPO 2%%%%%%%%%%%%%%
\begin{serie}{Di Abel o Armonica Generalizzata di tipo $2$}

\[
    \sum_{n=2}^{\infty} \frac{1}{n^{\alpha}\log^{\beta}(n)}
    \begin{cases}
        \text{converge} \quad se \ \alpha > 1 \ \land \forall \beta \\
        \text{converge} \quad se \ \alpha > 1 \ \lor
        \alpha = 1 \ \land \beta > 1 \\
        \text{diverge}  \quad  se \ \alpha < 1 \ \lor \alpha = 1
        \ \land \beta \leq 1  \\
    \end{cases} 
    \alpha,\beta \in \mathbb{R} 
\]

\end{serie}


%%%%%%%%%%%%%%
%%CONFRONTO%%%%
%%%%%%%%%%%%%%
\begin{criterio}{Del confronto}

    Siano $\sum_{}^{}a_k $ e $\sum_{}^{}b_k $ due serie a termini
    in $\mathbb{R^+}$ tali che
\[
    0 \leq a_n \leq b_n \quad \forall n \in \mathbb{N}    
\]
Allora:
       \begin{enumerate}
    \item $\sum_{}^{}b_k$ convergente $\Longrightarrow \sum_{}^{}a_k $ 
         convergente;
    \item $\sum_{}^{}a_k$ divergente $\Longrightarrow \sum_{}^{}b_k $ 
    divergente.
    
       \end{enumerate} 
\end{criterio}
%%%%%%%%%%%%%%%%%%%
%%%%%ASINTOTICO%%%%
%%%%%%%%%%%%%%%%%%
\begin{criterio}{Del confronto asintontico}

    Siano $\sum_{}^{}a_k $ e $\sum_{}^{}b_k $ due serie a termini
    in $\mathbb{R^+}$. Allora se
\[
      a_n \sim b_n \Longrightarrow a_n \ e \  b_n \quad 
       \text{hanno lo stesso carattere}  
\]
\end{criterio}
%%%%%%%%%%%%%%%%%%
%%%%%%RADICE%%%%%
%%%%%%%%%%%%%%%%%%
\begin{criterio}{Della radice}

    Sia $\sum_{}^{}a_n$ una serie a termini in $\mathbb{R}_0^+$. Allora se esiste
     $\lim_{n \to \infty} \sqrt[n]{a_n} = \lambda$
   \begin{enumerate}
    \item $\lambda > 1 \Longrightarrow a_n $ convergente;
    \item $\lambda < 1 \Longrightarrow a_n $ divergente;
   \end{enumerate}

\end{criterio}
%%%%%%%%%%%%%%%%%%
%%%%%%LEIBNIZ%%%%%
%%%%%%%%%%%%%%%%%%
\begin{criterio}{Di Leibniz}

    Sia $\sum_{}^{}(-1)^na_n$ una serie a termini di segno alterno. Allora se
  \begin{enumerate}
    \item $a_{n+1}<a_n$ ($a_n $ è definitivamente decrescente);
    \item $a_n \rightarrow 0$ ($a_n$ è infinitesima);
  \end{enumerate}
    
  la serie è convergente.
\end{criterio}
%%%%%%%%%%%%%%%%%%
%%%%%%CAUCHY%%%%%
%%%%%%%%%%%%%%%%%%
\begin{criterio}{Di condensazione(Cauchy)}

    Sia $\{a_n\}$ una successione a termini in $\mathbb{R}^+$.Allora se
  \begin{enumerate}
    \item $a_{n+1}<a_n$ ($a_n $ è definitivamente decrescente);
    \item $a_n \rightarrow 0$ ($a_n$ è infinitesima);
  \end{enumerate}
\[
    \sum_{}^{}a_n \quad \text{converge}  \Longleftrightarrow \sum_{}^{}2^na_{2^n} 
    \quad \text{converge}.
\]
\end{criterio}
%%%%%%%%%%%%%%%%%
%%%%%%%RAPPORTO%%
%%%%%%%%%%%%%%%%%
\begin{criterio}{Del rapporto}
    
    Sia $\sum_{}^{}a_n$ una serie a termini in $\mathbb{R}^+$ tale che
    esista $\lim_{n \to \infty} \frac{a_{n+1}}{a_n} = \lambda$.
    Allora se 
    \begin{enumerate}
        \item $\lambda>1 \Longrightarrow $ la serie non è convergente;
        \item $\lambda<1 \Longrightarrow $ la serie è convergente.
    \end{enumerate}
\end{criterio}
%%%%%%%%%%%%%%%%%%
%%%%%% RAABE %%%%%
%%%%%%%%%%%%%%%%%%
\begin{criterio}{Di Raabe}
    
    Sia $\sum_{}^{}a_n$ una serie a termini in $\mathbb{R}^+$ tale che
\[
        \frac{a_{n+1}}{a_n} = 1 - \frac{\alpha}{n} + o(\frac{1}{n}) 
        \quad , \alpha \in \mathbb{R} \quad per \ n \rightarrow +\infty
\]

    Allora se 
    \begin{enumerate}
        \item $\alpha>1 \Longrightarrow $ la serie è convergente;
        \item $\alpha<1 \Longrightarrow $ la serie non è convergente.
    \end{enumerate}
\end{criterio}
%%%%%%%%%%%%%%%
%%%%DIRICHLET %%
%%%%%%%%%%%%%%%
\begin{criterio}{Di Dirichlet}
    
    Siano $\{\alpha_n\}$ una successione in $\mathbb{C}$ e $\{ \beta_n\}$ una successione in $\mathbb{R}^+$ tali che
    \begin{enumerate}
        \item $\sum_{}^{}\alpha_n$ è limitata;
        \item $\{ \beta_n\}$ è monotona descrescente ed infinitesima.
    \end{enumerate}

    Allora la serie $\sum_{}^{}\alpha_n\beta_n$ è convergente.
\end{criterio}
    
    %%%%%%%%%%%%%%%%%%%%%%%%%%
%%%%%%%% TAYLOR %%%%%%%%%%
%%%%%%%%%%%%%%%%%%%%%%%%%%

\section*{Sviluppi di MacLaurin delle principali funzioni}
\paragraph*{Resto in forma di Peano}
\[
  e^x = \sum_{k = 0}^{\infty}\frac{x^k}{k!} =
        1 + x + \frac{x^2}{2}+ \frac{x^3}{3!} +
        ... + \frac{x^n}{n!}+o(x^n)
\]
\[
   sin(x) = \sum_{k = 0}^{\infty} \frac{(-1)^kx^{2k+1}}{(2k+1)!} =
   x - \frac{x^3}{3!}+ \frac{x^5}{5!} +
   ... + \frac{x^{2n+1}}{n!}+o(x^{2n+2})
\]
\[
    cos(x)=\sum_{k = 0}^{\infty}\frac{(-1)^kx^{2k}}{(2k)!} =
    1 - \frac{x^2}{2}+ \frac{x^4}{4!} +
    ... + \frac{(-1)^{n}x^{2n}}{n!}+o(x^{2n+1})
\]
\[
   tg(x)= x + \frac{x^3}{3} + \frac{2}{15}x^5+o(x^6) 
   \text{   \quad per   }\abs*{x}< \frac{\pi}{2}
\]
\[
   arctg(x)= \sum_{k = 0}^{\infty} \frac{(-1)^kx^{2k+1}}{2k+1}
    = x - \frac{x^3}{3} + \frac{x^5}{5}-\frac{x^7}{7} + ... +
    \frac{(-1)^nx^{2n+1}}{2n+1} + o(x^{2n+2})
\]
\[
   arcsin(x)= \sum_{k = 0}^{\infty} \frac{(2k)! \ x^{2k+1}}{2^{2k}(k!)^2
    (2k+1)} = x + \frac{x^3}{6} + \frac{3x^5}{40}+\frac{5x^7}{112} + ... +
    \frac{(2n)! \ x^{2n+1}}{2^{2n}(n!)^2(2n+1)} + o(x^{2n+1})
     \ \ \text{  per} \ \abs*{x}<1 
\]
\[
    sinh(x)=\sum_{k = 0}^{\infty}\frac{x^{2k+1}}{(2k+1)!} =
    x + \frac{x^3}{3!} + \frac{x^5}{5!} +
    ... + \frac{x^{2n+1}}{n!}+o(x^{2n+2}) 
\]
\[
   cosh(x)=\sum_{k = 0}^{\infty}\frac{x^{2k}}{(2k)!} =
    1 + \frac{x^2}{2}+ \frac{x^4}{4!} +
    ... + \frac{x^{2n}}{n!}+o(x^{2n+1})
\]   
\[
   log(1+x) =  \sum_{k = 0}^{\infty}\frac{(-1)^{k+1}x^{k}}{k} =
   x - \frac{x^2}{2} + \frac{x^3}{3} - \frac{x^4}{4} +
   ... + \frac{(-1)^{n+1}x^{n}}{n}+o(x^n)  \text{  per}\abs*{x}<1 
\]
\[
   (1+x)^{\alpha} =  \sum_{k = 0}^{\infty}\binom{\alpha}{k}  x^k =
   1 + \alpha x + \frac{\alpha(\alpha -1)}{2}x^2 + \frac{\alpha(\alpha -1)
   (\alpha -2)}{3!}x^3 +... + \binom{\alpha}{n}x^n+o(x^n) 
    \text{  per}\abs*{x}<1 
\]
\[
   \sqrt{1+x}=(1+x)^{1/2} =  \sum_{k = 0}^{\infty}\binom{\alpha}{k}  x^k =
   1 + \frac{x}{2} - \frac{x^2}{8} + \frac{1}{16}x^3 +... + 
   \binom{\frac{1}{2}}{n}x^n+o(x^n) 
\]
%%%%%%%%%%%%%%%%%%%%%%%%%%%%%%%%%%%%%%%%%%%%%%%%%%%%%%%%%%%%%%%%%%%%%%%%%
%
%   RESTO PER WITH BIG OH NOTATION
%
%%%%%%%%%%%%%%%%%%%%%%%%%%%%%%%%%%%%%%%%%%%%%%%%%%%%%%%%%%%%%%%%%%%%%%%%%
\paragraph*{Resto in forma O(·) utile per serie numeriche}
\[
  e^x = \sum_{k = 0}^{\infty}\frac{x^k}{k!} =
        1 + x + \frac{x^2}{2}+ \frac{x^3}{3!} +
        ... + \frac{x^n}{n!}+O(x^{n+1})
\]
\[
   sin(x) = \sum_{k = 0}^{\infty} \frac{(-1)^kx^{2k+1}}{(2k+1)!} =
   x - \frac{x^3}{3!}+ \frac{x^5}{5!} +
   ... + \frac{x^{2n+1}}{n!}+O(x^{2n+3})
\]
\[
    cos(x)=\sum_{k = 0}^{\infty}\frac{(-1)^kx^{2k}}{(2k)!} =
    1 - \frac{x^2}{2}+ \frac{x^4}{4!} +
    ... + \frac{(-1)^{n}x^{2n}}{n!}+O(x^{2n+2})
\]
\[
   tg(x)= x + \frac{x^3}{3} + \frac{2}{15}x^5+O(x^7) 
   \text{   \quad per   }\abs*{x}< \frac{\pi}{2}
\]
\[
   arctg(x)= \sum_{k = 0}^{\infty} \frac{(-1)^kx^{2k+1}}{2k+1}
    = x - \frac{x^3}{3} + \frac{x^5}{5}-\frac{x^7}{7} + ... +
    \frac{(-1)^nx^{2n+1}}{2n+1} + O(x^{2n+3})
\]
\[
   arcsin(x)= \sum_{k = 0}^{\infty} \frac{(2k)! \ x^{2k+1}}{2^{2k}(k!)^2
    (2k+1)} = x + \frac{x^3}{6} + \frac{3x^5}{40}+\frac{5x^7}{112} + ... +
    \frac{(2n)! \ x^{2n+1}}{2^{2n}(n!)^2(2n+1)} + O(x^{2n+2})
    \ \ \text{  per}\abs*{x}<1 
\]
\[
    sinh(x)=\sum_{k = 0}^{\infty}\frac{x^{2k+1}}{(2k+1)!} =
    x + \frac{x^3}{3!} + \frac{x^5}{5!} +
    ... + \frac{x^{2n+1}}{n!}+O(x^{2n+3}) 
\]
\[
   cosh(x)=\sum_{k = 0}^{\infty}\frac{x^{2k}}{(2k)!} =
    1 + \frac{x^2}{2}+ \frac{x^4}{4!} +
    ... + \frac{x^{2n}}{n!}+O(x^{2n+2})
\]   
\[
   log(1+x) =  \sum_{k = 0}^{\infty}\frac{(-1)^{k+1}x^{k}}{k} =
   x - \frac{x^2}{2} + \frac{x^3}{3} - \frac{x^4}{4} +
   ... + \frac{(-1)^{n+1}x^{n}}{n}+O(x^{n+1})  \text{  per}\ \abs*{x}<1 
\]
\[
   (1+x)^{\alpha} =  \sum_{k = 0}^{\infty}\binom{\alpha}{k}  x^k =
   1 + \alpha x + \frac{\alpha(\alpha -1)}{2}x^2 + \frac{\alpha(\alpha -1)
   (\alpha -2)}{3!}x^3 +... + \binom{\alpha}{n}x^n+O(x^{n+1}) 
    \text{  per}\abs*{x}<1 
\]
\[
   \sqrt{1+x}=(1+x)^{1/2} =  \sum_{k = 0}^{\infty}\binom{\alpha}{k}  x^k =
   1 + \frac{x}{2} - \frac{x^2}{8} + \frac{1}{16}x^3 +... + 
   \binom{\frac{1}{2}}{n}x^n+O(x^{n+1}) 
\]

\section*{Integrali in $\mathbb{R}$}


\begin{integrale}{Per integrazione di funzioni razionali}

\[
    \int_{}^{} \frac{dx}{(ax+b)^2+c^2} \, = \frac{1}{bc} arctg(\frac{ax+b}{c}) + c
\]
\end{integrale}
\begin{integrali}{Funzioni goniometriche al quadrato}
\[
    \int_{}^{} \sin[2](x) \,dx = \frac{x-\sin(x)\cos(x)}{2}   
\]
\[
    \int_{}^{} \cos[2](x) \, dx = \frac{x+\sin(x)\cos(x)}{2}   
\]

\end{integrali}
\begin{integrale}{dalle sostituzioni iperboliche}
    
    \[
        \int_{}^{} \frac{1}{\sqrt{x^2+ a^2}} \,dx = 
        log(x+ \sqrt{x^2+a^2}) + C = Sett \,sinh(\frac{x}{a}) + C
    \]

\end{integrale}
\begin{integrali}{Delle funzioni iperboliche}
    
    \[
        \int_{}^{} sinh \, cx \, dx = \frac{1}{c} cosh \, cx
        \qquad         
        \int_{}^{} cosh \, cx \, dx = \frac{1}{c} sinh \, cx
    \]
    \[
        \int_{}^{} sinh^2 \, cx \, dx = \frac{1}{4c} sinh \, 2cx -\frac{x}{2}
        \qquad
        \int_{}^{} cosh^2 \, cx \, dx = \frac{1}{4c} sinh \, 2cx +\frac{x}{2}
    \]
    \[
        \int_{}^{} \frac{dx}{sinh \, cx} = \frac{1}{c} log(tanh(\frac{cx}{2}))
        \quad o \quad 
        \int_{}^{} \frac{dx}{sinh \, cx} = \frac{1}{c} log(\frac{cosh \, cx -1}{cosh \, cx -1})
    \]
    \[
        \int_{}^{} \frac{dx}{cosh \, cx} = \frac{2}{c} arctg(e^{cx})  
    \]

\end{integrali}
%%%%%%%%%%%%%%%%%%%%%%%%%%%%%%%%%%%%%%%%%%%%%%%%%%%%%%%%%%%%%%%%%%%%%%%%%%%%%%%%%%%%%%%%%%%%%%%%%%%%%%%%%%%%%%%%%%%%%%%%%%%%%%%%%%%%

%%%%%%%%%%%%%%%%%%%%%%%
%% SENO COSENO GRADO 1 
%%%%%%%%%%%%%%%%%%%%%%%%%
\begin{tecnica}{Funzioni razionali composte da funzioni goniometriche di grado 1}
   
    Integrali del tipo:
\[
     \int_{}^{} R(sinx,cosx) \,dx         
\]
si razionalizzano con la sostituzione $t = tg(\frac{x}{2})$ e si sfruttano le formule parametriche
di sinx e cosx:
\[
    cosx = \frac{1-tg^2(\frac{x}{2})}{1+tg^2(\frac{x}{2})}    
    \qquad sinx = \frac{2tg(\frac{x}{2})}{1+tg^2(\frac{x}{2})}
\]
\[
  t = tg(\frac{x}{2}) \quad \rightarrow \quad cosx = \frac{1-t^2}{1+t^2} \quad sinx = \frac{2t}{1+t^2}
  \quad x = 2 arctg(t) \quad dx = \frac{2dt}{t^2 +1}
\]
Riconducendosi ad un integrale del tipo:
\[
    \int_{}^{} R(\frac{2t}{1+t^2},\frac{1-t^2}{1+t^2}) \,\frac{2dt}{t^2+1}  
\]
\end{tecnica}


\begin{tecnica}{Funzioni razionali composte da funzioni goniometriche di grado 2}
   
    Integrali del tipo:
\[
     \int_{}^{} R(sin^2x,cos^2x,sinxcosx) \,dx         
\]
si razionalizzano con la sostituzione $t = tg(x)$ e si sfruttano le formule parametriche
:
\[
  t = tg(x) \quad \rightarrow \quad cos^2x = \frac{1}{1+t^2} \quad sin^2x = \frac{t^2}{1+t^2}
  \quad sinxcosx = \frac{t}{1+t^2} \quad dx = \frac{dt}{1+t^2}
\]
Riconducendosi ad un integrale del tipo:
\[
    \int_{}^{} R(\frac{t^2}{1+t^2},\frac{1}{1+t^2},\frac{t}{1+t^2}) \,\frac{dt}{t^2+1}  
\]
\end{tecnica}
%%%%%%%%%%%%%%%%%%%%%

%%%%%%%%%%%%%%%%%%%%%
\begin{tecnica}{}


    Integrali del tipo
\[
    \int_{}^{} \sqrt{a^2-x^2}\,dx
\]
si risolvono con la sostituzione
\[
   x = a \sin(t) \rightarrow t = \arcsin(x) \quad dx = \cos(t) dt
\]
\end{tecnica}

%%%%%%%%%%%%%%%%%%%%%%%
%% SOSTITUZIONI IPERBOLICHE 
%%%%%%%%%%%%%%%%%%%%%%%%%
\begin{tecnica}{Sostituzioni iperboliche:1}

    Integrali del tipo 
    \[
        \int_{}^{} R(x, \sqrt{x^2+a^2}) \,dx         
    \]
    si razionalizzano con la sostituzione $x = asinh(t)$, per cui:
    \[
        x = asinh(t) \ \  dx = a cosh(t) \quad  \rightarrow \quad 
        \sqrt{x^2+a^2} = a cosh(t) 
    \]

\end{tecnica}
\begin{tecnica}{Sostituzioni iperboliche:2}

    Integrali del tipo 
    \[
        \int_{}^{} R(x, \sqrt{x^2 - a^2}) \,dx         
    \]
    si razionalizzano con la sostituzione $x = acosh(t)$, per cui:
    \[
        x = a cosh(t) \ \  dx = a sinh(t) \quad  \rightarrow \quad 
        \sqrt{x^2 - a^2} = a sinh(t) 
    \]
\end{tecnica}

%%%%%%%%%%%%%%%%%%%%%%%
%% FUNZIONI TRASCENDENTI 
%%%%%%%%%%%%%%%%%%%%%%%%%
\begin{tecnica}{Funzioni razionali trascendenti}

   Integrali del tipo 
\[
    \int_{}^{} R(e^{ax}) \,dx         
\]
si razionalizzano con la sostituzione $e^{ax} = t$.
    
\end{tecnica}
%%%%%%%%%%%%%%%%%%%%
% IRRAZIONALI 1
%%%%%%%%%%%%%%%%%%%
\begin{tecnica}{Integrazione di funzioni irrazionale tipo 1}
    
    Integrali del tipo 
    \[
        \int_{}^{} R(x,(\frac{ax+b}{cx+d})^{r_1},
        (\frac{ax+b}{cx+d})^{r_2},...,
        (\frac{ax+b}{cx+d})^{r_n}) \,dx  
        \qquad r_1,r_2,...,r_n \in \mathbb{Q}      
    \]
dove R è una funzione razionale, si integrano tramite la sostituzione
\[
   t^N = \frac{ax+b}{cx+d}    
\]
dove N è il minimo comune multiplo dei denominatori dei numeri
 $r_1,r_2,...,r_n$. \\
Esempio:
\[
    \int_{}^{} \frac{1+\sqrt{\frac{x+1}{x+2}}}{1- \sqrt[3]{
     \frac{x+1}{x+2}}}\, dx , \qquad \frac{x+1}{x+2} = t^6
\]
\end{tecnica}
%%%%%%%%%%%%%%%%%%%
% SOSTITUZIONI DI EULER %%%
%%%%%%%%%%%%%%%%%
\begin{tecnica}{Sostituzioni di Euler}
    
    Integrali del tipo 
    \[
        \int_{}^{} R(x,(\sqrt{ax^2+bx+c}) \, dx      
    \]
dove R è una funzione razionale. Le sostituzioni dipendono dai tre casi:
\begin{description}
    \item[$a>0$] 
    \[
      x = \frac{at^2}{b-2at} \ , \quad \sqrt{ax^2+bx+c} = -\sqrt{a}\frac{at^2-bt+c}
      {b-2at} \ , \quad dx = -2a\frac{at^2-bt+c}{(b-2at)^2}dt
    \] 
    \[
       -2a \int_{}^{} R(\frac{at^2}{b-2at}  ,  -\sqrt{a}(\frac{at^2-bt+c}{b-2at}))
       \, dt    
    \]
    Esempio: 
    \[
    \int_{}^{} \frac{1+\sqrt{x^2+3x}}{2-\sqrt{x^2+3x}} \, dx , \qquad x = \frac{t^2}{3-2t}
    \Longrightarrow -2\int_{}^{} \frac{(t^2-3t)(3+t-t^2)}{(3-2t)^2(t^2-7t+6)} \, dt
    \]

    \item[$a<0$]
    \[
    x = \frac{1}{2a}(\alpha \frac{1-t^2}{1+t^2}-b) \ , \ \alpha = \sqrt{b^2-4ac}  
    \ , \ dx = \frac{-2\alpha t dt}{\alpha(t^2+1)^2}  
    \]
    \[
        \Longrightarrow
       - \int_{}^{} R(\frac{1}{2a}(\alpha \frac{1-t^2}{1+t^2}-b), 
       \frac{t\alpha }{\sqrt{-a}(1+t^2)}) \cdot \frac{2\alpha t}{\alpha(t^2+1)^2}
       \, dt
    \]
    Esempio:
    \[
    \int_{}^{} \frac{\sqrt{2x-x^2}+x}{2-\sqrt{2x-x^2}} \, dx
     \quad a = -1,b = 2, c= 0, \alpha = 2 \longrightarrow x = 1- \frac{1-t^2}
     {1+t^2} = \frac{2t^2}{1+t^2}   
    \]
    \[
      \Longrightarrow \int_{}^{} \frac{4t^2(1+t)}{(t-1)^2(1+t^2)^2}\,dt    
    \]
     
    \item[$a=0$] L'integrale rientra nel caso trattato dalla tecnica precedente
    (funzioni irrazionali di tipo 1).  
\end{description}

\end{tecnica}

\section*{Equazioni differenziali}

\paragraph*{Lineari del I ordine}
     \subparagraph*{Omogenee}
Equazioni nella forma:
     \[
          x'(t) + a(t)y(t) = 0  
     \]
hanno un integrale generale del tipo 
\[
      y = ce^{-A(t)}   \quad dove \ A(t) = \int_{}^{}a(t)dt    
\]

    
     \subparagraph*{Non omogenee}
Equazioni nella forma:
\[
     x'(t) + a(t)x(t) =  f(t) 
\]
dove se $a(t) = 0$ l'equazioni differenziale è lineare, hanno una
soluzione particolare $x_p$
\[
       x_p = e^{-A(t)}(\int_{}^{}e^{A(t)}f(t)dt)
       \quad dove \ A(t) = \int_{}^{}a(t)dt
\]
per cui l'integrale generale è
\[
    x(t)= ce^{-A(t)} + e^{-A(t)}(\int_{}^{}e^{A(t)}f(t)dt) =
    e^{-A(t)}(c + (\int_{}^{}e^{A(t)}f(t)dt))
\]










%%%%%%%%%%%%%%%%%%%%%%%%%%%%%%%%%
%%%% 2 ordine 
%%%%%%%%%%%%%%%%%%%%%%%%%%%%%%%
\paragraph*{II ordine}
\subparagraph*{Omogenee}
Equazioni nella forma:
\[
     ax''(t) + bx'(t) + cx(t)= 0
\]
L'insieme delle soluzioni è uno spazio vettoriale di dimensione
2. Per cui la soluzione generale sarà
\[
     x(t)= c_1 x_1(t) + c_2x_2(t)    
\]
dove $x_1(t)$ e $x_2(t)$ sono basi dello spazio delle soluzioni e 
$c_1,c_2$ sono parametri liberi. \\
Si trovano le soluzioni dell'equazione caratteristica in $\mathbb{C}$:
\[
     az^2 + bz +c = 0    
\]
Si hanno tre casi:
\begin{enumerate}
    \item[$\lambda_1 \neq \lambda_2$] 
    $\rightarrow x(t)= c_1e^{\lambda_1 t}+ c_2e^{\lambda_2 t}$ |
    Base: $e^{\lambda_1 t},e^{\lambda_2 t}$
    \item[$\lambda_1 = \lambda_2$] 
    $\rightarrow x(t)= c_1e^{\lambda t}+ c_2te^{\lambda t}$ |
    Base: $e^{\lambda t}, te^{\lambda t}$
    \item[$\lambda_{1,2} = \alpha \pm i\beta$] 
    $\rightarrow x(t)= c_1e^{\alpha t}cos(\beta t)+ c_2
    e^{\alpha t}sin(\beta t)$ |
    Base: $e^{\alpha t}cos(\beta t),e^{\alpha t}sin(\beta t)$ 
\end{enumerate}

\subparagraph*{Non omogenee:Variazione delle costanti}
Equazioni nella forma:
\[
     ax''(t) + bx'(t) + cx(t)= f(t)
\]
1)Si determina la soluzione generale dell'omogenea associata:
\[
     x_0(t) = c_1 x_1(t) + c_2x_2(t)    
\]
2) Si trova una soluzione particolare nella forma
\[
     x_p(t) = c_1(t) x_1(t) + c_2(t) x_2(t)    
\]
Dal seguente sistema si ricavano le espressioni di $c'_1(t), c'_2(t)$:
\[
   \begin{cases}
    c_1'(t) x_1(t) + c_2'(t) x_2(t) = 0 
    \\
    c_1'(t) x_1'(t) + c_2'(t) x_2'(t) = f(t)
   \end{cases}    
\]
per poi trovare $c_1(t), c_2(t)$ integrando:
\[
    c_1(t) = \int_{}^{} c'_1(t) \, dt \qquad c_2(t) = \int_{}^{} c'_2(t) \,dt
\]
3)La soluzione generale è
\[
      x(t)= x_0(t) + x_p(t)    
\]

\end{document}
