\documentclass[a4paper]{article}
\usepackage[T1]{fontenc}
\usepackage[utf8]{inputenc}
\usepackage[main=italian, english]{babel}
\usepackage{bookmark}
\usepackage[a4paper, total={6in, 9in}]{geometry}
\usepackage{hyperref}

\begin{document}
	\title{Formulario Fisica della Materia}
	\author{Grufoony\\\url{https://github.com/Grufoony/Fisica_UNIBO}}
	\maketitle

    \section{Insieme microcanonico}
        \begin{itemize}
            \item $S=k_B\ln\Omega$ entropia di Boltzmann
        \end{itemize}

    \section{Insieme canonico}
        \begin{itemize}
            \item $\beta=\frac{1}{k_BT}$
            \item $Z_1=\sum_s e^{-\beta\epsilon_s}$ funzione di partizione (particella singola)
            \item $Z=(Z_1)^N$ funzione di partizione (N particelle distinguibili)
            \item $Z=\frac{(Z_1)^N}{N!}$ funzione di partizione (N particelle indistinguibili nel limite diluito)
            \item $F=-k_BT\ln{Z}$ energia libera di Helmoltz
            \item $\left\langle E \right\rangle = \frac{\partial\ln{Z}}{\partial\beta}$ energia media
        \end{itemize}

    \section{Insieme gran canonico}
        \begin{itemize}
            \item $\gamma=-\beta\mu$
            \item $\mu=\frac{\partial F}{\partial N}|_{T,V}=-T\frac{\partial S}{\partial N}|_{U,V}$ potenziale chimico
            \item $\Xi=\sum_Ne^{\beta\mu N}Z(N)$ funzione di partizione gran canonica
            \item $\Phi=k_BT\ln\Xi$ gran potenziale
        \end{itemize}

    \section{Gas ideale}
        \begin{itemize}
            \item $F=-Nk_BT\ln(n_Q)+Nk_BT\ln(N)-Nk_BT\ln(V)-Nk_BT$
            \item $P=\frac{Nk_BT}{V}$
            \item $S=Nk_B\left[\ln\left(\frac{n_Q}{n}\right)+\frac{5}{2}\right]$ equazione di Sackur-Tetrode
            \item $U=\frac{3}{2}Nk_BT$
            \item $C_V=\frac{3}{2}Nk_B$
            \item $\mu=k_BT\ln\left(\frac{n}{n_Q}\right)$
            \item $n_Q=\left(\frac{mk_BT}{2\pi\hbar^2}\right)^{\frac{3}{2}}$ concentrazione quantistica
            \item $\Phi=\frac{n_0}{4}\sqrt{\frac{8k_BT}{\pi m}}$ legge di Graham (effusione)
            \item $l=\frac{1}{n\pi d^2}$ libero cammino medio
            \item $\tau=\frac{l}{v_{rms}}$ tempo medio di collisione
            \item $\theta_t=\frac{\hbar^2\pi^2}{2mk_BL^2}$ temperatura caratteristica traslazioni
        \end{itemize} 

    \section{Teoria di Boltzmann}
        \begin{itemize}
            \item $f(\vec{v})=n_0\left(\frac{m}{2\pi k_BT}\right)^{\frac{3}{2}}e^{-\frac{m}{2k_BT}v^2}$ distribuzione velocità di Maxwell-Boltzmann
            \item $\left\langle v \right\rangle=\sqrt{\frac{8k_BT}{\pi m}}$ velocità media
            \item $v_{rms}=\sqrt{\frac{3k_BT}{m}}$ velocità quadratica media
            \item $v_{p}=\sqrt{\frac{2k_BT}{m}}$ velocità più probabile
        \end{itemize}

    \section{Cose quantistiche}
        \begin{itemize}
            \item $g(E)=\frac{g_s}{4\pi}\left(\frac{2m}{\hbar^2}\right)^{\frac{3}{2}}V\sqrt{E}$ occupazione media per lv energetico (3D)
            \item $\epsilon_j=\hbar\omega\left(j+\frac{1}{2}\right)$ energia oscillatore armonico quantistico
            \item $\left\langle n_s \right\rangle=e^{\beta(\mu-\epsilon_s)}$ distribuzione di Maxwell-Boltzmann
            \item $\Xi_f=\prod_s\left[1+e^{\beta(\mu-\epsilon_s)}\right]$
            \item $\left\langle n_s \right\rangle_f=\frac{1}{e^{\beta(\epsilon_s-\mu)}+1}$ distribuzione di Fermi-Dirac
            \item $\Xi_b=\prod_s\left[\frac{1}{1-e^{\beta(\mu-\epsilon_s)}}\right]$
            \item $\left\langle n_s \right\rangle_b=\frac{1}{e^{\beta(\epsilon_s-\mu)}-1}$ distribuzione di Bose-Einstein
            \item $Vu(\omega)d\omega=\frac{V\hbar}{\pi^2c^3}\frac{\omega^3}{e^{\beta\hbar\omega}-1}d\omega$ equazione di Planck per il corpo nero
            \item $Vu_{RJ}(\omega)d\omega=k_BTg(\omega)d\omega$ approssimazione di Rayleigh-Jeans per le basse frequenze
        \end{itemize}

    \section{Gas di Fermi}
        \begin{itemize}
            \item $\epsilon_F=\frac{\hbar^2}{2m}\left(\frac{6\pi^2n}{g_s}\right)^{\frac{2}{3}}$ energia di Fermi
            \item $\mu(T)\approx\epsilon_F\left(1-\frac{\pi^2}{12}\frac{T^2}{T_F^2}\right)$
            \item $U\approx\frac{3}{5}N\epsilon_F+\frac{\pi^2}{4}\frac{T^2}{T_F^2}N\epsilon_F$
            \item $C_V=\frac{\pi^2}{2}Nk_B\frac{T}{T_F}$
            \item $P=\frac{2}{3}\frac{U}{V}$
            \item $B=\frac{2}{5}n\epsilon_F$ compressibilità
            \item $\Phi=PV$
        \end{itemize}

    \section{Gas di Bose}
        \begin{itemize}
            \item $U=\frac{g_s}{4\pi^2}\left(\frac{2m}{\hbar^2}\right)^{\frac{3}{2}}\Gamma\left(\frac{5}{2}\right)\zeta\left(\frac{5}{2}\right)=0.77Nk_BT\left(\frac{T}{T_0}\right)^{\frac{3}{2}}$ ($T<T_0$)
            \item $C_V=1.93Nk_B\left(\frac{T}{T_0}\right)^{\frac{3}{2}}$ ($T<T_0$)
            \item $U=\frac{3}{2}Nk_BT\left[1-\frac{\zeta\left(\frac{3}{2}\right)}{2^{\frac{5}{2}}}\left(\frac{T_0}{T}\right)^{\frac{3}{2}}+\ldots\right]$ ($T>T_0$)
            \item $C_V=\frac{3}{2}Nk_B\left[1+0.231\left(\frac{T_0}{T}\right)^{\frac{3}{2}}+\ldots\right]$ ($T>T_0$)
        \end{itemize}
    
    \section{Corpo nero}
        \begin{itemize}
            \item $u=\frac{\pi^2}{15}\frac{(k_BT)^4}{(\hbar c)^3}$
            \item $P=\frac{u}{3}$
            \item $\mu=0$
            \item $G=0$
            \item $s=\frac{4}{3}\frac{u}{T}$
            \item $H=\frac{4}{3}U$
            \item $F=-\frac{U}{3}$
            \item $n=\left(\frac{2k_B^3\zeta(3)}{\pi^2c^3\hbar^3}\right)T^3$
        \end{itemize}

    \section{Cose termodinamiche}
        \begin{itemize}
            \item $U=TS-PV+\mu N$ energia interna
            \item $dU=TdS-PdV+\mu dN$
            \item $d\mu=-\frac{S}{N}dT+\frac{V}{N}dP$ relazione di Gibbs-Duhem
            \item $F=U-TS$ energia libera di Helmoltz
            \item $dF=-SdT-PdV+\mu dN$
            \item $H=U+PV$ entalpia
            \item $dH=TdS+VdP+\mu dN$
            \item $G=F+PV=H-TS=U-TS+PV$ energia libera di Gibbs
            \item $dG=-SdT+VdP+\mu dN$
            \item $dU=TdS-PdV$ equazione di Maxwell
            \item $\Phi=\mu N-F$
            \item $S=-\frac{\partial F}{\partial T}|_{N,V}$ entropia
            \item $C_V=\frac{\partial U}{\partial T}|_{V}$ capacità termica
            \item $P=-\frac{\partial U}{\partial V}|_{N,S}=-\frac{\partial F}{\partial V}|_{N,T}=T\frac{\partial S}{\partial V}|_{N,U}$ pressione
            \item $S=\frac{\partial\Phi}{\partial T}|_{\mu, V}$
            \item $P=\frac{\partial\Phi}{\partial V}|_{\mu,T}$
            \item $\left\langle N \right\rangle=\frac{\partial\Phi}{\partial\mu}|_{V,T}$
        \end{itemize}

    \section{Cose matematiche}
        \begin{itemize}
            \item $N!\simeq\sqrt{2\pi N}N^Ne^{-N}$ approssimazione di Stirling ($N>>1$)
            \item $\int_{-\infty}^{\infty}e^{-\alpha x^2}=\sqrt{\frac{\pi}{\alpha}}$ integrale gaussiano
            \item $\frac{d}{d\alpha}\int f(x,\alpha)dx=\int\frac{\delta f}{\delta\alpha}dx$ trucco di Feynman
            \item $\sum_{n=0}^\infty ne^{-an}=\frac{e^a}{(e^a-1)^2}$
            \item $\Gamma(n+1)=\int_0^\infty t^{n}e^{-t}dt=n!$ funzione gamma
            \item $I\approx\int_0^\mu k(\epsilon)d\epsilon+\frac{\pi^2}{6}(k_BT)^2k'(\mu)+o(T^4)$ espansione di Sommerfeld
            \item $\zeta(s)=\sum_{n=1}^\infty\frac{1}{n^s}$ funzione zeta di Riemann
            \item $\int_0^\infty\frac{x^n}{e^x-1}dx=\Gamma(n+1)\zeta(n+1)$
            \item $\left\langle y(\epsilon) \right\rangle=\int_0^\infty g(\epsilon)y(\epsilon)f(\epsilon)d\epsilon$
        \end{itemize}

\end{document}