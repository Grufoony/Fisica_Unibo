\documentclass[a4paper]{article}
\usepackage[T1]{fontenc}
\usepackage[utf8]{inputenc}
\usepackage[main=italian, english]{babel}
\usepackage{bookmark}
\usepackage[a4paper, total={6in, 9in}]{geometry}
\usepackage{hyperref}

\begin{document}
	\title{Formulario Fisica della Materia}
	\author{Grufoony\\\url{https://github.com/Grufoony/Fisica_UNIBO}}
	\maketitle

    \section{Insieme microcanonico}
        \begin{itemize}
            \item $S=k_B\ln\Omega$ entropia di Boltzmann
        \end{itemize}

    \section{Insieme canonico}
        \begin{itemize}
            \item $\beta=\frac{1}{k_BT}$
            \item $Z_1=\sum_s e^{-\beta\epsilon_s}$ funzione di partizione (particella singola)
            \item $Z=(Z_1)^N$ funzione di partizione (N particelle distinguibili)
            \item $Z=\frac{(Z_1)^N}{N!}$ funzione di partizione (N particelle indistinguibili)
            \item $F=-k_BT\ln{Z}$ energia libera di Helmoltz
            \item $\left\langle E \right\rangle = \frac{\partial\ln{Z}}{\partial\beta}$ energia media
        \end{itemize}

    \section{Insieme gran canonico}
        \begin{itemize}
            \item $\gamma=-\beta\mu$
            \item $\mu=\frac{\partial F}{\partial N}|_{T,V}=-T\frac{\partial S}{\partial N}|_{U,V}$ potenziale chimico
            \item $\Xi=\sum_Ne^{\beta\mu N}Z(N)$ funzione di partizione gran canonica
            \item $\Phi=k_BT\ln\Xi$ gran potenziale
        \end{itemize}

    \section{Cose cinetiche}
        \begin{itemize}
            \item $f(v)=\sqrt{\frac{m}{2\pi k_BT}}e^{-\frac{m}{2k_BT}v^2}$ distribuzione velocità di Maxwell-Boltzmann
        \end{itemize}

    \section{Cose quantistiche}
        \begin{itemize}
            \item $n_Q=\left(\frac{mk_BT}{2\pi\hbar^2}\right)^{\frac{3}{2}}$ concentrazione quantistica
            \item $\epsilon_j=\hbar\omega\left(j+\frac{1}{2}\right)$ energia oscillatore armonico quantistico
            \item $\left\langle n_s \right\rangle_f=\frac{1}{e^{\beta(\epsilon_s-\mu)}+1}$ distribuzione di Fermi-Dirac
            \item $\left\langle n_s \right\rangle_f=\frac{1}{e^{\beta(\epsilon_s-\mu)}-1}$ distribuzione di Bose-Einstein
            \item $\epsilon_F=\frac{\hbar^2}{2m}\left(\frac{6\pi^2n}{g_s}\right)^{\frac{2}{3}}$ energia di Fermi
            \item $\mu(T)\approx\epsilon_F\left(1-\frac{\pi^2}{12}\frac{T^2}{T_0^2}\right)$
            \item $Vu(\omega)d\omega=\frac{V\hbar}{\pi^2c^3}\frac{\omega^3}{e^{\beta\hbar\omega}-1}d\omega$ equazione di Plank per il corpo nero
            \item $Vu_{RJ}(\omega)d\omega=k_BTg(\omega)d\omega$ approssimazione di Rayleigh-Jeans per le basse frequenze
        \end{itemize}

    \section{Cose termodinamiche}
        \begin{itemize}
            \item $S=-\frac{\partial F}{\partial T}|_{N,V}$ entropia
            \item $C_V=\frac{\partial U}{\partial T}|_{V}$ capacità termica
            \item $P=-\frac{\partial U}{\partial V}|_{N,S}=-\frac{\partial F}{\partial V}|_{N,T}=T\frac{\partial S}{\partial V}|_{N,U}$ pressione
            \item $S=\frac{\partial\Phi}{\partial T}|_{\mu, V}$
            \item $P=\frac{\partial\Phi}{\partial V}|_{U,T}$
            \item $\left\langle N \right\rangle=\frac{\partial\Phi}{\partial\mu}|_{V,T}$
        \end{itemize}

    \section{Cose matematiche}
        \begin{itemize}
            \item $N!\simeq\sqrt{2\pi N}N^Ne^{-N}$ approssimazione di Stirling ($N>>1$)
            \item $\int_{-\infty}^{\infty}e^{-\alpha x^2}=\sqrt{\frac{\pi}{\alpha}}$ integrale gaussiano
            \item $\frac{d}{d\alpha}\int f(x,\alpha)dx=\int\frac{\delta f}{\delta\alpha}dx$ trucco di Feynman
            \item $\sum_{n=0}^\infty ne^{-an}=\frac{e^a}{(e^a-1)^2}$
            \item $\Gamma(n+1)=\int_0^\infty t^{n}e^{-t}dt=n!$ funzione gamma
            \item $I\approx\int_0^\mu k(\epsilon)d\epsilon+\frac{\pi^2}{6}(k_BT)^2k'(\mu)+o(T^4)$ espansione di Sommerfeld
            \item $\zeta(s)=\sum_{n=1}^\infty\frac{1}{n^s}$ funzione zeta di Riemann
        \end{itemize}

\end{document}