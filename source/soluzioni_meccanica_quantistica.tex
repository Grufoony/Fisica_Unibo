\documentclass[a4paper]{article}
\usepackage[T1]{fontenc}
\usepackage[utf8]{inputenc}
\usepackage[main=italian, english]{babel}
\usepackage{bookmark}
\usepackage[a4paper, total={6in, 9in}]{geometry}
\usepackage{hyperref}
\usepackage{amsmath, amsthm, amssymb, amsfonts}
\usepackage{pgfplots}
\usepackage{tikz}
\usepackage{caption}
\usepackage{tikz-network}
\usepackage{float}
 
\pgfplotsset{compat = newest}

\begin{document}
	\title{Soluzioni esercizi Meccanica Quantistica}
	\author{Grufoony\\\url{https://github.com/Grufoony/Fisica_UNIBO}}
	\maketitle

    \section*{Esercizio 1 (esame del 09/01/2017)}
        Una molecola di idrogeno ionizzata può essere descritta come un sistema unidimensionale formato da un singolo elettrone soggetto al potenziale
        \begin{equation*}
            U(x)=-\frac{\hbar^2\Omega}{m}\left[\delta(x-a)+\delta(x+a)\right]
        \end{equation*}
        ove $a>0$ è la semidistanza dei due protoni pensati fissi e $\Omega>0$ ha le dimensioni di una lunghezza inversa.
        Trovare gli autovalori e le autofunzioni dell'energia dello spettro discreto.
        \\
        \\
        Scriviamo subito l'equazione di Schroedinger
        \begin{equation*}
            \frac{d^2}{dx^2}\Phi+\left\{k^2-2\Omega\left[\delta(x-a)+\delta(x+a)\right]\right\}\Phi=0
        \end{equation*}
        La soluzione è nota
        \begin{equation*}
            \Phi^\pm(x)=
            \begin{cases}
                A^\pm e^{ikx}+B^\pm e^{-ikx}\quad x<-a\\
                C^\pm e^{ikx}+D^\pm e^{-ikx}\quad -a<x<a\\
                E^\pm e^{ikx}+F^\pm e^{-ikx}\quad a<x
            \end{cases}
        \end{equation*}
        Notiamo che il potenziale ha partità definita $U(-x)=U(x)$ allora possiamo imporre la condizione $\Phi^\pm(-x)=\pm\Phi^\pm(x)$, ottenendo
        \begin{equation*}
            \begin{cases}
                E^\pm=\pm A^\pm\\
                F^\pm=\pm B^\pm\\
                D^\pm=\pm C^\pm
            \end{cases}
        \end{equation*}
        La funzione assume ora la forma più semplice
        \begin{equation*}
            \Phi^\pm(x)=
            \begin{cases}
                C^\pm \left(e^{ik\left\lvert x\right\rvert }\pm e^{-ik\left\lvert x\right\rvert }\right)\quad \left\lvert x\right\rvert<a\\
                \sigma_\pm(x)\left[A^\pm e^{ik\left\lvert x\right\rvert }+B^\pm e^{-ik\left\lvert x\right\rvert }\right]\quad a<\left\lvert x\right\rvert
            \end{cases}
        \end{equation*}
        Notiamo però che la funzione d'onda non può esplodere all'infinito, quindi necessariamente si avrà $A^\pm=0$ e, riscalando $B^\pm=B^\pm e^{ika}$
        \begin{equation*}
            \Phi^\pm(x)=
            \begin{cases}
                C^\pm \left(e^{ik\left\lvert x\right\rvert }\pm e^{-ik\left\lvert x\right\rvert }\right)\quad \left\lvert x\right\rvert<a\\
                \sigma_\pm(x)B^\pm e^{-ik(\left\lvert x\right\rvert-a)}\quad a<\left\lvert x\right\rvert
            \end{cases}
        \end{equation*}
        Ora imponiamo la continuità della funzione e della sua derivata 
        \begin{equation}
            \begin{cases}
                \sigma_\pm(x)B^\pm=C^\pm \left(e^{ika}\pm e^{-ika}\right)\\
                -\sigma_\pm(x)B^\pm-C^\pm \left(e^{ika}\mp e^{-ika}\right)=\frac{2i\Omega}{k}\sigma_\pm(x)B^\pm
            \end{cases}
        \end{equation}
        Si può riscrivere il tutto nella forma
        \begin{equation}
            \begin{cases}
                \sigma_\pm(x)B^\pm=C^\pm \left(e^{ika}\pm e^{-ika}\right)\\
                \sigma_\pm(x)B^\pm\left(1+\frac{2i\Omega}{k}\right)=C^\pm \left(e^{-ika}\mp e^{ika}\right)
            \end{cases}
        \end{equation}
        Dividendo la seconda per la prima si ottiene
        \begin{equation*}
            \frac{e^{-ika}\mp e^{ika}}{e^{ika}\pm e^{-ika}}=1+\frac{2i\Omega}{k}
        \end{equation*}
        Le soluzioni in k per le due parità sono
        \begin{equation}
            \begin{cases}
                \tanh(ika)=-\frac{2i\Omega+k}{k}\\
                \tanh(ika)=\frac{k}{2i\Omega+k}
            \end{cases}
        \end{equation}
        Ogni equazione mi fornisce una soluzione quindi, posto $\tilde{k}=ik$, ho due soli livelli energetici corrispondenti ai valori di $\tilde{k}^\pm_0$.
        \begin{center}
            \begin{tikzpicture}
                \begin{axis}[
                    xmin = -10, xmax = 10,
                    ymin = 0, ymax =1.7,
                    samples = 200,
                    smooth,
                    thick,
                    xlabel = {$x$},
                    ylabel = {$\Phi^+(x)$},]
                    \addplot[
                        domain = -4:4,
                    ] {0.05*cosh(x)};
                    \addplot[
                        domain = 4:10,
                    ] {1.4*e^(-x+4)};
                    \addplot[
                        domain = -10:-4,
                    ] {1.4*e^(+x+4)};
                \end{axis}
            \end{tikzpicture}
        \end{center}
        \begin{center}
            \begin{tikzpicture}
                \begin{axis}[
                    xmin = -10, xmax = 10,
                    ymin = -2, ymax =2,
                    samples = 200,
                    smooth,
                    thick,
                    xlabel = {$x$},
                    ylabel = {$\Phi^-(x)$},]
                    \addplot[
                        domain = -4:4,
                    ] {0.05*sinh(x)};
                    \addplot[
                        domain = 4:10,
                    ] {1.4*e^(-x+4)};
                    \addplot[
                        domain = -10:-4,
                    ] {-1.4*e^(+x+4)};
                \end{axis}
            \end{tikzpicture}
        \end{center}
        
    \section*{Esercizio 1 (esame del 03/07/2017)}
    Una particlella è confinata in una regione sferica di raggio $R$ ed è soggetta ad un campo di forza centrale con potenziale
    \begin{equation*}
        U(r)=
        \begin{cases}
            0,\quad\quad 0\leq r\leq\frac{R}{2}\\
            U_0\quad\quad \frac{R}{2}<r\leq R
        \end{cases}
        \quad\quad U_0\neq 0
    \end{equation*}
    ove $U_0 > 0$ è una costante con le dimensioni di un'energia. Calcolare gli autovalori e le autofunzioni dell'energia della particella.
    Suggerimento: Derivare ma non tentare di risolvere l'equazione agli autovalori trascendente.
    \\
    \\
    Scriviamo subito l'equazione di Schroedinger radiale
    \begin{equation*}
        \frac{d^2\chi}{dr^2}+\left[k^2-\frac{l(l-1)}{r^2}-\frac{2m}{\hbar^2}U(r)\right]\chi=0
    \end{equation*}
    la cui soluzione è
    \begin{equation*}
        \chi(r)=
        \begin{cases}
            Akrj_l(kr)+Bkrn_l(kr)\quad\quad 0\leq r\leq\frac{R}{2}\\
            Ck'rj_l(k'r)+Dk'rn_l(k'r)\quad\quad \frac{R}{2}<r\leq R
        \end{cases}
    \end{equation*}
    con $k'^2=k^2-\frac{2m}{\hbar^2}U_0$
    Condizione al bordo $\chi(0)=0$ quindi $B=0$.
    Continuità della funzione
    \begin{equation*}
        Ak\frac{R}{2}j_l(k\frac{R}{2})=Ck'\frac{R}{2}j_l(k'\frac{R}{2})+Dk'\frac{R}{2}n_l(k'\frac{R}{2})
    \end{equation*}
    Ponendo $x_k=\frac{kR}{2}$, $x_{k'}=\frac{k'R}{2}$, $C=En_l(2x_{k'})$ e $D=-Ej_l(2x_{k'})$ si ottiene
    \begin{equation*}
        \chi(r)=
        \begin{cases}
            Akrj_l(kr)\quad\quad 0\leq r\leq\frac{R}{2}\\
            Ek'r\left[n_l(2x_{k'})j_l(k'r)-j_l(2x_{k'})n_l(k'r)\right]\quad\quad \frac{R}{2}<r\leq R
        \end{cases}
    \end{equation*}
    Continuità della derivata
    \begin{equation*}
        \begin{split}
            A\left\{x_kj_l(x_k)+x_k^2j_l'(x_k)\right\}&=E\left\{n_l(2x_{k'})\left[x_{k'}j_l(x_{k'})+x_{k'}^2j_l'(x_{k'})\right]-j_l(2x_{k'})\left[x_{k'}n_l(x_{k'})+x_{k'}^2n_l'(x_{k'})\right]\right\}\\
            Ax_k^2j_l'(x_k)&=E\left[n_l(2x_{k'})x_{k'}^2j_l'(x_{k'})-j_l(2x_{k'})x_{k'}^2n_l'(x_{k'})\right]
        \end{split}
    \end{equation*}
    I $k$ sono forniti quindi dal sistema di equazioni (risolvibile)
    \begin{equation*}
        \begin{cases}
            Ax_kj_l(x_k)&=E\left[n_l(2x_{k'})x_{k'}j_l(x_{k'})-j_l(2x_{k'})x_{k'}n_l(x_{k'})\right]\\
            Ax_k^2j_l'(x_k)&=E\left[n_l(2x_{k'})x_{k'}^2j_l'(x_{k'})-j_l(2x_{k'})x_{k'}^2n_l'(x_{k'})\right]
        \end{cases}
    \end{equation*}


    % \section*{Esercizio 2 (esame del 08/09/2017)}
        % Usando l'approssimazione di Born, calcolare la sezione d'urto differenziale e totale per la diffusione dal potenziale centrale
        % \begin{equation*}
        %     U(r)=\frac{U_0}{1+\frac{r^2}{R^2}}
        % \end{equation*}
        % ove $U_0\neq 0$ e $R > 0$ sono costanti con le dimensioni di un'energia e di una lunghezza rispettivamente.
        % \\
        % \\
        % Dalla teoria è nota la sezione d'urto differenziale
        % \begin{equation*}
        %     \frac{d\sigma_k(\theta)}{d\Omega}=\left(\frac{2m}{\hbar^2}\right)^2\left[\frac{1}{q}\int_0^\infty r\sin(qr)U(r)dr\right]^2
        % \end{equation*}
        % Risolviamo prima l'integrale ponendo $x=\frac{r}{R}$ e considerando $q>0$:
        % \begin{equation*}
        %     \begin{split}
        %         \int_0^\infty r\sin(qr)U(r)dr&=\int_0^\infty \frac{Rx\sin(qRx)U_0}{1+x^2}Rdx=\\
        %         &=R^2U_0\int_0^\infty \frac{x\sin(qRx)}{1+x^2}dx=\\
        %         &=R^2U_0\Im \left\{\int_0^\infty \frac{xe^{iqRx}}{1+x^2}dx\right\}=\\
        %         &=\frac{R^2U_0}{2}\Im \left\{\int_{-\infty}^\infty \frac{xe^{iqRx}}{1+x^2}dx\right\}=\\
        %         &=\frac{R^2U_0}{2}\Im \left\{2\pi i\frac{ze^{iqRz}}{2z}|_{z=i}\right\}=\\
        %         &=\frac{R^2U_0}{2}\Im \left\{\pi ie^{qR}\right\}=\\
        %         &=\frac{\pi R^2U_0}{2}e^{qR}
        %     \end{split}
        % \end{equation*}
        % \begin{center}
        %     \begin{tikzpicture}
        %         \begin{axis}[
        %             xmin = -5, xmax = 5,
        %             ymin = -5, ymax = 5,
        %             samples = 200,
        %             smooth,
        %             thick,
        %             xlabel = {$Re$},
        %             ylabel = {$Im$},]
        %             \draw [black,-stealth]
        %             (-4,0)
        %             -- % = line-to
        %             ++ % = calculate a vector sum
        %             (axis direction cs:8,0);
                    
        %             \draw [black] (1,0) circle [radius=2];

        %             \addplot [only marks,mark=*]
        %                 coordinates { (0,1) };
        %                 % label = {$i$}
        %         \end{axis}
        %     \end{tikzpicture}
        % \end{center}
        % Considerando anche il caso in cui $q<0$ si nota facilmente come:
        % \begin{equation*}
        %     \int_0^\infty r\sin(qr)U(r)dr=\frac{\pi R^2U_0}{2}e^{\left\lvert q\right\rvert R}
        % \end{equation*}
        % Quindi la sezione d'urto differenziale diviene
        % \begin{equation*}
        %     \frac{d\sigma_k(\theta)}{d\Omega}=\left(\frac{m\pi R^2U_0}{\hbar^2}\right)^2\frac{e^{2\left\lvert q\right\rvert R}}{q^2}
        % \end{equation*}

    \section*{Esercizio 1 (esame del 15/01/2018)}
        Una particella di massa $m$ è confinata nello semispazio unidimensionale $x\geq0$ ed è soggetta al potenziale
        \begin{equation*}
            U(x)=\frac{\hbar^2}{m}\Omega\delta(x-a)
        \end{equation*}
        ove $a>0$ e $\Omega\neq0$ sono costanti con le dimensioni di una lunghezza ed una lunghezza inversa, rispettivamente. 
        Trovare gli autovalori e le autofunzioni della energia.
        \\
        \\
        Scriviamo subito l'equazione di Schroedinger
        \begin{equation*}
            \frac{d^2}{dx^2}\Phi+\left\{k^2+2\Omega\delta(x-a)\right\}\Phi=0
        \end{equation*}
        La soluzione generica è
        \begin{equation*}
            \Phi(x)=
            \begin{cases}
                A\sin(kx)+B\cos(kx)\quad 0<x<a\\
                C\sin(kx)+D\cos(kx)\quad a<x
            \end{cases}
        \end{equation*}
        Condizioni al bordo $\Phi(0)=0$ quindi $B=0$.
        Imponiamo ora la continuità
        \begin{equation*}
            \begin{cases}
                A\sin(ka)=C\sin(ka)+D\cos(ka)\\
                Ck\cos(ka)-Dk\sin(ka)-Ak\cos(ka)=2\Omega A\sin(ka)
            \end{cases}
        \end{equation*}
        Facendo qualche conto si ottiene
        \begin{equation*}
            \begin{cases}
                A=C+D\cot(ka)\\
                Ck-Dk\tan(ka)-Ak=2\Omega A\tan(ka)
            \end{cases}
        \end{equation*}
        I $k_n$ sono definiti quindi dall'equazione
        \begin{equation*}
            \frac{Dk}{2\Omega}=-\frac{C\tan(ka)+D}{\tan(ka)+\cot(ka)}
        \end{equation*}
        e i livelli energetici sono dati da
        \begin{equation*}
            w_n=\frac{\hbar^2}{2m}k_n^2
        \end{equation*}
        La funzione d'onda è infine
        \begin{equation*}
            \Phi(x)=
            \begin{cases}
                \left[C+D\cot(ka)\right]\sin(kx)\quad 0<x<a\\
                C\sin(kx)+D\cos(kx)\quad a<x
            \end{cases}
        \end{equation*}
        \begin{center}
            \begin{tikzpicture}
                \begin{axis}[
                    xmin = 0, xmax = 5,
                    ymin = -2, ymax =2,
                    samples = 200,
                    smooth,
                    thick,
                    xlabel = {$x$},
                    ylabel = {$\Phi(x)$},]
                    \addplot[
                        domain = 0:1,
                    ] {(1+cot(deg(1)))*sin(deg(x))};
                    \addplot[
                        domain = 1:5,
                    ] {sin(deg(x))+cos(deg(x))};
                \end{axis}
            \end{tikzpicture}
        \end{center}
        
    \end{document}