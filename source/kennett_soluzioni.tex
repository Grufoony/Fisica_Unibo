\documentclass[a4paper]{article}
\usepackage[T1]{fontenc}
\usepackage[utf8]{inputenc}
\usepackage[main=italian, english]{babel}
\usepackage{bookmark}
\usepackage[a4paper, total={6in, 9in}]{geometry}
\usepackage{hyperref}
\usepackage{amsmath, amsthm, amssymb, amsfonts}

\begin{document}
	\title{Soluzioni esercizi Kennett}
	\author{Grufoony\\\url{https://github.com/Grufoony/Fisica_UNIBO}}
	\maketitle

    \section{Esercizio 9.5}
    (a) Mostrare che la funzione di partizione gran canonica per la radiazione di corpo nero assume la forma
    \begin{equation}
        \Xi=\prod_s\frac{1}{1-e^{-\beta\epsilon_s}}
        \label{equation:9.5a}
    \end{equation}
    \\
    \\
    La radiazione di corpo nero è costituita da fotoni, ossia bosoni.
    Dalla teoria (vista in classe) è nota la funzione di partizione gran canonica bosonica:
    \begin{equation*}
        \Xi_b=\prod_s\frac{1}{1-e^{\beta(\mu-\epsilon_s})}
    \end{equation*}
    Osservando ora come si possano aggiungere/rimuovere fotoni ad un corpo nero senza alcun costo energetico ($\mu=0$) la \ref{equation:9.5a} diviene ovvia.
    \\
    \\
    (b) Usando l'equazione \ref{equation:9.5a} oppure in altro modo si dimostri che l'entropia per unità di volume della radiazione di corpo nero a temperatura $T$ assume la forma
    \begin{equation}
        s=\frac{4\pi^2k_B^4T^3}{45\hbar^3c^3}
    \end{equation}
    \\
    \\
    Richiamiamo ora l'energia per unità di volume della radiazione di corpo nero (eq. 9.20 Kennett)
    \begin{equation*}
        u=\frac{\pi^2}{15}\frac{(k_BT)^4}{(\hbar c)^3}
    \end{equation*}
    e anche la sua pressione (eq. 9.29 Kennett)
    \begin{equation*}
        P=\frac{u}{3}
    \end{equation*}
    \\
    Richiamando ora la relazione termodinamica
    \begin{equation}
        U=TS-PV-\mu N
    \end{equation}
    ponendo $\mu=0$ (vedi punto a) e dividendo per il volume $V$, si ottiene
    \begin{equation*}
        s=\frac{u+P}{T}
    \end{equation*}
    che unita alle precedenti conclude la dimostrazione
    \begin{equation*}
        s=s=\frac{u+\frac{u}{3}}{T}=\frac{4}{3}\frac{u}{T}=\frac{4\pi^2k_B^4T^3}{45\hbar^3c^3}
    \end{equation*}
\end{document}