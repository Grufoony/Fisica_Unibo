\documentclass[a4paper]{article}
\usepackage[T1]{fontenc}
\usepackage[utf8]{inputenc}
\usepackage[main=italian, english]{babel}
\usepackage{bookmark}
\usepackage[a4paper, total={6in, 9in}]{geometry}
\usepackage{hyperref}
\usepackage{amsmath, amsthm, amssymb, amsfonts}

\begin{document}
	\title{Soluzioni esercizi Kennett}
	\author{Grufoony\\\url{https://github.com/Grufoony/Fisica_UNIBO}}
	\maketitle

    \section*{Ricavare Sackur-Tetrode}
        Per un gas perfetto è nota la
        \begin{equation*}
            F=Nk_BT\left[\ln\left(\frac{n}{n_Q}\right)-1\right]
        \end{equation*}
        Per calcolare l'entropia è sufficiente utilizzare la relazione
        \begin{equation*}
            S=-\frac{\partial F}{\partial T}
        \end{equation*}
        ricordandosi che la concentrazione quantistica è definita
        \begin{equation*}
            n_Q=\left(\frac{mk_BT}{2\pi\hbar^2}\right)^{\frac{3}{2}}
        \end{equation*}
        Svolgendo i calcoli
        \begin{equation*}
            \begin{split}
                S&=-\left\{Nk_B\left[\ln\left(\frac{n}{n_Q}\right)-1\right]-Nk_BT\frac{1}{n_Q}\left(\frac{mk_B}{2\pi\hbar^2}\right)^{\frac{3}{2}}\frac{3}{2}T^{\frac{1}{2}}\right\}\\
                &=-\left\{Nk_B\left[\ln\left(\frac{n}{n_Q}\right)-1-\frac{3}{2}\right]\right\}=\\
                &=Nk_B\left[\ln\left(\frac{n_Q}{n}\right)+\frac{5}{2}\right]
            \end{split}
        \end{equation*}

    \section*{Esercizio 5.4}
        La pressione sul monte Everset è $P_{Ev}\frac{1}{3}P_{atm}$ con $P_{atm}=101.3kPa$ e la temperatura è $T_{Ev}\simeq 30°C$.
        Calcolare il libero cammino medio.
        \\
        \\
        Dalla teoria è nota la formula pe ril libero cammino medio
        \begin{equation*}
            l=\frac{1}{n\pi d^2}
        \end{equation*}
        Dato che sempre di aria si tratta (sia sull'Everest che al livello del mare) e che non sappiamo $d$, risluta più comodo trovare
        \begin{equation*}
            \frac{l_{Ev}}{l_{atm}}=\frac{n_{atm}}{n_{Ev}}
        \end{equation*}
        Essendo l'aria in buona approssimazione un gas perfetto si può scrivere la densità come $n=\frac{P}{k_BT}$ e, inserendo il tutto nella relazione precedente (assumendo $T_{atm}\simeq300K$)
        \begin{equation*}
            \frac{l_{Ev}}{l_{atm}}=\frac{T_{Ev}P_{atm}}{T_{atm}P_{Ev}}=3\frac{T_{Ev}}{T_{atm}}=2.67
        \end{equation*}
        Dalla teoria è noto $l_{atm}\simeq 1\mu m$ quindi $l_{Ev}\simeq 2.67\mu m$

    \section*{Esercizio 5.5}
        Nello spazio intersetllare sono presenti giganti nubi di idrogeno molecolare (lunghezza di legame $0.74\times 10^{-10}m$).
        Sapendo che la massa di una nube è $m\sim 4\times 10^{45}kg$, il diametro è $D\sim 1.42\times 10^{18}m$ e la temperatura è $T\sim 10K$ calcolare il libero cammino medio e il tempo di collisione medio.
        \\
        \\
        Calcolando il volume della nube
        \begin{equation*}
            V=\frac{4}{3}\pi\left(\frac{D}{2}\right)^3=\frac{\pi}{6}D^3
        \end{equation*}
        e il numero di molecole $N=\frac{A}{2N_A}$ si ottiene
        \begin{equation*}
            l=\frac{V}{N\pi d^2}=\frac{D^3}{3N_Ad^2}=7.24\times 10^{11}m
        \end{equation*}
        Il tempo di collisione analogamente è
        \begin{equation*}
            \tau=\sqrt{\frac{m}{3k_BT}}l=l\sqrt{\frac{A}{3N_Ak_BT}}=6.48\times 10^{10}s
        \end{equation*}

    \section*{Esercizio 6.4}
        Calcolare $U$ e $C_V$ per un sistema a due livelli popolato da bosoni.
        \\
        \\
        Dalla teoria è nota la
        \begin{equation*}
            \Xi_b=\prod_s\left[\frac{1}{1-e^{\beta(\mu-\epsilon_s)}}\right]
        \end{equation*}
        Assumiamo i due livelli energetici come
        \begin{equation*}
            \begin{cases}
                \epsilon_0=0\\
                \epsilon_1=\epsilon
            \end{cases}
        \end{equation*}
        e in questo caso la funzione di partizione gran canonica diviene
        \begin{equation*}
            \Xi=\frac{1}{(1-e^{\beta\mu})(1-e^{\beta(\mu-\epsilon)})}
        \end{equation*}
        Dalla definizione di energia interna si può utilizzare il trucco di Feynman
        \begin{equation*}
            \begin{split}
                U&=\frac{1}{\Xi}\sum_{N=0}^\infty\sum_{s}^Ns\epsilon e^{\beta(N\mu-s\epsilon)}=\\
                &=-\frac{1}{\Xi}\frac{\epsilon}{\beta}\sum_{N=0}^\infty\sum_{s}^N\frac{\partial}{\partial\epsilon}e^{\beta(N\mu-s\epsilon)}=\\
                &=-\frac{1}{\Xi}\frac{\epsilon}{\beta}\frac{\partial}{\partial\epsilon}\sum_{N=0}^\infty\sum_{s}^Ne^{\beta(N\mu-s\epsilon)}=\\
                &=-\frac{1}{\Xi}\frac{\epsilon}{\beta}\frac{\partial\Xi}{\partial\epsilon}=\\
                &=-\frac{\epsilon}{\beta}\frac{\partial\ln\Xi}{\partial\epsilon}\\
            \end{split}
        \end{equation*}
        Possiamo ora calcolare l'energia interna
        \begin{equation*}
            \begin{split}
                U&=-\frac{\epsilon}{\beta}\frac{\partial\ln\Xi}{\partial\epsilon}=\\
                &=\frac{\epsilon}{\beta}\frac{\partial\ln(1-e^{\beta\mu})(1-e^{\beta(\mu-\epsilon)})}{\partial\epsilon}=\\
                &=\frac{\epsilon}{\beta}\frac{\partial\ln(1-e^{\beta(\mu-\epsilon)})}{\partial\epsilon}=\\
                &=\frac{\epsilon}{\beta}\frac{\beta e^{\beta(\mu-\epsilon)}}{(1-e^{\beta(\mu-\epsilon)})}=\\
                &=\epsilon\frac{e^{\beta(\mu-\epsilon)}}{(1-e^{\beta(\mu-\epsilon)})}=\\
                &=\frac{\epsilon}{e^{\beta(\epsilon-\mu)}-1}
            \end{split}
        \end{equation*}
        \\
        Per la capacità termica è sufficiente utilizzare la relazione
        \begin{equation*}
            \begin{split}
                C_V&=\frac{\partial U}{\partial T}=\\
                &=-\frac{\epsilon}{(e^{\beta(\epsilon-\mu)}-1)^2}\frac{\epsilon-\mu}{k_B}\left(-\frac{1}{T^2}\right)e^{\beta(\epsilon-\mu)}=\\
                &=\frac{k_B\beta^2\epsilon(\epsilon-\mu)}{(e^{\beta(\epsilon-\mu)}-1)(1-e^{\beta(\mu-\epsilon)})}
            \end{split}
        \end{equation*}

    \section*{Esercizio 9.5}
        (a) Mostrare che la funzione di partizione gran canonica per la radiazione di corpo nero assume la forma
        \begin{equation}
            \Xi=\prod_s\frac{1}{1-e^{-\beta\epsilon_s}}
            \label{equation:9.5a}
        \end{equation}
        \\
        \\
        La radiazione di corpo nero è costituita da fotoni, ossia bosoni.
        Dalla teoria (vista in classe) è nota la funzione di partizione gran canonica bosonica:
        \begin{equation*}
            \Xi_b=\prod_s\frac{1}{1-e^{\beta(\mu-\epsilon_s})}
        \end{equation*}
        Osservando ora come si possano aggiungere/rimuovere fotoni ad un corpo nero senza alcun costo energetico ($\mu=0$) la \ref{equation:9.5a} diviene ovvia.
        \\
        \\
        (b) Usando l'equazione \ref{equation:9.5a} oppure in altro modo si dimostri che l'entropia per unità di volume della radiazione di corpo nero a temperatura $T$ assume la forma
        \begin{equation}
            s=\frac{4\pi^2k_B^4T^3}{45\hbar^3c^3}
        \end{equation}
        \\
        \\
        Richiamiamo ora l'energia per unità di volume della radiazione di corpo nero (eq. 9.20 Kennett)
        \begin{equation*}
            u=\frac{\pi^2}{15}\frac{(k_BT)^4}{(\hbar c)^3}
        \end{equation*}
        e anche la sua pressione (eq. 9.29 Kennett)
        \begin{equation*}
            P=\frac{u}{3}
        \end{equation*}
        \\
        Richiamando ora la relazione termodinamica
        \begin{equation}
            U=TS-PV-\mu N
        \end{equation}
        ponendo $\mu=0$ (vedi punto a) e dividendo per il volume $V$, si ottiene
        \begin{equation*}
            s=\frac{u+P}{T}
        \end{equation*}
        che unita alle precedenti conclude la dimostrazione
        \begin{equation*}
            s=s=\frac{u+\frac{u}{3}}{T}=\frac{4}{3}\frac{u}{T}=\frac{4\pi^2k_B^4T^3}{45\hbar^3c^3}
        \end{equation*}

    \section*{Esercizio 9.9}
        Calcolare la temperatura critica per un BEC in 2 dimensioni.
        \\
        \\
        Richiamiamo ora la densità degli stati 2D calcolata in precedenza:
        \begin{equation*}
            g(\epsilon)=g_s\frac{m}{2\pi\hbar^2}L^2
        \end{equation*}
        Possiamo calcolare il numero medio di particelle con la classica formula
        \begin{equation*}
            N=\int_0^\infty g(\epsilon)f(\epsilon)d\epsilon
        \end{equation*}
        la quale, posto $n=\frac{N}{L^2}$ diventa
        \begin{equation*}
            n=g_s\frac{m}{2\pi\hbar^2}\int_0^\infty \frac{1}{e^{\beta(\epsilon-\mu)}-1}d\epsilon
        \end{equation*}
        Per la temperatura critica si ha $e^{\beta_0}>>1$ e si può approssimare
        \begin{equation*}
            n\approx g_s\frac{m}{2\pi\hbar^2}\int_0^\infty e^{-\beta(\epsilon-\mu)}d\epsilon
        \end{equation*}
        la cui soluzione è
        \begin{equation*}
            n=g_s\frac{mk_B}{2\pi\hbar^2}T_0
        \end{equation*}
        Invertendo l'equazione precedente si ottiene la relazione desiderata
        \begin{equation*}
            T_0=\frac{2\pi\hbar^2}{mk_B}\frac{n}{g_s}
        \end{equation*}
\end{document}