\documentclass[a4paper]{article}
\usepackage[T1]{fontenc}
\usepackage[utf8]{inputenc}
\usepackage[main=italian, english]{babel}
\usepackage{bookmark}

% THIS SOFTWARE IS PROVIDED BY THE COPYRIGHT HOLDERS AND CONTRIBUTORS "AS IS"
% AND ANY EXPRESS OR IMPLIED WARRANTIES, INCLUDING, BUT NOT LIMITED TO, THE
% IMPLIED WARRANTIES OF MERCHANTABILITY AND FITNESS FOR A PARTICULAR PURPOSE ARE
% DISCLAIMED. IN NO EVENT SHALL THE COPYRIGHT HOLDER OR CONTRIBUTORS BE LIABLE
% FOR ANY DIRECT, INDIRECT, INCIDENTAL, SPECIAL, EXEMPLARY, OR CONSEQUENTIAL
% DAMAGES (INCLUDING, BUT NOT LIMITED TO, PROCUREMENT OF SUBSTITUTE GOODS OR
% SERVICES; LOSS OF USE, DATA, OR PROFITS; OR BUSINESS INTERRUPTION) HOWEVER
% CAUSED AND ON ANY THEORY OF LIABILITY, WHETHER IN CONTRACT, STRICT LIABILITY,
% OR TORT (INCLUDING NEGLIGENCE OR OTHERWISE) ARISING IN ANY WAY OUT OF THE USE
% OF THIS SOFTWARE, EVEN IF ADVISED OF THE POSSIBILITY OF SUCH DAMAGE.

\usepackage[a4paper, total={6in, 9in}]{geometry}
\usepackage{hyperref}
\usepackage{amsmath, amsthm, amssymb, amsfonts}
\usepackage{pgfplots}
\usepackage{tikz}
\usepackage{caption}
\usepackage{tikz-network}
\usepackage{float}
 
\pgfplotsset{compat = newest}

\begin{document}
	\title{Soluzioni esercizi Fisica della Materia}
	\author{Grufoony\\\url{https://github.com/Grufoony/Fisica_UNIBO}}
	\maketitle

    \section*{Ricavare Sackur-Tetrode}
        Per un gas perfetto è nota la
        \begin{equation*}
            F=Nk_BT\left[\ln\left(\frac{n}{n_Q}\right)-1\right]
        \end{equation*}
        Per calcolare l'entropia è sufficiente utilizzare la relazione
        \begin{equation*}
            S=-\frac{\partial F}{\partial T}
        \end{equation*}
        ricordandosi che la concentrazione quantistica è definita
        \begin{equation*}
            n_Q=\left(\frac{mk_BT}{2\pi\hbar^2}\right)^{\frac{3}{2}}
        \end{equation*}
        Svolgendo i calcoli
        \begin{equation*}
            \begin{split}
                S&=-\left\{Nk_B\left[\ln\left(\frac{n}{n_Q}\right)-1\right]-Nk_BT\frac{1}{n_Q}\left(\frac{mk_B}{2\pi\hbar^2}\right)^{\frac{3}{2}}\frac{3}{2}T^{\frac{1}{2}}\right\}\\
                &=-\left\{Nk_B\left[\ln\left(\frac{n}{n_Q}\right)-1-\frac{3}{2}\right]\right\}=\\
                &=Nk_B\left[\ln\left(\frac{n_Q}{n}\right)+\frac{5}{2}\right]
            \end{split}
        \end{equation*}
        \begin{center}
            \begin{tikzpicture}
                \begin{axis}[
                    xmin = 0, xmax = 2,
                    ymin = -7, ymax = 5,
                    samples = 200,
                    smooth,
                    thick,
                    xlabel = {$T$},
                    ylabel = {$\frac{S}{Nk_B}$},]
                    \addplot[
                        domain = -2:2,
                    ] {5/2*(ln(x)+1)};
                \end{axis}
            \end{tikzpicture}
        \end{center}

    \section*{Integrale di Bose}
        \begin{equation*}
            \begin{split}
                I=\int_0^\infty\frac{x^n}{e^x-1}dx&=\int_0^\infty\frac{x^ne^{-x}}{1-e^{-x}}dx=\\
                &=\int_0^\infty x^ne^{-x}\sum_{k=0}^\infty e^{-kx}dx=\\
                &=\sum_{k=0}^\infty\int_0^\infty x^ne^{-x(k+1)}dx
            \end{split}
        \end{equation*}
        Ponendo ora $h=k+1$ e $y=hx$ e  si ottiene
        \begin{equation*}
            \begin{split}
                I&=\sum_{h=1}^\infty\int_0^\infty \frac{y^n}{h^{n+1}}e^{-y}dy=\\
                &=\sum_{h=1}^\infty\frac{1}{h^{n+1}}\int_0^\infty y^ne^{-y}dy=\\
                &=\zeta(n+1)\Gamma(n+1)
            \end{split}
        \end{equation*}

    \section*{Esercizio 4.17 (Kennett)}
        Considerare un sistema di spin non interagenti, ciascuno con energia $\epsilon=-s\mu B$, con $s=\pm 1$.
        Trovare l'energia libera di Helmholtz per un sistema di $N$ spin in un solido.
        \\
        \\
        Cominciamo con lo scrivere la funzione di partizione di unaparticella singola
        \begin{equation*}
            Z_1=e^{\beta\mu B}+e^{-\beta\mu B}=2\cosh(\beta\mu B)
        \end{equation*}
        Trovandoci in un solido gli spin sono disposti a reticolo, quindi distinguibili.
        L'energia libera di Helmholtz è
        \begin{equation*}
            F=-k_BT\ln(Z_1^N)=-Nk_BT\ln[2\cosh(\beta\mu B)]
        \end{equation*}
        \\
        \\
        (a) Sapendo che la magnetizzazione si può esprimere come $M=-\frac{1}{\mu}\frac{\partial F}{\partial B}$, trovare la magnetizzazione per spin $m$.
        \\
        \\
        Avendo già l'energia di Helmholtz possiamo procedere con il calcolo
        \begin{equation*}
            \begin{split}
                M&=-\frac{1}{\mu}\frac{\partial F}{\partial B}=\frac{Nk_BT}{\mu}\frac{\partial\ln[2\cosh(\beta\mu B)]}{\partial B}\\
                m&=\frac{k_BT}{\mu}\frac{2\sinh(\beta\mu B)\beta\mu}{2\cosh(\beta\mu B)}=\tanh(\beta\mu B)
            \end{split}
        \end{equation*}
        \begin{center}
            \begin{tikzpicture}
                \begin{axis}[
                    xmin = -3, xmax = 3,
                    ymin = -1.1, ymax = 1.1,
                    samples = 200,
                    smooth,
                    thick,
                    xlabel = {$\beta\mu B$},
                    ylabel = {$m$},]
                    \addplot[
                        domain = -3:3,
                    ] {tanh(x)};
                \end{axis}
            \end{tikzpicture}
        \end{center}
        (b) Trovare la capacità termica a campo magnetico costante e verificare la presenza di un'anomalia di Schottky.
        \\
        \\
        Per il calcolo useremo questa volta la funzione di partizione
        \begin{equation*}
            \begin{split}
                C_V&=\frac{\partial U}{\partial T}=-\frac{\partial}{\partial T}\frac{\partial \ln(Z_1^N)}{\partial \beta}=\\
                &=-N\frac{\partial}{\partial T}\frac{\partial \ln(2\cosh(\beta\mu B))}{\partial \beta}=\\
                &=-N\frac{\partial}{\partial T}\frac{2\sinh(\beta\mu B)\mu B}{2\cosh(\beta\mu B)}=-N\mu B\frac{\partial}{\partial T}\tanh(\beta\mu B)=\\
                &=-N\mu B\frac{(-\mu B)}{k_BT^2\cosh^2(\beta\mu B)}=Nk_B\frac{(\beta\mu B)^2}{\cosh^2(\beta\mu B)}\\
            \end{split}
        \end{equation*}
        \begin{center}
            \begin{tikzpicture}
                \begin{axis}[
                    xmin = 0, xmax = 5,
                    ymin = 0, ymax = 0.5,
                    samples = 200,
                    smooth,
                    thick,
                    xlabel = {$\beta\mu B$},
                    ylabel = {$\frac{C_V}{Nk_B}$},]
                    \addplot[
                        domain = 0:5,
                    ] {x^2/cosh(x)^2};
                \end{axis}
            \end{tikzpicture}
        \end{center}
    
    \section*{Esercizio 5.4 (Kennett)}
        La pressione sul monte Everest è $P_{Ev}=\frac{1}{3}P_{atm}$ con $P_{atm}=101.3kPa$ e la temperatura è $T_{Ev}\simeq 243K$.
        Calcolare il libero cammino medio.
        \\
        \\
        Dalla teoria è nota la formula per il libero cammino medio
        \begin{equation*}
            l=\frac{1}{n\pi d^2}
        \end{equation*}
        Dato che sempre di aria si tratta (sia sull'Everest che al livello del mare) e che non sappiamo $d$, risulta più comodo trovare
        \begin{equation*}
            \frac{l_{Ev}}{l_{atm}}=\frac{n_{atm}}{n_{Ev}}
        \end{equation*}
        Essendo l'aria in buona approssimazione un gas perfetto si può scrivere la densità come $n=\frac{P}{k_BT}$ e, inserendo il tutto nella relazione precedente (assumendo $T_{atm}\simeq300K$)
        \begin{equation*}
            \frac{l_{Ev}}{l_{atm}}=\frac{T_{Ev}P_{atm}}{T_{atm}P_{Ev}}=3\frac{T_{Ev}}{T_{atm}}=2.67
        \end{equation*}
        Dalla teoria è noto $l_{atm}\simeq 1\mu m$ quindi $l_{Ev}\simeq 2.67\mu m$

    \section*{Esercizio 5.5 (Kennett)}
        Nello spazio intersetllare sono presenti giganti nubi di idrogeno molecolare (lunghezza di legame $0.74\times 10^{-10}m$).
        Sapendo che la massa di una nube è $m\sim 4\times 10^{45}kg$, il diametro è $D\sim 1.42\times 10^{18}m$ e la temperatura è $T\sim 10K$ calcolare il libero cammino medio e il tempo di collisione medio.
        \\
        \\
        Calcolando il volume della nube
        \begin{equation*}
            V=\frac{4}{3}\pi\left(\frac{D}{2}\right)^3=\frac{\pi}{6}D^3
        \end{equation*}
        e il numero di molecole $N=\frac{A}{2N_A}$ si ottiene
        \begin{equation*}
            l=\frac{V}{N\pi d^2}=\frac{D^3}{3N_Ad^2}=7.24\times 10^{11}m
        \end{equation*}
        Il tempo di collisione analogamente è
        \begin{equation*}
            \tau=\sqrt{\frac{m}{3k_BT}}l=l\sqrt{\frac{A}{3N_Ak_BT}}=6.48\times 10^{10}s
        \end{equation*}

    \section*{Esercizio 6.4 (Kennett)}
        Calcolare $U$ e $C_V$ per un sistema a due livelli popolato da bosoni.
        \\
        \\
        Dalla teoria è nota la
        \begin{equation*}
            \Xi_b=\prod_s\left[\frac{1}{1-e^{\beta(\mu-\epsilon_s)}}\right]
        \end{equation*}
        Assumiamo i due livelli energetici come
        \begin{equation*}
            \begin{cases}
                \epsilon_0=0\\
                \epsilon_1=\epsilon
            \end{cases}
        \end{equation*}
        e in questo caso la funzione di partizione gran canonica diviene
        \begin{equation*}
            \Xi=\frac{1}{(1-e^{\beta\mu})(1-e^{\beta(\mu-\epsilon)})}
        \end{equation*}
        Dalla definizione di energia interna si può utilizzare il trucco di Feynman
        \begin{equation*}
            \begin{split}
                U&=\frac{1}{\Xi}\sum_{N=0}^\infty\sum_{s}^Ns\epsilon e^{\beta(N\mu-s\epsilon)}=\\
                &=-\frac{1}{\Xi}\frac{\epsilon}{\beta}\sum_{N=0}^\infty\sum_{s}^N\frac{\partial}{\partial\epsilon}e^{\beta(N\mu-s\epsilon)}=\\
                &=-\frac{1}{\Xi}\frac{\epsilon}{\beta}\frac{\partial}{\partial\epsilon}\sum_{N=0}^\infty\sum_{s}^Ne^{\beta(N\mu-s\epsilon)}=\\
                &=-\frac{1}{\Xi}\frac{\epsilon}{\beta}\frac{\partial\Xi}{\partial\epsilon}=\\
                &=-\frac{\epsilon}{\beta}\frac{\partial\ln\Xi}{\partial\epsilon}\\
            \end{split}
        \end{equation*}
        Possiamo ora calcolare l'energia interna
        \begin{equation*}
            \begin{split}
                U&=-\frac{\epsilon}{\beta}\frac{\partial\ln\Xi}{\partial\epsilon}=\\
                &=\frac{\epsilon}{\beta}\frac{\partial\ln(1-e^{\beta\mu})(1-e^{\beta(\mu-\epsilon)})}{\partial\epsilon}=\\
                &=\frac{\epsilon}{\beta}\frac{\partial\ln(1-e^{\beta(\mu-\epsilon)})}{\partial\epsilon}=\\
                &=\frac{\epsilon}{\beta}\frac{\beta e^{\beta(\mu-\epsilon)}}{(1-e^{\beta(\mu-\epsilon)})}=\\
                &=\epsilon\frac{e^{\beta(\mu-\epsilon)}}{(1-e^{\beta(\mu-\epsilon)})}=\\
                &=\frac{\epsilon}{e^{\beta(\epsilon-\mu)}-1}
            \end{split}
        \end{equation*}
        \begin{center}
            \begin{tikzpicture}
                \begin{axis}[
                    xmin = 0, xmax = 2,
                    ymin = -0.1, ymax = 2,
                    samples = 200,
                    smooth,
                    thick,
                    xlabel = {$T$},
                    ylabel = {$\frac{U}{\epsilon}$},]
                    \addplot[
                        domain = 0:2,
                    ] {1/(e^(1/x)-1)};
                \end{axis}
            \end{tikzpicture}
        \end{center}
        Per la capacità termica è sufficiente utilizzare la relazione
        \begin{equation*}
            \begin{split}
                C_V&=\frac{\partial U}{\partial T}=\\
                &=-\frac{\epsilon}{(e^{\beta(\epsilon-\mu)}-1)^2}\frac{\epsilon-\mu}{k_B}\left(-\frac{1}{T^2}\right)e^{\beta(\epsilon-\mu)}=\\
                &=\frac{k_B\beta^2\epsilon(\epsilon-\mu)}{(e^{\beta(\epsilon-\mu)}-1)(1-e^{\beta(\mu-\epsilon)})}
            \end{split}
        \end{equation*}
        \begin{center}
            \begin{tikzpicture}
                \begin{axis}[
                    xmin = 0, xmax = 2,
                    ymin = -0.1, ymax = 1.1,
                    samples = 200,
                    smooth,
                    thick,
                    xlabel = {$T$},
                    ylabel = {$\frac{C_V}{k_B\epsilon}$},]
                    \addplot[
                        domain = 0:2,
                    ] {1/(x^2)*1/(e^(1/x)-1)*1/(1-e^(-1/x))};
                \end{axis}
            \end{tikzpicture}
        \end{center}

    \section*{Esercizio 9.5 (Kennett)}
        (a) Mostrare che la funzione di partizione gran canonica per la radiazione di corpo nero assume la forma
        \begin{equation}
            \Xi=\prod_s\frac{1}{1-e^{-\beta\epsilon_s}}
            \label{equation:9.5a}
        \end{equation}
        \\
        \\
        La radiazione di corpo nero è costituita da fotoni, ossia bosoni.
        Dalla teoria (vista in classe) è nota la funzione di partizione gran canonica bosonica:
        \begin{equation*}
            \Xi_b=\prod_s\frac{1}{1-e^{\beta(\mu-\epsilon_s)}}
        \end{equation*}
        Osservando ora come si possano aggiungere/rimuovere fotoni ad un corpo nero senza alcun costo energetico ($\mu=0$) la \ref{equation:9.5a} diviene ovvia.
        \\
        \\
        (b) Usando l'equazione \ref{equation:9.5a} oppure in altro modo si dimostri che l'entropia per unità di volume della radiazione di corpo nero a temperatura $T$ assume la forma
        \begin{equation}
            s=\frac{4\pi^2k_B^4T^3}{45\hbar^3c^3}
        \end{equation}
        \\
        \\
        Richiamiamo ora l'energia per unità di volume della radiazione di corpo nero (eq. 9.20 Kennett)
        \begin{equation*}
            u=\frac{\pi^2}{15}\frac{(k_BT)^4}{(\hbar c)^3}
        \end{equation*}
        e anche la sua pressione (eq. 9.29 Kennett)
        \begin{equation*}
            P=\frac{u}{3}
        \end{equation*}
        \\
        Richiamando ora la relazione termodinamica
        \begin{equation}
            U=TS-PV-\mu N
        \end{equation}
        ponendo $\mu=0$ (vedi punto a) e dividendo per il volume $V$, si ottiene
        \begin{equation*}
            s=\frac{u+P}{T}
        \end{equation*}
        che unita alle precedenti conclude la dimostrazione
        \begin{equation*}
            s=s=\frac{u+\frac{u}{3}}{T}=\frac{4}{3}\frac{u}{T}=\frac{4\pi^2k_B^4T^3}{45\hbar^3c^3}
        \end{equation*}

    \section*{Esercizio 9.9 (Kennett)}
        Calcolare la temperatura critica per un BEC in 2 dimensioni.
        \\
        \\
        Richiamiamo ora la densità degli stati 2D calcolata in precedenza:
        \begin{equation*}
            g(\epsilon)=g_s\frac{m}{2\pi\hbar^2}L^2
        \end{equation*}
        Possiamo calcolare il numero medio di particelle con la classica formula
        \begin{equation*}
            N=\int_0^\infty g(\epsilon)f(\epsilon)d\epsilon
        \end{equation*}
        la quale, posto $n=\frac{N}{L^2}$ diventa
        \begin{equation*}
            n=g_s\frac{m}{2\pi\hbar^2}\int_0^\infty \frac{1}{e^{\beta(\epsilon-\mu)}-1}d\epsilon
        \end{equation*}
        Per la temperatura critica si ha $e^{\beta_0}>>1$ e si può approssimare
        \begin{equation*}
            n\approx g_s\frac{m}{2\pi\hbar^2}\int_0^\infty e^{-\beta(\epsilon-\mu)}d\epsilon
        \end{equation*}
        la cui soluzione è
        \begin{equation*}
            n=g_s\frac{mk_B}{2\pi\hbar^2}T_0
        \end{equation*}
        Invertendo l'equazione precedente si ottiene la relazione desiderata
        \begin{equation*}
            T_0=\frac{2\pi\hbar^2}{mk_B}\frac{n}{g_s}
        \end{equation*}

    \section*{Esercizio 1 (esame del 13/01/2022)}
        (a) Calcolare la capacitá termica $C_V$ per un sistema di $N$ oscillatori armonici distinguibili, ciascuno con energia $\epsilon_s=\left(n+\frac{1}{2}\right)\hbar\omega$ studiandone i limiti a basse ($k_BT<<\hbar\omega$) e alte ($k_BT>>\hbar\omega$) temperature.
        \\
        \\
        Si può subito calcolare la funzione di partizione dalla definizione
        \begin{equation*}
            \begin{split}
                Z_1&=\sum_se^{-\beta\epsilon_s}=\sum_{n=0}^\infty e^{-\beta\hbar\omega\left(n+\frac{1}{2}\right)}=\\
                &=e^{-\beta\frac{\hbar\omega}{2}}\sum_{n=0}^\infty e^{-\beta\hbar\omega n}=-\frac{e^{-\beta\frac{\hbar\omega}{2}}}{e^{-\beta\hbar\omega}-1}\\
                &=\frac{e^{-\beta\frac{\hbar\omega}{2}}}{1-e^{-\beta\hbar\omega}}=\frac{e^{\beta\frac{\hbar\omega}{2}}}{e^{\beta\hbar\omega}-1}
            \end{split}
        \end{equation*}
        Richiamiamo ora la formula per l'energia interna
        \begin{equation*}
            \begin{split}
                U&=-\frac{\partial\ln(Z_1^N)}{\partial\beta}=-N\frac{\partial}{\partial\beta}\left[\ln(e^{\beta\frac{\hbar\omega}{2}})-\ln(e^{\beta\hbar\omega}-1)\right]=\\
                &=-N\frac{\partial}{\partial\beta}\left[\beta\frac{\hbar\omega}{2}-\ln(e^{\beta\hbar\omega}-1)\right]=\\
                &=-N\left[\frac{\hbar\omega}{2}-\frac{\hbar\omega e^{\beta\hbar\omega}}{e^{\beta\hbar\omega}-1}\right]=\\
                &=-N\hbar\omega\left[\frac{1}{2}-\frac{e^{\beta\hbar\omega}}{e^{\beta\hbar\omega}-1}\right]=\\
                &=N\hbar\omega\left[\frac{1}{1-e^{-\beta\hbar\omega}}-\frac{1}{2}\right]
            \end{split}
        \end{equation*}
        Da qui è possibile ricavare la capacità termica
        \begin{equation*}
            \begin{split}
                C_V&=\frac{\partial U}{\partial T}=N\hbar\omega\frac{\partial}{\partial T}\left[\frac{1}{1-e^{-\beta\hbar\omega}}-\frac{1}{2}\right]=\\
                &=N\hbar\omega\frac{\left(-\frac{\hbar\omega}{k_B}\right)\left(-\frac{1}{T^2}\right)(-e^{-\beta\hbar\omega})}{(1-e^{-\beta\hbar\omega})^2}=\\
                &=-\frac{Nk_B(\beta\hbar\omega)^2(e^{-\beta\hbar\omega})}{(1-e^{-\beta\hbar\omega})^2}=\\
                &=\frac{Nk_B(\beta\hbar\omega)^2}{(e^{\beta\hbar\omega}-1)(1-e^{-\beta\hbar\omega})}
            \end{split}
        \end{equation*}
        \begin{center}
            \begin{tikzpicture}
                \begin{axis}[
                    xmin = 0, xmax = 10,
                    ymin = 0, ymax = 1.1,
                    samples = 200,
                    smooth,
                    thick,
                    xlabel = {$\beta\hbar\omega$},
                    ylabel = {$\frac{C_V}{Nk_B}$},]
                    \addplot[
                        domain = -2:10,
                    ] {x^2*1/(e^x-1)*1/(1-e^(-x))};
                \end{axis}
            \end{tikzpicture}
        \end{center}
        (b) Paragonare graficamente il risultato ottenuto con il caso del sistema a due livelli (con energie $\epsilon_1=-\frac{\hbar\omega}{3}$ e $\epsilon_2=\frac{\hbar\omega}{3}$) evidenziandone analogie e differenze.
        \\
        \\
        Analogamente a prima
        \begin{equation*}
            Z_1=e^{\beta\frac{\hbar\omega}{3}}+e^{-\beta\frac{\hbar\omega}{3}}=2\cosh\beta\frac{\hbar\omega}{3}
        \end{equation*}
        L'energia interna viene
        \begin{equation*}
            U=-N\frac{\hbar\omega}{3}\tanh\beta\frac{\hbar\omega}{3}
        \end{equation*}
        e di conseguenza la capacità termica
        \begin{equation*}
            C_V=Nk_B\frac{\left(\beta\frac{\hbar\omega}{3}\right)^2}{\cosh^2\beta\frac{\hbar\omega}{3}}
        \end{equation*}
        \begin{center}
            \begin{tikzpicture}
                \begin{axis}[
                    xmin = 0, xmax = 8,
                    ymin = 0, ymax = 0.6,
                    samples = 200,
                    smooth,
                    thick,
                    xlabel = {$\beta\hbar\omega$},
                    ylabel = {$\frac{C_V}{Nk_B}$},]
                    \addplot[
                        domain = -2:10,
                    ] {x^2*1/cosh(x)^2};
                \end{axis}
            \end{tikzpicture}
        \end{center}
        (c) \emph{Opzionale} Interpretare l'andamento alle alte T attraverso il teorema dell'equipartizione dell'energia.
        \\
        \\
        Avendo $N$ oscillatori armonici con 2 gradi di libertà quadratici ciascuno con contributo $\frac{1}{2}k_BT$ ritroviamo la capacità termica a $Nk_B$.

    \section*{Esercizio 2 (esame del 13/01/2022)}
        (a) Calcolare l'entropia $S$ per un sistema di $N$ oscillatori armonici distinguibili, spiegando perché allo zero assoluto ($T = 0$) $S = 0$ e studiandone l'andamento (e la dipendenza da $T$) alle alte $T$.
        \\
        \\
        Dalla teoria usiamo la formula dell'entropia
        \begin{equation*}
            \begin{split}
                S&=-\frac{\partial F}{\partial T}=Nk_B\frac{\partial}{\partial T}T\left[\ln e^{\beta\frac{\hbar\omega}{2}}-\ln (e^{\beta\hbar\omega}-1)\right]=\\
                &=Nk_B\frac{\partial}{\partial T}T\left[\beta\frac{\hbar\omega}{2}-\ln (e^{\beta\hbar\omega}-1)\right]=\\
                &=-Nk_B\frac{\partial}{\partial T}\left[T\ln (e^{\beta\hbar\omega}-1)\right]=\\
                &=-Nk_B\left[\ln (e^{\beta\hbar\omega}-1)+T\frac{\left(\frac{\hbar\omega}{k_B}\right)\left(-\frac{1}{T^2}\right)(e^{\beta\hbar\omega})}{e^{\beta\hbar\omega}-1}\right]=\\
                &=Nk_B\left[\frac{\beta\hbar\omega}{1-e^{-\beta\hbar\omega}}-\ln (e^{\beta\hbar\omega}-1)\right]             
            \end{split}
        \end{equation*}
        \begin{center}
            \begin{tikzpicture}
                \begin{axis}[
                    xmin = 0, xmax = 30,
                    ymin = -0.1, ymax = 3,
                    samples = 200,
                    smooth,
                    thick,
                    xlabel = {$T$},
                    ylabel = {$\frac{S}{Nk_B}$},]
                    \addplot[
                        domain = -2:30,
                    ] {1/x*1/1-e^(-1/x)-ln(e^(1/x)-1)};
                \end{axis}
            \end{tikzpicture}
        \end{center}
        Per $T = 0$, $S = 0$ in quanto avendo tutte le particelle allo stesso livello energetico ho solo una disposizione possibile $S=k\ln\Omega=k\ln 1=0$.
        \\
        \\
        (b) \emph{Opzionale} Stimare il numero di stati occupati ad alte temperature.
        \\
        \\
        Non avendo restrizioni sul numero di oscillatori per livello energetico possiamo considerare questi come bosoni.
        Dalla teoria l'occupazione dei bosoni è data dalla
        \begin{equation*}
            \left\langle n_s \right\rangle_b=\frac{1}{e^{\beta\epsilon_s}-1}
        \end{equation*}
        Inserendo nella formula precedente l'espressione dell'energia fornita nell'esercizio 1 e facendo il limite per alte temperature si trova
        \begin{equation*}
            \begin{split}
                \left\langle n_s \right\rangle&=\frac{1}{e^{\beta\hbar\omega\left(s+\frac{1}{2}\right)}-1}\approx\frac{1}{1+\beta\hbar\omega\left(s+\frac{1}{2}\right)-1}=\\
                &=\frac{1}{\beta\hbar\omega\left(s+\frac{1}{2}\right)}=\frac{k_BT}{\hbar\omega\left(s+\frac{1}{2}\right)}
            \end{split}
        \end{equation*}
        \\
        \\
        (c) Paragonare graficamente il risultato ottenuto con il caso di un gas di molecole diatomiche.
        Per il calcolo dell'entropia del gas di molecole diatomiche considerare solo i contributi dovuti al moto traslazionale ($Z_t$) e rotazionale ($Z_r$), considerando che per temperature superiori alla temperatura caratteristica del moto rotazionale $\theta_r$ la funzione $Z_r$ si puó approssimare a $\frac{T}{\theta_r}$.
        \\
        \\
        Ricaviamo innanzitutto la funzione di partizione
        \begin{equation*}
            Z=Z_tZ_r=n_QV\frac{T}{\theta_r}
        \end{equation*}
        Analogamente al punto precedente
        \begin{equation*}
            \begin{split}
                S&=-\frac{\partial F}{\partial T}\approx Nk_B\frac{\partial}{\partial T}\left[T\left(\ln n_QV\frac{T}{\theta_r}-\ln(N) +1\right)\right]=\\
                &=Nk_B\frac{\partial}{\partial T}\left[T\left(\ln\frac{n_Q}{n}+\ln\frac{T}{\theta_r}+1\right)\right]=\\
                &=Nk_B\left(\ln\frac{n_Q}{n}+\ln\frac{T}{\theta_r}+1\right)+Nk_BT\left(\frac{3}{2}\frac{1}{T}+\frac{1}{T}\right)=\\
                &=Nk_B\left(\ln\frac{n_Q}{n}+\ln\frac{T}{\theta_r}+\frac{7}{2}\right)
            \end{split}
        \end{equation*}
        \begin{center}
            \begin{tikzpicture}
                \begin{axis}[
                    xmin = 0, xmax = 2,
                    ymin = -7, ymax = 10,
                    samples = 200,
                    smooth,
                    thick,
                    xlabel = {$T$},
                    ylabel = {$\frac{S}{Nk_B}$},]
                    \addplot[
                        domain = -2:2,
                    ] {3/2*ln(x)+ln(x)+7/2};
                \end{axis}
            \end{tikzpicture}
        \end{center}

    \end{document}