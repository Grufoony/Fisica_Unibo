
% THIS SOFTWARE IS PROVIDED BY THE COPYRIGHT HOLDERS AND CONTRIBUTORS "AS IS"
% AND ANY EXPRESS OR IMPLIED WARRANTIES, INCLUDING, BUT NOT LIMITED TO, THE
% IMPLIED WARRANTIES OF MERCHANTABILITY AND FITNESS FOR A PARTICULAR PURPOSE ARE
% DISCLAIMED. IN NO EVENT SHALL THE COPYRIGHT HOLDER OR CONTRIBUTORS BE LIABLE
% FOR ANY DIRECT, INDIRECT, INCIDENTAL, SPECIAL, EXEMPLARY, OR CONSEQUENTIAL
% DAMAGES (INCLUDING, BUT NOT LIMITED TO, PROCUREMENT OF SUBSTITUTE GOODS OR
% SERVICES; LOSS OF USE, DATA, OR PROFITS; OR BUSINESS INTERRUPTION) HOWEVER
% CAUSED AND ON ANY THEORY OF LIABILITY, WHETHER IN CONTRACT, STRICT LIABILITY,
% OR TORT (INCLUDING NEGLIGENCE OR OTHERWISE) ARISING IN ANY WAY OUT OF THE USE
% OF THIS SOFTWARE, EVEN IF ADVISED OF THE POSSIBILITY OF SUCH DAMAGE.

\documentclass[10pt,a4paper]{article}
\usepackage[T1]{fontenc}
\usepackage[utf8]{inputenc}
\usepackage[italian]{babel}
\usepackage[left=2.5cm, right=2.5cm, top=3cm, bottom=3cm]{geometry}
\usepackage{amsmath}
\usepackage{amsthm}
\usepackage{amssymb}
\usepackage{accents}
\usepackage{interval}
\newtheorem{teorema}{Teorema}[section]
\newcommand{\teor}[2][]{\begin{teorema}[#1]#2\end{teorema}}
\newcommand{\R}{\mathbb{R}}
\newcommand{\N}{\mathbb{N}}
\newcommand{\C}{\mathbb{C}}
\newcommand{\Rbar}{\overline{\mathbb{R}}}
\newcommand{\Lim}[1][]{\xrightarrow[#1]{}}
\renewcommand{\,}{\text{, }}

\title{Anlisi 1}
\author{Nicolò Montalti}
\date{}

\begin{document}
\maketitle
\section{Limiti di successioni}
\teor[Unicità del limite]{Sia $\alpha\, \beta \in \Rbar$.
    \[
        \text{Se } a_n \Lim \alpha, a_n \Lim \beta \implies \alpha = \beta
    \]
}
\teor[TLS Limiti delle sottosuccessioni]{
    Sia $\alpha \in \Rbar$.
    \[
        \text{Se } a_n \Lim \alpha \implies \forall (a_k)_n \text{ s.s di } a_n \text{ si ha }(a_k)_n \Lim \alpha
    \]
}
\teor[TPS Permanenza del segno]{
    Siano $(a_n)\, (b_n)$ s. in $\R$\, $a\,b \in \Rbar$.
    \[
        \text{Se } a_n \Lim a\, b_n \Lim b\, a_n \leq b_n \text{ } \forall n \in \N \implies a \leq  b
    \]
}
\teor[T2C Due carabinieri]{
    Siano $(a_n)\, (b_n)\, (c_n)$ s. in $\R$\, $\lambda \in \R$.
    \[
        \text{Se } a_n \Lim \lambda\, b_n \Lim \lambda\, a_n \leq b_n \leq c_n \text{ } \forall n \in \N \implies b_n \Lim \lambda
    \]
}
\teor[TSM Successioni monotone]{
    \begin{align*}
        \text{Se } (a_n) \nearrow & \implies \exists \lim_{n\to\infty} a_n = \sup_{n \in \N} (a_n) \\
        \text{Se } (a_n) \searrow & \implies \exists \lim_{n\to\infty} a_n = \inf_{n \in \N} (a_n)
    \end{align*}
}
\teor[Criterio del rapporto]{
    Sia $a_n > 0$ s. in $\R \, \frac{a_{n+1}}{a_n} \Lim \lambda \in \Rbar$.
    \begin{align*}
        \text{Se } \lambda > 1 & \implies a_n \Lim +\infty \\
        \text{Se } \lambda < 1 & \implies a_n \Lim 0
    \end{align*}
}
\teor[CSE Continuità sequenziale dell'esponenziale]{
    Sia $x_n$ s. in $\R$.
    \[
        \text{Se } x_n \Lim x_0 \in \R \implies e^{x_n} \Lim e^{x_0}
    \]
}
\teor[CSL Continuità sequenziale del logaritmo]{
    Sia $x_n$ s. in $\R$.
    \[
        \text{Se } x_n \Lim x_0 \in \R \implies log_a(x_n) \Lim log_a(x_0)
    \]
}
\teor[Criterio di Cesaro]{
    Siano $a_n\, b_n$ s. in $\R\, b_n \nearrow \text{s.}\, b_n \Lim +\infty$.
    \[
        \text{Se } \exists \lim_{n\to\infty}\frac{a_{n+1} - a_n}{b_{n+1}-b_n} = \lambda \in \Rbar \implies \lim_{n\to\infty}\frac{a_n}{b_n} = \lambda
    \]
}
\section{Limiti di funzioni}
Siano $A \subseteq \R^N \, f:A \rightarrow \R^M \, \alpha \in D(A) \, \lambda\, \mu \in \R^M (\bar{\R} \text{ se } M = 1)$
\teor[Unicità del limite]{
    \[
        \text{Se } f(x) \Lim[x \to \alpha] \lambda\, f(x) \Lim[x \to \alpha] \mu \implies \lambda = \mu
    \]
}
\teor[TLR Limite delle restrizioni]{
    Sia $B \subseteq A\, \alpha \in D(B)$.
    \[
        \text{Se } f(x) \Lim[x \to \alpha] \lambda \implies f|_B(x) \Lim[x \to \alpha] \lambda
    \]
    dove $f|_B: B \rightarrow \R^M \, f|_B(x) := f(x)$
}
\teor[CSL Caratterizzazione sequenziale del limite]{
    \[
        f(x) \Lim[x \to \alpha] \lambda \iff f(x_n) \Lim \lambda \quad \forall (x_n) \text{ s. in } A \smallsetminus {\alpha} \, x_n \to \alpha
    \]
}
\teor[TLM Limite delle funzioni monotone]{
    Sia $x_0 \in \R^M\, f \nearrow$
    \begin{align*}
        \text{Se } x_0 \in D(A_{x_0}^-) & \implies f(x) \Lim[x \to x_0^-] \sup_{A_{x_0}^-}f \\
        \text{Se } x_0 \in D(A_{x_0}^+) & \implies f(x) \Lim[x \to x_0^+] \inf_{A_{x_0}^-}f \\
        \text{Se } +\infty \in D(A)     & \implies f(x) \Lim[x \to +\infty] \sup_{A}f       \\
        \text{Se } -\infty \in D(A)     & \implies f(x) \Lim[x \to -\infty] \inf_{A}f
    \end{align*}
    dove $A_{x_0}^- := A \text{ }\cap \text{ } \interval[open]{-\infty}{ x_0} \, A_{x_0}^+ := A \text{ }\cap \text{ } \interval[open]{x_0}{ +\infty}$. Un risultato analogo vale per $f \searrow$
}
\section{Continuità}
\teor[CSC Caratterizzazione sequenziale della continuità]{Sia $x_0 \in A$.
    Siano $A \subseteq \R^N \, f:A \to \R^M \, x_0 \in A$.
    \[
        f \text{ è continua in } x_0 \iff f(x_n) \Lim f(x_0) \quad \forall x_n \to x_0
    \]
}
\teor[Bolzano]{
    Siano $a\, b \in \R \text{ con } a < b\, f \in C(\interval{a}{b}, \R)$.
    \[
        \text{Se } f(a)f(b) \leq 0 \implies \exists x_0 \in \interval{a}{b} : f(x_0) = 0
    \]
}
Sia $I$ un intervallo di $\R$.
\teor[TVI Valore intermedio]{
    Sia $f \in C(I,\R)$.
    \begin{align*}
        \text{Dato } y \in \R : f(a) < f(y) < f(b) & \implies \exists x_0 \in I : f(x_0) = y \\
        f(I)                                       & \text{ è un intervallo}
    \end{align*}
}
\teor{
    Sia $f \in C(I, \R)\, f \text{ 1-1}$, posto $J = f(I)$. Allora
    \begin{align*}
         & J \text{ è un intervallo, } f \xrightarrow[\text{su}]{1-1} J \\
         & f^{-1} \in C(J)
    \end{align*}
}
\teor[Weierstrass]{
    Sia $A \in \R$ chiuso e limitato, $f \in C(A, \R) \implies \exists \max_A f \, \exists \min_A f$

}
\section{Derivate}
\teor[Fermat]{
    Sia $f:A \to \R \, x_0 \in \accentset{\circ}{A}\, f'(x_0) \in \R$.
    \[
        \text{Se } x_0 \text{ è un p.to estremante locale} \implies f'(x_0) = 0
    \]
}
\teor[Rolle]{
    Siano $a,b \in \R \, a < b \, f \in C(\interval{a}{b}, \R) \, f$ derivabile in $\interval[open]{a}{b}$.
    \[
        \text{Se } f(a) = f(b) \implies \exists x_0 \in \interval[open]{a}{b}: f'(x_0) = 0
    \]
}
\teor[TVML Valor medio di Lagrange]{
    Siano $a,b \in \R \, a < b \, f \in C(\interval{a}{b}, \R) \, f$ derivabile in $\interval[open]{a}{b}$.
    \[
        \text{Allora } \exists x_0 \in \interval[open]{a}{b} : f'(x_0) = \frac{f(b) - f(a)}{b-a}
    \]
}
\teor[TM Test monotonia]{
    Sia $I$ un intervallo su $\R$, $f:I \to \R$ derivabile in $I$.
    \begin{align}
        f'(x) = 0 \quad \forall x \in I             & \iff f \text{ è costante in } I \\
        f'(x) \ge 0 \quad           \forall x \in I & \iff f \nearrow                 \\
        f'(x) > 0 \quad           \forall x \in I   & \implies f \nearrow s.          \\
        f \nearrow s.                               & \iff
        \begin{cases}
             & f'(x) \ge 0 \quad \forall x \in I \\  &\text{int}(\{x \in \R / f'(x) = 0\}) = \emptyset
        \end{cases}
    \end{align}
}
\teor[Taylor con resto di Peano]{
    Siano $I$ intervallo di $\R \, x_0 \in I \, f$ derivabile n volte in $x_0$. Allora
    \[
        f(x) = \sum_{k = 0}^{n} \frac{f^{(k)} (x-x_0)^k}{k!} + o((x-x_0)^n)_{x \to x_0}
    \]
}
\teor[Tayor con resto di Lagrange]{
    Siano $I$ intervallo aperto di $\R \, x_0 \in I \, f$ derivabile n volte in $x_0$. Allora $\forall x \in I \quad \exists y_x \in (x_0,x)$ tale che
    \[
        f(x) = T_{n-1}(x) + \frac{f^{(n)}(y_x)}{n!}(x-x_0)^{n}
    \]
}
\teor[TH De l'Hôpital]{
    Siano $I$ intervallo di $\R \, x_0 \in D(I) \, f,g: I \setminus \{x_0\} \to \R $ derivabili in $I \setminus \{ x_0 \}$ con $g'(x) \neq 0$ $ \forall x \in I \setminus \{x_0\}$. $f(x) \, g(x) \Lim[x \to x_0] 0$ o $|g(x)| \Lim[x \to x_0] +\infty$. Allora
    \[
        \text{se } \frac{f'(x)}{g'(x)} \Lim[x \to x_0] \lambda \in \Rbar \implies \frac{f(x)}{g(x)} \Lim[x \to x_0] \lambda
    \]

}
\teor{
    Sia I intervallo non banale di $\R \, f : I \to \R$ derivabile in I. Allora $f$ è convessa $\iff f' \nearrow$
}
\teor{
    Sia $f : I \to \R$ derivabile due volte in I. Allora $f$ è convessa $\iff f''(x) \ge 0 \quad \forall x \in I$
}
\teor{
    Sia $f \in C^{n+1}(I, \R)$, $x_0 \in \accentset{\circ}{I}$ con $f^{(k)}(x_0) = 0$ per $k \le n$ e $f^{(n+1)}(x_0) \ne 0$. Allora
    \begin{align*}
         & x_0 \text{ è un punto di massimo locale forte di f se (n+1) è pari e } f^{(n+1)}(x_0) < 0 \\
         & x_0 \text{ è un punto di minimo locale forte di f se (n+1) è pari e } f^{(n+1)}(x_0) > 0  \\
         & x_0 \text{ non è un punto estremante se (n+1) è dispari}
    \end{align*}
}
\section{Integrali}
\teor{
    Sia $f : \interval{a}{b} \to \R$ limitata e $c \in \interval[open]{a}{b}$.
    \begin{align*}
        \text{Se } f \in C(\interval{a}{b}, \R)                                         & \implies f \in R_{\interval{a}{b}} \\
        \text{Se } f \text{ è monotona }                                                & \implies f \in R_{\interval{a}{b}} \\
        \{ x \in  \interval{a}{b} / f \text{ non è continua in } x \} \text{ è finito } & \implies f \in R_{\interval{a}{b}} \\
        f \in R_{\interval{a}{c}} \, f \in R_{\interval{c}{b}}                          & \iff f \in R_{\interval{a}{b}}
    \end{align*}
}
\teor{
    Siano $f,g \in R_{\interval{a}{b}}$. Allora
    \begin{align*}
        \text{ Se } f(x) \le g(x) \quad \forall x \in \interval{a}{b} & \implies \int_a^b f(x)dx \le \int_a^b g(x)dx \\
        \left | \int_a^b f(x) dx \right |                             & \le \left | \int_a^b |f(x)| dx \right |
    \end{align*}
}
\teor[1TFC Primo teorema fondamentale del calcolo]{
    Siano $f \in R_{\interval{a}{b}} \, F(X) = \int_a^x f(t) dt : \interval{a}{b} \to \R$. Allora
    \begin{align*}
        F                              & \in C(\interval{a}{b}, \R) \\
        \text{Se f è continua in } x_0 & \implies f(x_0) = F'(x_0)
    \end{align*}
}
\teor[2TFC Secondo teorema fondamentale del calcolo]{
    SIano $f \in R_{\interval{a}{b}}$ e $\Phi$ una primitiva di $f$. Allora
    \[
        \int_a^b f(t) dt = \left [ \Phi(t) \right ]_a^b = \Phi(b) - \Phi(a)
    \]
}
\teor[Taylor con resto integrale]{
    Sia $f \in C^{n+1}(\interval{a}{b}, \R) \, x_0 \in \interval{a}{b}$. Allora
    \[
        f(x) = T_n(x) + \frac{1}{n!} \int_{x_0}^x (x -t)^n f^{(n+1)}(t) dt
    \]
}

\section{Integrali generalizzati}
Siano $a \in \R \, \beta \in \Rbar \, a < \beta \,  f \in C(\mathclose [ a, \beta \mathopen [, \R)$.
\teor[C$\exists$ Criterio di esistenza]{
\[
    \text{Se } f(x) \ge 0 \quad \forall x \in \mathclose [ a, \beta \mathopen [ \implies \exists \int_a^{\beta}f(x)dx \in \interval{0}{+\infty}
\]
}
\teor[CC Criterio del confronto]{
Sia $f(x) \ge 0 \, g(x) \ge 0 \, \forall x \in \mathclose [ a, \beta \mathopen [ $.
\[
    \text{Se } f(x) = \mathcal{O}_{x \to \beta ^ -}(g(x)) \, \int_a^{\beta} g \in \R \implies \int_a^{\beta} f \in \R
\]
}
\teor[CCA Criterio di convergenza assoluta]{
    \[
        \text{Se } \int_a^{\beta} \left | f \right | \in \R \implies \exists \int_a^{\beta} f \in \R
    \]
}
\teor[CFO Criterio delle funzioni oscillanti]{
Sia $f \in C(\mathclose [ a, \beta \mathopen [, \R) : $ f abbia una primitiva limitata in $\mathclose [ a, \beta \mathopen [$, $g \in C^{1}(\mathclose [ a, \beta \mathopen [, \R) : g \searrow_{x \to \beta^-} 0$. Allora
\[
    \exists  \int_a^{\beta} fg \in \R
\]
}
\section{Serie}
\teor[CZ Criterio zero]{
    Sia $a_n$ succcessione in $\R$.
    \[
        \text{Se } \sum_{k=1}^{+\infty} a_k \text{ conv. } \implies a_n \Lim[n \to \infty] 0
    \]
}
\teor[C$\exists$ Criterio di esistenza]{
    Sia $a_n$ succcessione in $\R \, a_n \ge 0 \quad \forall n \in \N$. Allora
    \[
        \sum_{k=1}^{+\infty} a_k \text{ conv.}
    \]
}
\teor[CI Criterio integrale]{
    Sia $f \in C(\mathopen [ 1, +\infty \mathclose ]) \, f \searrow \, f \ge 0$. Allora
    \[
        \sum_{k=1}^{+\infty} f(k) \text{ conv. } \iff \int_1^{+\infty} f(x) dx \text{ conv.}
    \]
}
\teor[CAD Criterio di Abel-Dirichlet]{
    Siano $a_n$ successione in $\R$, $b_n$ succession in $\C$. Allora
    \[
        \text{se } a_n \searrow 0 \, \sum_{k=1}^{N} b_k \text{ è limitata} \implies \sum_{k=1}^{+\infty} a_k b_k \text{ conv.}\]
}
\teor[CL Criterio di Leibnitz]{
    Siano $a_n$ successione in $\R \, a_n \searrow 0$. Allora
    \[
        \sum_{k = 1}^{+\infty} (-1)^k a_k \text{ conv.} \]
}
\end{document}

