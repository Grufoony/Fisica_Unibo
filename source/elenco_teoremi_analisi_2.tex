
% THIS SOFTWARE IS PROVIDED BY THE COPYRIGHT HOLDERS AND CONTRIBUTORS "AS IS"
% AND ANY EXPRESS OR IMPLIED WARRANTIES, INCLUDING, BUT NOT LIMITED TO, THE
% IMPLIED WARRANTIES OF MERCHANTABILITY AND FITNESS FOR A PARTICULAR PURPOSE ARE
% DISCLAIMED. IN NO EVENT SHALL THE COPYRIGHT HOLDER OR CONTRIBUTORS BE LIABLE
% FOR ANY DIRECT, INDIRECT, INCIDENTAL, SPECIAL, EXEMPLARY, OR CONSEQUENTIAL
% DAMAGES (INCLUDING, BUT NOT LIMITED TO, PROCUREMENT OF SUBSTITUTE GOODS OR
% SERVICES; LOSS OF USE, DATA, OR PROFITS; OR BUSINESS INTERRUPTION) HOWEVER
% CAUSED AND ON ANY THEORY OF LIABILITY, WHETHER IN CONTRACT, STRICT LIABILITY,
% OR TORT (INCLUDING NEGLIGENCE OR OTHERWISE) ARISING IN ANY WAY OUT OF THE USE
% OF THIS SOFTWARE, EVEN IF ADVISED OF THE POSSIBILITY OF SUCH DAMAGE.

\documentclass[10pt,a4paper, twocolumn]{article}
\usepackage[T1]{fontenc}
\usepackage[utf8]{inputenc}
\usepackage[italian]{babel}
\usepackage[left=2cm, right=2cm, top=3cm, bottom=3cm]{geometry}
\usepackage{amsmath}
\usepackage{amsthm}
\usepackage{amssymb}
\usepackage{accents}
\usepackage{interval}
\newtheorem{teorema}{Teorema}[section]
\newcommand{\teor}[2][]{\begin{teorema}[#1]#2\end{teorema}}
\newcommand{\R}{\mathbb{R}}
\newcommand{\N}{\mathbb{N}}
\newcommand{\C}{\mathbb{C}}
\newcommand{\Rbar}{\overline{\mathbb{R}}}
\newcommand{\Lim}[1][]{\xrightarrow[#1]{}}
\renewcommand{\,}{\text{, }}

\title{Definizioni e teoremi di Analisi 2}
\author{Nicolò Montalti}
\date{}

\begin{document}
\maketitle

\section{Spazi metrici}
\begin{itemize}
    \item Definizione di spazi metrici
    \item Metriche $\delta, d_1, d_2, d_{\infty}$ in $\R^n$
    \item Metriche $d_1, d_2, d_{\infty}$ in $C([0,1])$
    \item Disuguaglianza di Minkowski
    \item Def intorno circolare aperto
    \item Def insieme aperto, chiuso, compatto e connesso
    \item Def topologia
    \item Unione e intersezione di aperti e chiusi
    \item Funzione distanza, diametro di un insieme
    \item Def intorno di un punto
    \item Chiusura, punti aderenti, interno, punti interni e frontiera
    \item Teorema di Bolzano-Weirstrass
    \item Def insieme denso
\end{itemize}

\section{Spazi metrici completi}
\begin{itemize}
    \item Def successione e convergenza
    \item Unicità del limite
    \item Def successione si Cauchy
    \item Def spazio metrico completo
    \item Completezza dei sottospazi $(B([a,b],\R), D_{\infty}$ e $C([a,b],\R), d_{\infty})$ completi
    \item Densità di $C([a,b],\R), d_1)$ in $L^1$ e $C([a,b],\R), d_2)$ in $L^2$
    \item Principio di Cantor ("scolapasta")
    \item Def isometria e completamento 
    \item Def funzione continua (def equivalenti) e uniformemente continua
    \item Def funzione lipschitziana
    \item Def contrazione e teorema delle contrazioni
\end{itemize}

\section{Spazi di Banach e di Hilbert}
\subsection{Spazi di Banach}
\begin{itemize}
    \item Def spazio vettoriale e spazio normato
    \item Metrica indotta da una norma
    \item Def insieme limitato
    \item Def spazio di Banach
\end{itemize}
\subsection{Spazi di Hilbert}
\begin{itemize}
    \item Def prodotto interno reale e complesso
    \item Spazi $l_2(\R), C([a,b],\R), L^2$
    \item Disuguaglianza di Cauchy-Schwarz
    \item Regola del parallelogramma
    \item Def spazio di Hilbert
    \item Def ortogonalità e teorema di Pitagora
    \item Spazio ortogonale e teorema delle proiezioni
\end{itemize}

\section{Differenziabilità}
\begin{itemize}
    \item Def funzione differenziabile
    \item Continuità e derivabilità di una funzione differenziabile
    \item Derivata direzionale
    \item Def differenziabilità due volte
    \item Teorema di Schwarz
    \item Def differenziabilità per una funzione a valori vettoriali
    \item Jacobiana della composta e dell'inversa
    \item \dots massimi e minimi \dots
\end{itemize}

\section{Varietà}
\begin{itemize}
    \item Def varietà regolare
    \item Def spazio tangente e spazio normale
    \item Teorema di Dini
    \item Def estremanti condizionati e punti critici
    \item Teorema di Fermat
    \item Moltiplicatori di Lagrange, condizione necessaria e sufficiente
\end{itemize}

\section{Misura e integrazione}
\subsection{Misura di Peano-Jordan}
\begin{itemize}
    \item Def intervallo semiaperto superiormente
    \item Def plurintervallo e sua Misura
    \item Proprietà di $\mu_n$
    \item \dots
\end{itemize}

\subsection{Integrale secondo Riemann}
\begin{itemize}
    \item Def integrale di una funzione non negativa
    \item Def integrale di una funzione di segno variabile
    \item Proprietà dell'integrale su insiemi di misura nulla
    \item Approssimazione con valore assoluto
    \item Misura del grafico e integrabilità
    \item Teorema della media integrale
    \item Linearità, monotonia e addittività
    \item Sommabilità per funzioni continue e limitate
    \item Sommabilità di funzioni $f:A\setminus{x_0} \rightarrow \R$
    \item Def dominio misurabile
    \item Teoremi di riduzione: integrali doppi su intervalli e domini normali
    \item Teorema di Cavalieri
    \item Teorema del cambiamento di variabile
\end{itemize}

\section{Curve e integrali curvilinei}
\begin{itemize}
    \item Def curva in forma parametrica
    \item Def curve e parametrizzazioni regolari, semplici, aperte e chiuse
    \item Def omemomorfismo e diffeomorfismo
    \item Teorema del cambiamento di parametrizzazione
    \item Def curva regolare a tratti
    \item Def punto di arresto
    \item Def orientamento di una curva
    \item Regolarità, semplicità e orientabilità
    \item Orientamento per cambi di parametrizzazione
    \item Def parametrizzazione r.a.t orientabile
    \item Def lunghezza di una curva e curva rettificabile
    \item Teorema rettificabilità di curve regolari
    \item Lunghezza di una curva per cambi di parametrizzazione
    \item Ascissa curvilinea
    \item Def lavoro e di campo conservativo
    \item Proprietà del campo potenziale
    \item Def campo irrotazionale e legame con conservatività
    \item Campi conservativi e lavoro
    \item Condizioni sufficienti perché un campo sia conservativo
    \item Forme esatte e chiuse
    \item Def insieme convesso, stelato e semplicemente connesso
    \item Lemma di Poincarè
    \item Teorema "della circonferenza"
\end{itemize}

\section{Superfici e integrali di\\superficie}
\begin{itemize}
    \item Def aperto regolare in $\R^2$
    \item Def superficie, supericie regolare, semplice e regolare con bordo
    \item Def orientamento di una superficie e cambi di parametrizzazione
    \item Area di una superficie
    \item Area di una superficie di rivoluzione
    \item Def flusso di un campo vettoriale e cambio di parametrizzazione
    \item Def orientamento canonico di un aperto regolare
    \item Orientamento canonico indotto da una parametrizzazione
    \item Def superficie regolare a tratti
    \item Teorema di Stokes
    \item Formula di Gauss-Green
    \item Teorema di Stokes per le forme differenziabilità
    \item Def aperto regolare in $\R^3$
    \item Teorema della divergenza
    \item Generalizzazione dei teoremi di Stokes e della divergenza alle k-forme in $\R^n$
\end{itemize}
\end{document}