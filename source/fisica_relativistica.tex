\documentclass[a4paper]{article}
\usepackage[T1]{fontenc}
\usepackage[utf8]{inputenc}
\usepackage[main=italian, english]{babel}
\usepackage{bookmark}
\usepackage[a4paper, total={6in, 9in}]{geometry}
\usepackage{hyperref}
\usepackage{amsmath}

\begin{document}
	\title{Formulario Relatività Ristretta}
	\author{Grufoony}
	\maketitle
    \begin{abstract}
        Se non diversamente specificato, i due sistemi di riferimento sono presi tali per cui S sia fermo e S' si muova a velocità $\vec{V}=V\hat{x}$.
        c indica sempre la velocità della luce. Le grandezze $x_0$ indicano misurazioni effettuate nel sdr con il corpo a riposo.
    \end{abstract}
    \section{Cinematica relativistica}
        \begin{itemize}
            \item $\beta=\frac{v}{c}$
            \item $\gamma=\frac{1}{\sqrt{1-\beta^2}}$
            \item $L(v)=\frac{L_0}{\gamma}$ contrazione di Lorentz-FitzGerald
            \item $m(v)=\gamma m_0$ ipotesi di Lorentz sulla massa
            \item $t'=\gamma t$ dilatazione dei tempi
            \item Trasformazioni di Lorentz:\\
                $$\begin{cases}
                    x'=\gamma(x-Vt)\\
                    y'=y\\
                    z'=z\\
                    t'=\gamma\left(t-\frac{V}{c^2}x\right)
                \end{cases}$$
            \item Trasformazioni di Lorentz (velocità):\\
                $$\begin{cases}
                    u_x'=\frac{u_x-V}{1-\frac{u_xV}{c^2}}\\
                    u_y'=\frac{u_y}{\gamma\left(1-\frac{u_xV}{c^2}\right)}\\
                    u_z'=\frac{u_z}{\gamma\left(1-\frac{u_xV}{c^2}\right)}
                \end{cases}$$
            \item Trasformazioni di Lorentz (accelerazioni):\\
                $$\begin{cases}
                    a_x'=\frac{a_x}{\gamma^3\left(1-\frac{u_xV}{c^2}\right)^3}\\
                    a_y'=\frac{a_y}{\gamma^2\left(1-\frac{u_xV}{c^2}\right)^2}+\frac{\frac{u_yV}{c^2}}{\gamma^2\left(1-\frac{u_xV}{c^2}\right)^3}a_x\\
                    a_z'=\frac{a_z}{\gamma^2\left(1-\frac{u_xV}{c^2}\right)^2}+\frac{\frac{u_zV}{c^2}}{\gamma^2\left(1-\frac{u_xV}{c^2}\right)^3}a_x
                \end{cases}$$
            \item $\beta_p'=\sqrt{1-\frac{(1-\beta^2)(1-\beta_p^2)}{(1-\beta\beta_{px})^2}}$ con $\beta_{px}=\frac{v_x}{c}$
            \item $\gamma_p'=\gamma\gamma_p\left(1-\frac{\vec{v_p}\cdot\vec{V}}{c^2}\right)$
            \item $N(t)=N_0e^{-\frac{t}{\tau_0}}$ decadimento particellare
            \item $\nu=\nu_0\frac{\sqrt{1-\beta^2}}{1+\beta\cos{\theta}}$ effetto Doppler con $\beta>0$ in allontanamento
        \end{itemize}
    \section{Dinamica Relativistica}
        \begin{itemize}
            \item $\vec{p}=m(v)\vec{v}=\gamma m_0\vec{v}$ impulso relativistico
            \item $E(v)=\mathcal{T}+m_0c^2$ energia relativistica
            \item $E^2=m_0^2c^4+p^2c^2$
            \item $\vec{p}=\frac{E}{c^2}\vec{v}$
            \item In un sistema di N particelle valgono:\\
                $$\begin{cases}
                    \sum_{i=1}^{N}E_i=E_{TOT}\\
                    \sum_{i=1}^{N}\vec{p}_i=\vec{p}_{TOT}
                \end{cases}$$
            \item Per un fotone $m_0=0\Rightarrow E=h\nu=\frac{hc}{\lambda}; p=\frac{E}{c}=\frac{h}{\lambda}$ 
            \item Trasformazioni dell'impulso:\\
                $$\begin{cases}
                    p_x'=\gamma\left(p_x-\frac{\beta}{c}E\right)\\
                    p_y'=p_y\\
                    p_z'=p_z\\
                    E'=\gamma(E-c\beta p_x)
                \end{cases}$$
            \item $\vec{F}=\frac{d}{dt}(\gamma m_0\vec{v})$ Forza
            \item $\vec{F}=F_\perp+F_\|=m_0\gamma\vec{a}+m_0\gamma^3\left(\vec{a}\cdot\frac{\vec{v}}{c}\right)\frac{\vec{v}}{c}$
            \item Trasformazioni della forza:\\
                $$\begin{cases}
                    F_x'=F_x-\frac{Vu_y}{c^2\left(1-\frac{Vu_y}{c^2}\right)}F_y-\frac{Vu_z}{c^2\left(1-\frac{Vu_z}{c^2}\right)}F_z\\
                    F_y'=\frac{F_y}{\gamma\left(1-\frac{Vu_x}{c^2}\right)}\\
                    F_z'=\frac{F_z}{\gamma\left(1-\frac{Vu_x}{c^2}\right)}
                \end{cases}$$
            \item $\Delta\lambda=\lambda'-\lambda=\frac{h}{m_0c}(1-\cos{\theta})$ effetto Compton
            \item $\omega_c=\frac{qB}{m_0\gamma}$ frequenza angolare di ciclotrone
            \item $r_c=\gamma\frac{m_0{v_0}_y}{qB}$ raggio di ciclotrone
            \item $E=\frac{{E'}^2}{2m_pc^2}$ scontro tra protone fermo/in moto. E energia protone in moto. E' energia finale del sistema
        \end{itemize}
    \section{Elettromagnetismo}
        \begin{itemize}
            \item Maxwelliaml:\\
                $$\begin{cases}
                    \vec{E'}=\gamma\vec{E}-(\gamma-1)\frac{\vec{V}}{V^2}(\vec{V}\cdot\vec{E})+\gamma(\vec{V}\wedge\vec{B})\\
                    \vec{B'}=\gamma\vec{B}-(\gamma-1)\frac{\vec{V}}{V^2}(\vec{V}\cdot\vec{B})+\frac{\gamma}{c^2}(\vec{V}\wedge\vec{E})
                \end{cases}$$\\
                $$\begin{cases}
                    \vec{E_\|'}=\vec{E'}\\
                    \vec{E_\perp'}=\gamma(\vec{E}+\vec{V}\wedge\vec{B})_\perp\\
                    \vec{B_\|'}=\vec{B'}\\
                    \vec{B_\perp'}=\gamma(\vec{B}-\frac{\vec{V}\wedge\vec{E}}{c^2})_\perp
                \end{cases}$$
            \item Trasformazioni campo elettrico:
                $$\begin{cases}
                    E_x'=E_x\\
                    E_y'=\gamma(E_y-vB_z)\\
                    E_z'=\gamma(E_z+vB_y)
                \end{cases}$$
            \item Trasformazioni campo magnetico:
                $$\begin{cases}
                    B_x'=B_x\\
                    B_y'=\gamma\left(B_y+\frac{v}{c^2}E_z\right)\\
                    B_z'=\gamma\left(B_z-\frac{v}{c^2}E_y\right)
                \end{cases}$$\\
        \end{itemize}
	\section{Il fantastico mondo di Minkowski}
        \begin{itemize}
            \item $ds^2=dx^2+dy^2+dz^2-c^2dt^2$ spazio di Minkowski
            \item $ds^2>0$ intervallo \textit{time-like}: esiste un sdr in cui gli eventi avvengono nello stesso punto (in tempi diversi).
                Possibile una reazione causa effetto tra i due
            \item $ds^2=0$ intervallo \textit{light-like}: la luce può viaggiare tra i due eventi. Relazione causa-effetto solo tramite segnali luminosi
            \item $ds^2<0$ intervallo \textit{space-like}: esiste un sdr in cui gli eventi avvengono simultaneamente (in posti diversi)
            \item $x^\mu=(ct,x,y,z)=(ct,\vec{r})$ quadrivettore posizione
            \item $ds^2=g_{\mu\nu}dx^\mu d x^\nu$ notazione tensoriale, con $g_{\mu\nu}$ tensore metrico
            \item $x_\mu=g_{\mu\nu}x^\nu$
            \item $v^\mu=(\gamma c, \gamma v_x, \gamma v_y, \gamma v_z)=(\gamma c, \gamma\vec{v})$ quadrivettore velocità
            \item $a^\mu=\gamma(\dot{\gamma}c, \dot{\gamma}\vec{v}+\gamma\vec{a})$ quadrivettore accelerazione
            \item $a^\mu=(0, \vec{\alpha})$ accelerazione propria
            \item $p^\mu=m_0v^\mu=(\gamma m_0c, \gamma m_0\vec{v})$ quadrivettore impulso (ogni componente si conserva separatamente)
            \item $f^\mu=\frac{dp^\mu}{d\tau}=\left(\frac{\gamma}{c}\vec{F}\cdot\vec{v}, \gamma\vec{F}\right)$ quadrivettore forza
            \item $\mu=\sqrt{\frac{1+\beta^2}{1-\beta^2}}$ fattore di scala unità spaziale di S'
        \end{itemize}
    \section{Invarianti relativistici}
        \begin{itemize}
            \item $m_0$
            \item $E^2-c^2p^2$
            \item $ds^2=c^2dt^2-dx^2-dy^2-dz^2$
        \end{itemize}
    \section{Appendice}
        \begin{itemize}
            \item $\alpha=\arctan{\beta}$ aberrazione classica
            \item $\alpha=\arcsin{\beta}$ aberrazione relativistica
            \item $\Delta\alpha\simeq\frac{1}{2}\beta^3$
            \item $\cos\theta=\frac{v_x}{c}=\frac{\cos\theta'+\beta}{1+\beta\cos\theta'}$ relativistic beaming, con $\theta'$ mezzo angolo di emissione
            \item se $\vec{F}\|\vec{v}$ allora $\vec{F}=m_0\gamma^3\vec{a}$
            \item se $\vec{F}\perp\vec{v}$ allora $\vec{F}=m_0\gamma\vec{a}$
            \item in un moto rettilineo $\vec{\alpha}=\gamma^3\vec{a}$
            \item in un moto circolare $\alpha=\gamma^2\frac{v^2}{R}$
            \item per un fotone $E=pc=h\nu$
            \item energia cinetica massima per una particella accelerata $\mathcal{T}=\frac{q^2B^2R^2}{2m}$ con R raggio acceleratore
        \end{itemize}
\end{document}