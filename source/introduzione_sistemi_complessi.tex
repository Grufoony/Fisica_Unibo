\documentclass[12pt, a4paper]{article}
\usepackage[T1]{fontenc}
\usepackage[utf8]{inputenc}
\usepackage[main=italian]{babel}
\usepackage{bookmark}
\usepackage[a4paper, total={6in, 9in}]{geometry}
\usepackage{amsmath, amsthm, amssymb, amsfonts}

\theoremstyle{theorem}
\newtheorem{definition}{Definizione}[section]
\newtheorem{theorem}{Teorema}[section]

\begin{document}
	\title{Appunti dal corso Introduzione ai Sistemi Complessi}
	\author{Grufoony}
	\maketitle
	\section{Sistema Complesso}
		\begin{definition}
			\textit{Sistema Complesso} è un sistema dinamico composto da sottosistemi interagenti tra loro.
		\end{definition}
		Per lo studio di un sistema complesso si usa solitamente un approccio olistico, ossia studiando prevalentemente le proprietà macroscopiche del sistema totale, senza considerare i singoli sottosistemi.
		Un'osservazione importante che va effettuata è che un sistema complesso \textbf{prevede}, non descrive.
		Alcune delle proprietà principali sono:
		\begin{itemize}
			\item \textbf{complessità}: presenza di molti d.o.f. (molti sottosistemi)
			\item \textbf{proprietà emergenti}: derivano dal grande numero di sottosistemi. Ad esempio possiamo definire \textit{fluido} un insieme di molte particelle ma la particella singola non può essere fluida.
			\item \textbf{autorganizzazione}: i sistemi complessi sono ibridi, ossia metà stocastici e metà deterministici. Per studiarli devo dare ugual peso a entrambi gli aspetti.
		\end{itemize}
	\section{Distribuzioni}
		Vediamo ora una serie di distribuzioni e teoremi ad esse legati che ci aiuteranno nell'analisi dei sistemi.
		\begin{definition}\hfill
			\begin{itemize}
				\item Gaussiana\\	$\rho(x)=\frac{1}{\sqrt{2\pi}\sigma}e^{-\frac{(x-\mu)^2}{2\sigma^2}}$
				\item Esponenziale\\	$\rho(x)=\frac{1}{k}e^{-\frac{x}{k}}$
				\item Potenza\\		$\rho(x)\propto\frac{1}{x^a}$, con $a>0$
			\end{itemize}
		\end{definition}
		\begin{definition}
			Momenti di una distribuzione:\\
			$<x^k>=\int_{-\infty}^{+\infty}x^k\rho(x)dx$
		\end{definition}
		\begin{theorem}
			Invarianza di scala:\\
			se $\rho(x)\propto\frac{1}{x^a}$ allora posto $y=\lambda x$ si ha $\rho(y)=\frac{\lambda^a}{x^a}\propto\frac{1}{y^a}$
		\end{theorem}
		\begin{theorem}
			Limite centrale:\\
			Siano ${x_k}$ variabili casuali indipendenti, allora:\\
			$\lim_{N\to\infty}z=\frac{1}{\sqrt{N}}\sum_{k=1}^{N}x_k=\frac{1}{\sqrt{2\pi}\sigma}e^{-\frac{z^2}{2\sigma^2}}$
		\end{theorem}
		Ora possiamo dare una definizione di probabilità:
		\begin{definition} Probabilità:\\
			$p(x\in[a,b])=\int_{a}^{b}\rho(x)dx$
		\end{definition}
		\begin{definition} Probabilità cumulata:\\
			$p(x\leq a)=\int_{-\infty}^{a}\rho(x)dx$
		\end{definition}
	\section{Costruzione di un modello}
		Punto fondamentale di un sistema complesso è costruire un modello matematico che riesca a riprodurre le sue caratteristiche fondamentali, per poi studiarlo.
		La prima cosa da definire è l'\textit{ambiente} in cui ci troviamo. Questo può essere neutro o avere caratteristiche, ad esempio una distribuzione di nutrimento (per sistemi biologici).
		Altro punto fondamentale è definire \textit{spazio e tempo}. Spesso non fa differenza la scelta di spazi e tempi discreti rispetto ai continui, quindi è preferibile assumere una discretizzazione iniziale per poi passare al continuo successivamente.
		Una volta definito lo spazio bisogna poi decidere le condizioni al contorno, ossia il comportamento ai bordi. 
		Posso a questo punto avere \textit{barriere} di tre tipi:
		\begin{itemize}
			\item \textbf{riflettenti}, dove ho un bordo \textit{non} oltrepassabile. Si crea quindi un fenomeno di \textbf{attrattività delle pareti}.
			\item \textbf{periodico}, dove ho i bordi coincidenti (esco da una parte e rientro dall'altra). Lo spazio assume in questo caso una forma toroidale.
			\item \textbf{assorbenti}, dove gli oggetti "uscenti" vengono distrutti. In questo caso bisogna di introdurre delle \textit{sorgenti} nel modello per evitare di perdere tutti i soggetti.
		\end{itemize}
		Si nota facilmente come più piccolo sia il modello, più importante sia il contributo degli effetti di bordo.\hfill
		Nella maggior parte dei sistemi non tutti i soggetti hanno le stesse caratteristiche: si definiscono allora \textbf{classi} di appartenenza, legate tra loro da relazioni matematiche.
		

\end{document}