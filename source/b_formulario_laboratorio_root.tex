
% Copyright (c) 2022, Grufoony
% All rights reserved.

% THIS SOFTWARE IS PROVIDED BY THE COPYRIGHT HOLDERS AND CONTRIBUTORS "AS IS"
% AND ANY EXPRESS OR IMPLIED WARRANTIES, INCLUDING, BUT NOT LIMITED TO, THE
% IMPLIED WARRANTIES OF MERCHANTABILITY AND FITNESS FOR A PARTICULAR PURPOSE ARE
% DISCLAIMED. IN NO EVENT SHALL THE COPYRIGHT HOLDER OR CONTRIBUTORS BE LIABLE
% FOR ANY DIRECT, INDIRECT, INCIDENTAL, SPECIAL, EXEMPLARY, OR CONSEQUENTIAL
% DAMAGES (INCLUDING, BUT NOT LIMITED TO, PROCUREMENT OF SUBSTITUTE GOODS OR
% SERVICES; LOSS OF USE, DATA, OR PROFITS; OR BUSINESS INTERRUPTION) HOWEVER
% CAUSED AND ON ANY THEORY OF LIABILITY, WHETHER IN CONTRACT, STRICT LIABILITY,
% OR TORT (INCLUDING NEGLIGENCE OR OTHERWISE) ARISING IN ANY WAY OUT OF THE USE
% OF THIS SOFTWARE, EVEN IF ADVISED OF THE POSSIBILITY OF SUCH DAMAGE.

\documentclass[a4paper]{article}
\usepackage[T1]{fontenc}
\usepackage[utf8]{inputenc}
\usepackage[main=italian, english]{babel}
\usepackage{bookmark}
\usepackage[a4paper, total={6in, 9in}]{geometry}
\usepackage{hyperref}
\usepackage[dvipsnames]{xcolor}
\definecolor{LightGray}{gray}{0.85}

\begin{document}
	\title{Guida esame ROOT}
	\author{Russo Alessandro, Antonelli Tommaso, Berselli Gregorio\\\url{https://github.com/Grufoony/Fisica_UNIBO}}
    \maketitle
\begin{abstract}
    Le notazioni utilizzate in questo schema sono $\vec{x}$ per rappresntare un generico array x, dim indica sempre una dimensione, xCode indica un codice di root inerente a x, opt indica sempre la stringa di caratteri che ROOT interpreta come opzioni, i indica un indice (o una quantit\'a indicizzabile), n indica sempre un intero.
\end{abstract}
\section{Istogrammi}
    \colorbox{LightGray}{TH1F("name", "title", nBins, xmin, xmax)} Dichiarazione di un istogramma 1D
    \begin{itemize}
        \item \colorbox{LightGray}{Fill(x)} o \colorbox{LightGray}{Fill(x, noccorrenze)} fill del bin x 
        \item \colorbox{LightGray}{Draw("opt")} per disegnare il grafico (E-barre errore, P-solo punti, SAME-sul grafico precedente)
        \item \colorbox{LightGray}{GetMean()}
        \item \colorbox{LightGray}{GetMeanError()}
        \item \colorbox{LightGray}{GetRMS()} radice della varianza
        \item \colorbox{LightGray}{GetRMSError()}
        \item \colorbox{LightGray}{GetMaximum()} o \colorbox{LightGray}{GetMinimum()}
        \item \colorbox{LightGray}{GetEntries()} numero di ingressi inseriti
        \item \colorbox{LightGray}{Integral()} come GetEntries ma non tiene conto dei pesi assegnati
        \item \colorbox{LightGray}{Integral(bin1, bin2)} integrale tra bin1 e bin2
        \item \colorbox{LightGray}{GetIntegral()} l'array dei conteggi cumulativi
        \item \colorbox{LightGray}{Sumw2()} imposta la somma in quadratura degli errori
    \end{itemize}
    Metodi dei bin:
    \begin{itemize}
        \item \colorbox{LightGray}{GetMaximumBin()} indice del bin contenente il valore massimo
        \item \colorbox{LightGray}{GetBinCenter(ibin)} centro del bin
        \item \colorbox{LightGray}{GetBinContent(ibin)} numero di elementi contenuti nel bin
        \item \colorbox{LightGray}{GetBinContent(0)} numero di UNDERFLOW
        \item \colorbox{LightGray}{GetBinContent(nbin+1)} numero di OVERFLOW
        \item \colorbox{LightGray}{GetBinError(ibin)}
        \item \colorbox{LightGray}{SetBinContent(ibin, val)} in y prende il valore val
    \end{itemize}
    Estetica:
    \begin{itemize}
        \item \colorbox{LightGray}{SetMarkerStyle(markerCode)}
        \item \colorbox{LightGray}{SetMarkerSize(dim)} setta la dimensione del marker a dim
        \item \colorbox{LightGray}{SetFillColor(colorCode)}
        \item \colorbox{LightGray}{SetLineColor(colorCode)}
        \item \colorbox{LightGray}{GetXaxis()->SetTitle("name")} setta il titolo dell'asse x (analogo per y)
        \item \colorbox{LightGray}{GetYaxis()->SetTitleOffset(dim)} offset del titolo
        \item \colorbox{LightGray}{GetYaxis()->SetTitleSize(dim)} dimensione carattere titolo
        \item \colorbox{LightGray}{SetLineWidth(dim)} spessore linea
    \end{itemize}
    Operazioni (sia h1 un istogramma):
    \begin{itemize}
        \item \colorbox{LightGray}{n*h1} prodotto per lo scalare n
        \item \colorbox{LightGray}{Divide(const *h1, const *h2, c1, c2, "opt")} esegue la divisione (c1*h1)/(c2*h2). Di solito si richiede l'opzione "B"
        \item \colorbox{LightGray}{Add(const *h1, const *h2, c1, c2, "opt")} esegue la somma c1*h1+c2*h2
    \end{itemize}
\section{Grafici}
    \colorbox{LightGray}{TGraph(npoints, $\vec{x}$, $\vec{y}$)} 
    \colorbox{LightGray}{TGraph(filename, *format="\%lg \%lg", "opt")} da file
    \colorbox{LightGray}{TGraphErrors(n, x, y, ex, ey)} grafico con barre errore
    \colorbox{LightGray}{TGraphErrors(filename, format="\%lg \%lg \%lg \%lg", "opt")}
    \begin{itemize}
        \item \colorbox{LightGray}{Draw("opt")} metodo per disegnare (A-disegna gli assi, P-disegna il marker dei punti, E-disegna barre errore)
        \item \colorbox{LightGray}{SetPoint(i, x, y)} inserisce il punto (x,y) nella posizione i
        \item \colorbox{LightGray}{GetPoint(i, x, y)} assegna alle variabili (x,y) i valori del punto i
        \item \colorbox{LightGray}{GetX()} returna $\vec{x}$ (analogo per y)
        \item \colorbox{LightGray}{GetN()} returna il numero dei punti
        \item \colorbox{LightGray}{Integral()} returna l'area sotto il grafico
        \item \colorbox{LightGray}{GetCorrelationFactor()}
        \item \colorbox{LightGray}{GetCovariance()}
    \end{itemize}
    Estetica:
    \begin{itemize}
        \item \colorbox{LightGray}{SetTitle("title", "titleXaxis", "titleYaxis")}
        \item \colorbox{LightGray}{SetMarkerStyle(markerCode)}
        \item \colorbox{LightGray}{SetMarkerColor(colorCode)}
        \item \colorbox{LightGray}{SetLineColor(colorCode)}
        \item \colorbox{LightGray}{SetFillColor(colorCode)}
    \end{itemize}
\section{Funzioni}
    \colorbox{LightGray}{TF1("name", "function", xmin, xmax)} nella funzione i parametri vanno indicati con [i]\\
    NB: per funzioni definite a tratti, siano f(x) e g(x) funzioni generiche
    \begin{verbatim}
        "function" = "f(x)*(x>=a && x<b) + g(x)*(x>=b && x<c) + ..."
    \end{verbatim}
    \begin{itemize}
        \item \colorbox{LightGray}{Draw()}
        \item \colorbox{LightGray}{SetParameter(i, value)} setta il valore value al parametro iesimo
        \item \colorbox{LightGray}{SetParameters(value1, value2, ..., valuen)} setta il valore value i al parametro i
        \item \colorbox{LightGray}{SetParLimits(i, xmin, xmax)} definisce il range del parametro i
        \item \colorbox{LightGray}{Eval(x)} returna f(x)
        \item \colorbox{LightGray}{Integral(a, b)} integrale della funzione nel range [a,b]
        \item \colorbox{LightGray}{DrawDerivative()} grafica la derivata prima
        \item \colorbox{LightGray}{DrawIntegral()} grafica l’integrale
        \item \colorbox{LightGray}{Derivative(x)} o \colorbox{LightGray}{Derivative2(x)} o \colorbox{LightGray}{Derivative3(x)} calcola derivate
    \end{itemize}
\section{Fitting}
    Metodo \colorbox{LightGray}{Fit("name", "opt")} valido per pi\'u classi. Funzioni predefinite "gaus" "expo" "poln"
    OptFit:
    \begin{itemize}
        \item "R": usa il range della funzione (di default usa il range dell'istogramma)
        \item "L": usa il metodo della massima verosimiglianza (default con chi-quadro)
        \item "Q": stampa il minimo necessario dei risultati del fit
        \item "V": stampa tutto quello che riesce dei risultati del fit
        \item "S": ritorna risultati utili del fit
    \end{itemize}
    Metodi:
    \begin{itemize}
        \item \colorbox{LightGray}{gStyle->SetOptFit(optCode)} visualizza le info del fit
        \item \colorbox{LightGray}{GetFunction(“f”)} restituisce la funzione utilizzata nel fit (va utilizzata su un oggetto fittato)
        \item \colorbox{LightGray}{GetChisquare()}
        \item \colorbox{LightGray}{GetNDF()}
        \item \colorbox{LightGray}{GetParameter(i)} valore del parametro i-esimo
        \item \colorbox{LightGray}{GetParError(i)}
        \item \colorbox{LightGray}{GetParameters($\vec{par}$)} mette in $\vec{par}$ i parametri
        \item \colorbox{LightGray}{GetParErrors($\vec{epar}$)}
        \item \colorbox{LightGray}{GetCorrelationMatrix()} stampabile tramite Print()
        \item \colorbox{LightGray}{GetCovarianceMatrix()} stampabile tramite Print()
    \end{itemize}
\section{Random}
    \colorbox{LightGray}{gRandom->SetSeed()} per inizializzare la generazione
    Distribuzioni predefinite:
    \begin{itemize}
        \item \colorbox{LightGray}{Uniform(a, b)} uniforme in [a, b]
        \item \colorbox{LightGray}{Rndm()} uniforme in [0, 1]
        \item \colorbox{LightGray}{Gaus($\mu$, $\sigma$)}
        \item \colorbox{LightGray}{Poisson($\mu$)}
        \item \colorbox{LightGray}{Binomial(ntot, prob)}
        \item \colorbox{LightGray}{Exp($\tau$)} esponenziale decrescente
        \item \colorbox{LightGray}{Integer(i)} uniforme intera in [0, i-1]
        \item \colorbox{LightGray}{Landau(moda, $\sigma$)}
        \item \colorbox{LightGray}{f1->GetRandom()} genera un valore casuale dalla funzione f1
        \item \colorbox{LightGray}{FillRandom("name", n)} filla l'oggetto con n elementi della funzione "name"
    \end{itemize}
\section{Canvas}
    \colorbox{LightGray}{TCanvas("name", "title", "width", "height")}
    \begin{itemize}
        \item \colorbox{LightGray}{Divide(i, j)} divide la canvas in i*j pad
        \item \colorbox{LightGray}{cd(i)} seleziona la pad i
        \item \colorbox{LightGray}{Print("name.extension")} stampa la canvas su file
    \end{itemize}
    \subsection{Legenda}
    \colorbox{LightGray}{TLegend(x1, y1, x2, y2, "name")}
    \begin{itemize}
        \item \colorbox{LightGray}{SetFillColor(n)}
        \item \colorbox{LightGray}{AddEntry(obj, "description")} associa a obj una linea con descrizione
        \item \colorbox{LightGray}{Draw("SAME")} disegna la legenda sullo stesso grafico
    \end{itemize}
\section{List}
    \colorbox{LightGray}{TList()}
    \begin{itemize}
        \item \colorbox{LightGray}{Add(obj)} con obj qualunque oggetto di root (TH1F, TF1, ...)
        \item \colorbox{LightGray}{At(i)} returna l'elemento i-esimo della lista. NB: returna sempre un TObject, al momento dell'utilizzo specificarne il tipo con un static\textunderscore cast<type>
        \item \colorbox{LightGray}{Print()} stampa a schermo la lista
        \item \colorbox{LightGray}{At(i)->InheritsFrom("type")} restituisce 1 se l'elemento i-esimo \'e di tipo type, 0 altrimenti
    \end{itemize}
\section{ROOT Files}
    \colorbox{LightGray}{TFile("name", "opt")}
    Options:
    \begin{itemize}
        \item \colorbox{LightGray}{NEW} o \colorbox{LightGray}{CREATE} crea un nuovo file aperto in lettura, se gi\'a esiste non si apre
        \item \colorbox{LightGray}{RECREATE} come CREATE ma se il file esiste lo sovrascrive
        \item \colorbox{LightGray}{UPDATE} apre un file esistente in scrittura, se non esiste lo crea
        \item \colorbox{LightGray}{READ} apre un file esistente in lettura (default)
    \end{itemize}
    Metodi:
    \begin{itemize}
        \item \colorbox{LightGray}{TObject->Write()} scrive sul file il TObject
        \item \colorbox{LightGray}{Get("name")} returna l'oggetto salvato di nome "name"
        \item \colorbox{LightGray}{Close()} chiude il file. Fondamentale inserirlo per evitare problemi
    \end{itemize}
\section{Benchmark}
    Metodi:
    \begin{itemize}
        \item \colorbox{LightGray}{gBenchmark->Start("name")} inizia un processo "name"
        \item \colorbox{LightGray}{gBenchmark->Print("name")} stampa le tempistiche in quel momento
        \item \colorbox{LightGray}{gBenchmark->Show("name")} interrompe il processo e ne stampa le tempistiche
        \item \colorbox{LightGray}{gBenchmark->Stop("name")} interrompe il processo "name"
    \end{itemize}
\section{Esempi}
    \subsection{Lettura da file}
    \begin{verbatim}
        ifstream in;
        in.open("maxwell.dat"); //nome del file
        Float_t x,y;
        while (1) {
        in >> x >> y;
        if (!in.good()) break;
        h1->Fill(y);
        }
        in.close();
    \end{verbatim}
    \subsection{Scrivere una classe su un ROOT File}
    In "MyClass.h"\\
    Mettere inheritance: public TObject\\
    Chiamare macro in fondo (prima della chiusura delle parentesi: ClassDef(MyClass, 1))\\

    \begin{verbatim}
        class MyClass: public TObject {
        ...
        ClassDef(MyClass, 1) //rende la classe scrivibile sul file ROOT
        }; 

        In "MyClass.cxx"
        Chiamare macro in fondo: ClassImp(MyClass)

        Se hai fatto questo poi per scriverla sul file: //con A oggetto di MyClass
        1) TFile *file = new TFile("test.root","recreate"); 
        2) A.Write("A"); 
        3) file->Close();
    \end{verbatim}
    NB: per usare una classe esterna da ROOT:\\
    \begin{verbatim}
        gROOT->LoadMacro("MyClass.cxx++"); //compiling MyClass
    \end{verbatim}
    \subsection{Risoluzione o Smearing}
        \subsubsection{A valore fisso}
        \begin{verbatim}
            for(Int_t i=0;i<nGen;i++){
                h[0]->Fill(gRandom->Gaus(fixedValue, res));
            }
        \end{verbatim}
        \subsubsection{Con una distribuzione}
        \begin{verbatim}
            for(Int_t i=0;i<nGen;i++){
                h[1]->Fill(gRandom->Gaus(distribution, res));
            }
        \end{verbatim}
    \subsection{Efficienza}
        \subsubsection{Costante}
        \begin{verbatim}
            (efficenza del 70%)
            x=gRandom->Gaus(mean, sigma);
            y=gRandom->Rndm();
            if(y<0.7){h1->Fill(x)}
        \end{verbatim}
        \subsubsection{Variabile}
        \begin{verbatim}
            TF1 *eff = new TF1(“eff”, "function", min, max);
            x=gRandom->Gaus(mean, sigma);
            y=gRandom->Rndm();
            if(y < eff->Eval(x)) {
                h1->Fill(x)
            }
        \end{verbatim}
        \subsubsection{Istogramma dell'efficienza}
        \begin{verbatim}
            TH1F *hEff=new Th1F(*hTot); //hTot istogramma con tutte le x(non solo quelle scartate)
            hEff->Divide(h1, hTot, 1, 1, “B”); //h1 istogramma con solo le x accettate
        \end{verbatim}
    \subsection{Compilazione}
        \subsubsection{da ROOT}
        \begin{verbatim}
            .L macro.cpp oppure root [0] gROOT->LoadMacro(“macro.cpp”)
            macro() [per eseguire]
        \end{verbatim}
        \subsubsection{da SHELL}
        \begin{verbatim}
            g++ -o ExampleMacro .exe macro.cpp `root-config --cflags --libs`
            ./macro.exe [per eseguire]  //IMPORTANTE: la macro deve contenere un main()
        \end{verbatim}

\end{document}