\documentclass[11pt]{article}
\usepackage[T1]{fontenc}
\usepackage[utf8]{inputenc}
\usepackage[main=italian, english]{babel}
\usepackage{amsmath}
\usepackage{listings}
\usepackage{xcolor}
\usepackage{tabularx}
\usepackage{array}
\usepackage{caption}
\usepackage[export]{adjustbox}
\usepackage{float}
\usepackage{moresize}

\begin{document}
\title{\LARGE{\textbf{Guida di root per l'esame}}}
\author{\Large{Grufoony}}
\maketitle
\section{Funzioni}
Per creare una funzione con root basta utilizzare il costruttore:
TF1("nome","funzione",Xmin, Xmax)
\begin{itemize}
    \item Xmin - Xmax rappresenta il dominio della funzione;
    \item nell'espressione della funzione possono essere esplicitati parametri del tipo [i], dove la i rappresenta l'i-esimo parametro. Per settarne il valore utilizzare SetParameter(i, value);
    \item per disegnare la funzione Draw();
\end{itemize}
Alternativamente:
TF1("nome","funzione", Xmin, Xmax, Nparametri)
SetParameters(value1, value2, ..., valueN)
\end{document}